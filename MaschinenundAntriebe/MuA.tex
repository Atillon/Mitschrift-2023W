\documentclass[a4paper]{article}

\usepackage[a4paper,top=2cm,bottom=2cm,left=3cm,right=3cm,marginparwidth=1.75cm]{geometry}
\usepackage[utf8]{inputenc}
\usepackage[T1]{fontenc}
\usepackage{textcomp}
\usepackage[ngerman]{babel}
\usepackage{amsmath, amssymb, nccmath}
\usepackage{accents}

% figure support
\usepackage{import}
\usepackage{xifthen}
\pdfminorversion=7
\usepackage{pdfpages}
\usepackage{transparent}
\newcommand{\incfig}[1]{%
    \def\svgwidth{\columnwidth}
    \import{./figures/}{#1.pdf_tex}
}

\pdfsuppresswarningpagegroup=1

\title{Mitschrift\\Maschinen und Antriebe Wintersemester 2023}
\author{DINC Atilla (11917652)}

\begin{document}
\normalsize
\maketitle
%\tableofcontent\newpage

% ~~~~~~~~~~~~~~~~~~~~~~~~~~~~ Start of the document ~~~~~~~~~~~~~~~~~~~~~~~~~~~~
\subsection*{18.10.2023}

\section*{Wiederholung der letzten Woche}
Wendepol hilft dabei, dass die Stromumkehr schneller stattfinden kann.
Weiters: Erregerwicklung, Ankerpolwicklung, \ldots\newline
Siehe Abb 2.23 zur Verdeutlichung der Wirkung  der Wendepolwicklung.
Die Zeitspanne die zur Auslegung der Wendepolwicklung notwendig ist, ergibt sich durch die Zeit, beim Wechsel zwischen der 2 kurzgeschlossenen Ankerpolwicklungen.
Da die Kompensationswicklung auch schon zur Senkung der Flussdichte bei der Kommutierung hilft, ist neben einer Kompensationswicklung eine weniger starke Wendepolwicklung notwendig.
\section*{2.2. Bauformen}
Heutiger Trend: gestanzte Maschinen. 
Man beachte Situationen, bei denen bei niedrigen Drehzahlen, hohe Leistungen übertragen werden.
In solchen Fällen muss ein Zwangslüfter verbaut werden.
Beispiel: Aufzug der ohne Bremse stillsteht
Gegenbeispiel: Pumpe, die geringen Fluss fördert

Abb. 2.25: welchen Dauermagnet würdest du erwarten? Ferrit oder Neodym, Eisen Bohr?
Man stelle sich vor, wie man die Feldlinien schließen muss:
    wenige Pole -> Verhältnis zwischen Pol-Joch-Länge ist hoch -> nicht stark genug? keine Ahnung hab ich nicht verstanden
daher muss es Ferrit sein

Abb. 2.26: a) eher Ferritmagnet b) eher Hochenergiemagnet (neodym, eisen bohr)

Abb. 2.27: logisch: eine Schleife muss immer etwa die Breite eines Hauptpols haben
Anwendungen: Sonderbauform für Fahrräder, Scooter und einige weitere

Abb. 2.28: nur die Wicklung dreht sich
Vorteil:
\begin{itemize}
\item sehr geringes Trägheitsmoment -> sehr hohe Dynamik
\item wenig Rippelmoment, da die Oberfläche eben ist
\item kleine Induktivitäten
\end{itemize}

\subsection*{Magnetkreisauslegung}
Verdeutlichung am Beispiel Abb. 2.30, siehe S64-S66 im Skriptum. Weiters siehe TUWEL-Aufgaben.

Kennlinie Dauermagnet:
B-H Kennlinie $B(H)=\frac{\Delta B}{\Delta H}$ 
Als Analogon kann das Verhalten eines einfachen Stromkreises betrachtet werden: Batterie + Widerstand.
Kennlinie Batterie:
U-I Kennlinie:
$U(I)$ konstante Spannung bis sich  $I$ an  $I_{max}$ annähert.
 Wenn der Strom steigt, steigt die Spannung, bis der Strom zu groß ist und nicht mehr geliefert werden kann.

 Anwendung des Durchflutungssatzes: $-2.H_{Luft}.\delta_{Luft}=H_{magnet}.\delta_{Magnet}$
 Quellenfreiheit: $B_{Magnet}.A_{Magnet}=B_{Luft}.A_{Luft}$

 $B_{Luft}=µ_0.H_{Luft}$
 $H_{Luft}=\frac{B_{Luft}.A_{Magnet}}{µ_0.A{Luft}}$ 
 Durch einsetzen in den Durchflutungssatz folgt:
 $\frac{B_{Magnet}}{H_{Magnet}}=-\frac{d_{Magnet}.µ_0.A_{Luft}}{2.A_{Magnet}.\delta_{Luft}}$
  Man sieht, dass dieses Ergebnis eine negative Steiung hat, während unsere B(H) Kennlinie vom Anfang (siehe auch Abb 2.29) eine positive Steigung hat. Aus dem Schnittpunkt kann ein Arbeitspunkt bestimmt werden.
Notiz: Die Kennlinie des Dauermagnet muss nicht zwingend linear sein.

\subsection*{2.3.1 Induzierte Spannung}
Erinnerung: Verbraucherzählpfielsystem (VZPS) - Strom fließt aus der Quelle raus

Nach dem VZPS würde der Schleifenstrom in Abb. 2.32 einen Fluss in die Zeichenebene hinein erzeugen.
    ->Flüsse in die Zeichenebene hinein (Kreuze) müssen somit positiv und Flüsse aus der Ebene heraus (Punkte) müssen negative gezählt werden.
Siehe Skriptum S67 und S68 für Rechenschritte. Herleitung gibt uns beschreibt die Flussverkettungsänderung über die Zeit.

\subsection*{2.3.2 Kraft aus einen Leiter}
Diesmal wird eine Stromquelle an den Klemmen vorgestellt.
Abb.: Stromquelle als Trafo dargestellt parallel zur Schleife mit Klemmenspannung $U_K$

$\Delta t . P=U_{13}.I.\Delta t$
$\Delta W = U_{13}.I.\Delta t$
Wenn die magnetische Arbeit kosntant bleibt, muss die gesamte elektrische Arbeit in mechanische Arbeit umgewandelt werden.
Aus $\Phi = const$ und $I =const$ folgt  $L=const$ und daher auch  $W_{mag}=L^2.\frac{I}{2}=const$ 
->$\Delta W=U_{13}.I.\Delta t = W_{mag} + W_{mech} = W_{mech}$

\subsection*{2.3.3 Kraft auf Leiter in Nuten und im Luftspalt}
siehe Abb. Tafelbild: quelle speist eine Schleife im Rotor, das Rotorfeld schließt sich über den Luftspalt + Anker
siehe Diagram. Tafelbild: 
$U=\frac{d\Phi}{dt}$ 
$I=H.\delta .2$
 $B=µ_0.H$
  $\Phi=B.A=B.\frac{r.\pi.l}{2}=µ_0.\frac{I}{2\delta}.\frac{r.\pi.l}{2}=L.I$ 
  $U=\frac{d\Phi}{dt}=L.\frac{dI}{dt}$ 

$I.u_r=\frac{d\Phi}{dt}.I$ 
$I.u.dt=I.d\Phi$ 
$dW=I.d\Phi$
 $A=L.\frac{I^2}{2}$ 
 siehe Abbildung: Tafelbild: Diagram -> Joch -> Zoom
siehe Abbildung: Tafelbild: Überlegungen mit Freilaufdioden
siehe Abbildung: Tafelbild: Umrechnung auf Drehmoment
Man beachte, das ist nur die eine Schleife, keine Erregung oder sonst etwas. Daher ist das die magnetische Energie, die die Schleife in diese Anordnung hinein.

Wir sehen: Das Drehmoment ist dort sehr hoch, wo die Energieänderung sehr hoch ist. Man betrachte $M=\frac{\Delta W}{\Delta \gamma}$
Abb. 2.33 sehr unverständlich erklärt.

\section*{2.3.4 Gesamte an den Klemmen wirksame induzierte Spannung}
Wurde nicht durchbesprochen aber Gl. 2.29 wurde betont.

\section*{2.3.5  Gesamtes in der Maschine wirksames Drehmoment}
Gleichung 2.33 ist ebenfalls wichtig.

\section*{Zusammenfassung Gleichungen der GSM}
\begin{itemize}
    \item Doppelte Drehzahl -> doppelte Spannung
    \item doppelter Fluss -> doppelte \ldots
    /item etc.
Das ist sehr wichtig.
\end{itemize}

\section*{2.4 Wichtige Schaltungen der GSM}
\subsection*{2.4.1 Permanenterregte GSM}
\subsubsection*{Flussverkettungsgleichung}
Man bedenke auch$ k\Phi n=k\Phi \frac{\Omega}{2\pi} =k'\Phi\Omega$.
\begin{equation}\label{2.39} \text{ist eine lineare Differentialgleichung} \end{equation}

\subsubsection*{Mechanische Gleichung}
zu 2.45 und 2.46: einfach nachlesen

\subsubsection*{Spannungsgesteuerte PM-GSM}
Die spannungsgesteuerte GSM gibt die Drehzahl vor.
allgemein - Übertragungsfunktion: $f(s)=\frac{x_{a}(s)}{x_{e}(s)}$\newline
aus 2.45 und 2.46 kommt man auf 2.47.\newline
Interpretation:
\begin{itemize}
    \item Nullstellen des Nenners bestimmen die Stabilität
\end{itemize}
In Abbildung 2.35 gilt als Zwischengrösse: $\frac{1}{\Theta_{m}}=\frac{d\Omega_{M}}{dt}$, weshalb nach dem Integrator die Ausgangsgrösse $\Omega_{M}$ erscheint.

Fazit: Eine spannungsgesteuerte GSM ist alles andere als hochdynamisch. Die Spannungs muss behutsam hochgefahren werden (siehe Eingangszweig $\frac{1}{R_{A}}\to \frac{L_{A}}{R_{A}}\to const$)

\boxed{Empfohlene Übung: rechnen Sie 2.48 nach und untersuchen Sie die Schwingungsbedingungen.}

\subsubsection*{Stationärverhalten der Spannungssteuerung}
Abb 2.37 wurde ausführlich besprochen:
\begin{itemize}
    \item n-Verhalten bei Last
    \item Strom-Spannungsverhalten in unterschiedlichen Situationen
    \item Alle quadranten abfahren (Hinweis: Generatorbetrieb und Bremsbetrieb)
\end{itemize}

\subsubsection*{Einfluss einer stossförmigen Laständerung}
Die Pole einer Störübertragungsfunktion sind die gleichen, wie die einer Führungsübertragungsfunktion. Sind also die der Führung stabil, so sind auch die der Störung stabil.\newline
Wenn der Bruch aus Gleichugn 2.53 in 2 Brüche gespalten wird, ist der $R_{A}$-Bruch ein normaler PT2 und der $s\cdot L_{A}$-Bruch ist ein gedämpfter Differenzierer DT2.

Die Steuerungen in Abb. 2.38 sind sehr simpel aber nicht sehr hochdynamisch.

\subsection*{Stromgesteuerte PM-Gleichstrommaschine}
Die stromgesteuerte Maschine prägt das Drehmoment ein.\newline
Für das Lastverhalten kann man die mechanische Gleichung $\Theta \dot\Omega=m_{i}=k'\Phi I$ betrachten.


Bis Seite 84 besprochen.

\section*{22.11.2023}
\section*{Fremderregte Gleichstrommaschine}
\subsection*{Aufbau und Schaltbild}
\subsubsection*{Feldschwächbereich}


\subsection*{Bezogene Werte}
Frage: Ist das bezogene Moment negativ definiert? Wirkt nämlich laut Abb. 2.46 so.

\subsection*{Kennlinien und 4-Quadrantenbereich}
Die Maschine wird im Feldschwächbereich "weicher". Sei fällt eher in die Knie als sonst.
\subsubsection*{Ankerrückwirkung}
Der Fluss über einen Pol hinweg sollte konstant (homogen) sein. Durch die erzeugten Flüsse um die einzelnen Ankerwicklungen wird der Polfluss lokal gesenkt/angehoben. Das eine Ende des Ankers weist einen geringeren Fluss und das andere Ende einen höheren Fluss auf.
\[ B=B_{0}\implies B(\phi)=B_{0}+(\phi-\phi_{0})k \]
\[ k\text{\ldots konstante als Geradensteigung} \]
\[ \phi_{0}\text{\ldots Ankerdrehwinkel zum Mittelpunkt des Polschuhs}\]

\subsubsection*{Stabile und instabile Arbeitspunkte}

\section*{2.4.3 Die Gleichstrom-Nebenschlussmaschine}



\end{document}
