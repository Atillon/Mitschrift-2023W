\documentclass[a4paper]{article}

\usepackage[a4paper,top=2cm,bottom=2cm,left=3cm,right=3cm,marginparwidth=1.75cm]{geometry}
\usepackage[utf8]{inputenc}
\usepackage[T1]{fontenc}
\usepackage{textcomp}
\usepackage[ngerman]{babel}
\usepackage{amsmath, amssymb, nccmath}
\usepackage{accents}

% figure support
\usepackage{import}
\usepackage{xifthen}
\pdfminorversion=7
\usepackage{pdfpages}
\usepackage{transparent}
\newcommand{\incfig}[1]{%
    \def\svgwidth{\columnwidth}
    \import{./figures/}{#1.pdf_tex}
}

\pdfsuppresswarningpagegroup=1

\title{1. Übung\\Maschinen und Antriebe}
\author{DINC Atilla (11917652)}

\begin{document}
\normalsize
\maketitle
%\tableofcontent\newpage

% ~~~~~~~~~~~~~~~~~~~~~~~~~~~~ Start of the document ~~~~~~~~~~~~~~~~~~~~~~~~~~~~
\section*{Angabe}
    \begin{fleqn}[2em]
        \[ \text{Rotordurchmesser:}d_{R}=30cm=0,3m \]
        \[ \text{Eisenlänge:} l_{Fe}=25cm=0,23m\]
        \[ \text{Luftspalt:} \delta=1,2mm=0,0012m\]
        \[ \text{Zahnbreite:}b_{Z}=180mm=0,18m \]
        \[ \text{Jochbreite:}b_{J}=11cm=0,11m \]
        \[ \text{mittlerer Durchmesser des Jochs:}d_{J}=550mm=0,55m \]
        \[ \text{Windungszahl der Erregerwicklung:}N_{E}=600 \]
    \end{fleqn}

    \section*{Frage 1}
    Annahme: Eisen sei ideal permeabel $\mu_{r}\to \infty$ 
    \subsection*{Erregerstrom $I_{E}$}
    Im Luftspalt soll die Flussdichte $B_{\delta}=1,2T$ vorgegeben.
    Aus dem Durchflutungssatz folgt:\newline
    \[ \Theta=I_{E}\cdot N_{E}=H_{L}\cdot 2\delta \]
    \[ \implies I_{E}=\frac{\frac{B_{L}}{\mu_{0}\mu_{r}}2\delta}{N_{E}}\approx3,8197A\]
    \subsection*{Selbstinduktivität}
    Aus dem Durchflutungssatz wird ein Ausdruck hergeleitet, der Erregerstrom und magnetischen Fluss verknüpft.
    \[ I_{E}\cdot N_{E}=H_{L}\cdot 2\delta =\frac{B_{L}}{\mu_{0}\mu_{r}}\cdot 2\delta=\frac{\Phi}{\mu_{0}\mu_{r}A}\cdot 2\delta\]
    \begin{equation} \implies L=\frac{\Phi}{I_{E}}=\frac{N_{E}\mu_{0}\mu_{r}A}{2\delta} \end{equation}
    
    Um die Krümmung des Rotors in der Fläche zu berücksichtigen muss wird die Bogenlänge des Luftspaltes bestimmt.
    \[ \text{Bogenlänge eines Kreissegments:}b=d\cdot \arcsin(\frac{s}{d}) \]
    \[ d\text{\ldots Kreisdurchmesser} \]
    \[ s\text{\ldots Kreissehne} \]
    \[ A_{k}=b\cdot l_{Fe} = (d_{R}+\delta)\arcsin(\frac{b_{Z}}{d_{R}+\delta})\cdot l_{Fe}\approx 0,04823 m^2\]
    \[ L= \frac{N_{E}\mu_{0}\mu_{r}A_{k}}{2\delta}\approx 15,152 mH\]
    
    \subsection*{Erregerwiderstand und Verlustleistung}
    Zur Berechnung des Erregerwiderstandes wird die maximal zulässige Stromdichte $J$, der spezifischer Widerstand von Kupfer $\rho_{Cu}$ sowie dessen Temperaturabhängigkeit $\alpha$ gegeben:
    \begin{fleqn}[2em]
      \[ J=4 \frac{A}{mm^2}\]
      \[ \rho_{Cu}=0,0178 \frac{\Ohm mm^2}{m} \text{\ldots online herausgesucht weil in Letto nur der Leitwert gegeben ist}\]
      \[ \alpha = 0,00393 K^{-1}\]
    \end{fleqn}
    Daraus können direkt folgende Größen bestimmt werden:
    \begin{fleqn}[2em]    
    \[ \overline{l}=1,2\cdot U_{Z}=1,2\cdot 2(b_{Z}+l_{Fe})=1,032m \text{\ldots mittlere Leistungslänge pro Wicklung} \]
    \[ A_{Cu}=\frac{I_{E}}{J} = 0,95493mm^2 \]
    \[ R_{E,20} =\rho_{Cu}*\frac{\overline{l}N_{E}}{A_{Cu}}=11,542\Ohm\]
    \[ R_{E,80}=R_{E,20}\cdot (1+\alpha\cdot (60K))=14,263\Ohm\]
    \[ P_{V,E,80}=I_{E}^2\cdot R_{E,80}=208,11 W\]
    \end{fleqn}

    \section*{Frage 2}
    Annahme: Eisen habe eine Permeabilität $\mu_{r,Fe}=2000$

    \subsection*{Luftspaltflussdichte}
    Der zuvor ermittelte Erregerstrom soll verwendet werden. Es wird wieder der Durchflutungssatz angewandt:
    \[I_{E}\cdot N_{E}=H_{Z}\cdot l_{Z}+H_{R}\cdot d_{R}+H_{J}\cdot \frac{d_{J}\pi}{2}+H_{L}2\delta\]
    Weil Streuflüsse vernachlässigt werden dürfen, muss der Fluss in den unterschiedlichen Segmenten gleich bleiben (Satz des magn. Hüllenflusses). Deshalb kann mit den geometrischen Beziehungen die Flussdichte in allen Segmenten bestimmt werden.
    \begin{fleqn}[2em]
        \[ \Phi_{L}=\Phi_{Z} \implies B_{L}\cdot A_{L}=B_{Z}\cdot A_{Z} \implies B_{Z}=B_{L} \frac{A_{L}}{A_{Z}} \]
        \[ \Phi_{L}=\Phi_{R} \implies B_{L}\cdot A_{L}=B_{R}\cdot A_{R}\implies B_{R}=B_{L} \frac{A_{L}}{A_{R}} \]
        \[ \frac{1}{2}\Phi_{L}=\Phi_{J} \implies \frac{1}{2}B_{L}\cdot A_{L}=B_{J}\cdot A_{J}\implies B_{J}=\frac{1}{2}B_{L}\frac{A_{L}}{A_{J}}\]
    \end{fleqn}
    Die Querschnitte für die unterschiedlichen Segmente berechnen sich zu:
    \begin{fleqn}[2em]
        \[ A_{Z}= b_{z}*l_{Fe}=0,045 m^2\]
        \[ A_{R}=A_{Z} = 0,045 m^2\]
        \[ A_{J}=b_{J}*l_{Fe}=0,0275 m^2 \]
        \[ A_{L}\approx0,04823 m^2 \]
    \end{fleqn}
    In den Durchflutungssatz einsetzen:
    \begin{fleqn}[2em]
        \[ I_{E}\cdot N_{E}&= \frac{1}{\mu_{0}}
            (\frac{1}{\mu_{r,Fe}}B_{Z}l_{Z}
        + \frac{1}{\mu_{r,Fe}}B_{R}d_{R}
        +\frac{1}{\mu_{r,Fe}}B_{J}\frac{d_{J}\pi}{2}
        +\frac{1}{\mu_{r,L}B_{L}2\delta}
    )\]
    \[=\frac{1}{\mu_{0}}(
    \frac{1}{\mu_{r,Fe}}B_{L}\frac{A_{L}}{A_{Z}}l_{Z}
        +\frac{1}{\mu_{r,Fe}}B_{L}\frac{A_{L}}{A_{R}}d_{R}
        +\frac{1}{\mu_{R,Fe}}B_{L}\frac{A_{L}}{2A_{J}}\frac{d_{J}\pi}{2}
        +\frac{1}{\mu_{r,L}}B_{L}2\delta
    )\]
    \[=\frac{B_{L}A_{L}}{\mu_{0}}(\ldots)
    \]
    \[
        \implies B_{L}=I_{E}\cdot N_{E}\cdot \mu_{0}\frac{1}{A_{L}}*\frac{1}{(\ldots)}
    \]
    \end{fleqn}
\end{document}
