\documentclass[a4paper]{article}

\usepackage[a4paper,top=2cm,bottom=2cm,left=3cm,right=3cm,marginparwidth=1.75cm]{geometry}
\usepackage[utf8]{inputenc}
\usepackage[T1]{fontenc}
\usepackage{textcomp}
\usepackage[ngerman]{babel}
\usepackage{amsmath, amssymb, nccmath}
\usepackage{accents}

% figure support
\usepackage{import}
\usepackage{xifthen}
\pdfminorversion=7
\usepackage{pdfpages}
\usepackage{transparent}
\newcommand{\incfig}[1]{%
    \def\svgwidth{\columnwidth}
    \import{./figures/}{#1.pdf_tex}
}

\pdfsuppresswarningpagegroup=1

\title{Mitschrift\\Automatisierungsübung}
\author{DINC Atilla (11917652)}

\begin{document}
\normalsize
\maketitle
%\tableofcontent\newpage

% ~~~~~~~~~~~~~~~~~~~~~~~~~~~~ Start of the document ~~~~~~~~~~~~~~~~~~~~~~~~~~~~
\section*{Wiederholung - Theoretiche Grundlagen}
\begin{itemize}
    \item Koordinatentransformation für einfacheren Rechenweg
    \item Diagonalform zur leichteren Lösung
    \item Jordanblöcke
    \item 3.30 auswendig lernen, erspart Zeit bei der Prüfung
\end{itemize}

\section*{2. Übung: Lineare dynamische Systeme}
\subsection*{2.1. Zwei autonome Systeme}
\[ A_{1}=\begin{pmatrix}
1 && 0 && 0\\
0 && 1 && -1\\
0 && 0 && 2\\
\end{pmatrix}  \]
Gesucht: $x=V_{1}z$, $\tilde{A}$_{1}=V_{1}^{-1}A_{1}V_{1}
\begin{itemize}
    \item Für Eigenwerte gilt: $Ax=\lambda x\implies (A-\lambda E)x=0$ 
        \[\begin{bmatrix}
        1-\lambda && 0 && 0\\
        0 && 1-\lambda && -1\\
        0 && 0 && 2-\lambda\\
        \end{bmatrix} 
        =
    (1-\lambda)^{2}(2-\lambda)=0\]
        \[ \implies \lambda_{1}=1\text{,}n_{1}=2\text{,}g_{1}= \]
        \[ \implies \lambda_{2}=2\text{,}n_{2}=1\text{,}g_{2}=1 \]
        
    \item Für Eigenvektoren gilt: $Kern(A-\lambda_{i}E)$
        \[ g_{1}=dim(Kern(A-\lambda_{1}E))=n-rang(A-\lambda_{1}E) \]
        \lambda_{1}: $(A-\lambda_{1}E)=\begin{bmatrix}
        0 && 0 && 0\\
        0 && 0 && -1\\
        0 && 0 && 1\\
        \end{bmatrix}
        \implies rang(\ldots )=1$
        V_{1}: $(A-\lambda_{1}E)v_{1}=0$
        \implies $v_{1,1}=\begin{pmatrix} 1\\ 0\\ 0\end{pmatrix}$
         \implies $v_{1,2}=\begin{pmatrix} 0\\ 1\\ 0\end{pmatrix}$
        \lambda_{2} \implies $v_{2}=\begin{pmatrix} 0\\ -1\\ 1\end{pmatrix}$

        Transformationsmatrix $V_{1}=\begin{pmatrix} v_{11}&& v_{12}&& v_{2}\end{pmatrix}$
       \[ V_{1}=\begin{pmatrix}
       1 && 0 && 0\\
       0 && 1 && -1\\
       0 && 0 && 1\\
       \end{pmatrix} 
       \cdot 
       \begin{pmatrix}
       1 && 0 && 0\\
       0 && 1 && 0\\
       0 && 0 && 1\\
       \end{pmatrix} 
   \]
   \[ V_{1}^{-1}=\begin{pmatrix}
   1 && 0 && 0\\
   0 && 1 && 1\\
   0 && 0 && 1\\
   \end{pmatrix}  \]

   \[ \tilde{A}_{1}=V_{1}^{-1}A_{1}V_{1}=\begin{pmatrix}
   1 && 0 && 0\\
   0 && 1 && 1\\
   0 && 0 && 1\\
   \end{pmatrix} 
\cdot 
\begin{pmatrix}
1 && 0 && 0\\
0 && 1 && -1\\
0 && 0 && 2\\
\end{pmatrix} 
\cdot 
\begin{pmatrix}
1 && 0 && 0\\
0 && 1 && -1\\
0 && 0 && 1\\
\end{pmatrix} \]
\[ \tilde{A}_{1}=\begin{pmatrix}
1 && 0 && 0\\
0 && 1 && 0\\
0 && 0 && 2\\
\end{pmatrix}  \]
Man hätte das Ergebnis auch einfach ablesen können, aber bei der Prüfung würden nicht so einfache Matrizen auftreten.

\end{itemize}
\[ A_{2}=\begin{pmatrix}
1 && 2 && 2\\
0 && 1 && -1\\
0 && 0 && 2\\
\end{pmatrix}  \]
Gesucht: $x=V_{2}$, $\tilde{A}$_{2}=V_{2}^{-1}A_{2}V_{2}

\begin{itemize}
    \item EW:
        \[ det(A-\lambda E) \]
            \[\implies \lambda_{1}\]     
            \[\implies \lambda_{1}\]     
            Wieder Vielfachheiten bestimmen.
    \item EV:
        \[ v_{1.1}=\begin{pmatrix} 1\\ 0\\ 0\end{pmatrix} \]
        v_{1.2}: $(A-\lambda_{1}E)v_{1.2}=v_{1.1}$
         \[ \implies v_{1.2}=\begin{pmatrix} 0\\ \frac{1}{2}\\ 0\end{pmatrix} \]
        v_{2}: $(A-\lambda_{2}E)v_{2}=0$ 
        \[ \implies v_{2}=\begin{pmatrix} 0\\ -1\\ 1\end{pmatrix} \]
    \item Transformationsmatrix:
        \[ V_{2}=
        \begin{pmatrix}
        1 && 0 && 0\\
        0 && \frac{1}{2} && -1\\
        0 && 0 && 1\\
        \end{pmatrix}^{-1}
    \]
    \[ \tilde{A}_{2}=V_{2}^{-1}A_{2}V
        =\begin{pmatrix}
        1 && 1 && 0\\
        0 && 1 && 0\\
        0 && 0 && 2\\
        \end{pmatrix} 
    \]
    Rest will ich nicht mitschreiben
    \[ \hat{\Phi}_{2}(t)=e^{\lambda Et}\begin{pmatrix}
    1 && t && 0\\
    0 && 1 && 0\\
    0 && 0 && 1\\
    \end{pmatrix}  \]
    
\end{itemize}

\subsection*{2.2 Schwingungsfähiges Feder-Masse-System}
\[ \dot x =\begin{pmatrix}
0 && 1\\
-\frac{k}{m} && -\frac{d}{m}
\end{pmatrix} \vec{x} + \begin{pmatrix} 0\\ \frac{1}{m}\end{pmatrix}u\]
\[ \vec{x}(0)=\vec{x}_{0} \]
\[ y=\begin{pmatrix} 1&& 0\end{pmatrix} \vec{x} \]

\begin{itemize}
    \item EW: $det(A-\lambda E)=0$
        \[ \lambda_{1,2}=-\frac{d}{2m}\pm \sqrt{(\frac{d}{2m})^{2} -\frac{k}{m}}=-1\pm j\sqrt{2}  \] 
    \item EV: $(A-\lambda_{1}E)v_{1}=0$
         \[ \begin{pmatrix}
        1-j\sqrt{2}  && 1\\
         -3 && -1-j\sqrt{2} 
         \end{pmatrix}  v_{1}=0 \]
        \[ \implies \begin{pmatrix}
        1-j\sqrt{2}  && 1\\
        -3(1-j\sqrt{2} ) && -3
        \end{pmatrix}  
        v_{1}
        =\begin{pmatrix}
        1-j\sqrt{2}  && 1\\
        0 && 0
        \end{pmatrix} 
        v_{1}=0
    \]
      \[ v_{1}=\begin{pmatrix} 1+j\sqrt{2} \\ -3\end{pmatrix} \]
      \[ \lambda_{2}=-1-j\sqrt{2} \implies v_{2}=\begin{pmatrix} 1-j\sqrt{2} \\ -3\end{pmatrix}  \]
      
      \[ V=\begin{pmatrix} v_{1}&& v_{2}\end{pmatrix}=\begin{pmatrix} v_{1}&& v_{1}\conjcompl\end{pmatrix} \]

      laut irgendeiner Formel im Skriptum:
      \[ V=\begin{pmatrix} Re{v_{1}}&& Img{v_{2}}\end{pmatrix}
  = \begin{pmatrix}
  1 && \sqrt{2} \\
  -3 && 0
  \end{pmatrix} \]

  \[ V^{-1}=\begin{pmatrix}
  0 && -\frac{1}{3}\\
  \frac{1}{\sqrt{2} } && \frac{1}{\sqrt{2}3 }
  \end{pmatrix}  \]
  \[ V^{-1}=\frac{adj(V_{i,j})}{det(V)}=\frac{(-1)^{i+j} M_{j,i}}{det(V)}=\frac{1}{3\sqrt{2} }\begin{pmatrix}
  0 && -\sqrt{2} \\
  3 && 1
  \end{pmatrix}  \]
  
  \item
      \[ \tilde{A}=V^{-1}AV=\frac{1}{3\sqrt{2} } \begin{pmatrix}
      0 && -\sqrt{2} \\
      3 && 1
      \end{pmatrix} \cdot 
  \begin{pmatrix}
  0 && 1\\
  -3 && -2
  \end{pmatrix} \begin{pmatrix}
  1 && \sqrt{2} \\
  -3 && 0
  \end{pmatrix} =\begin{pmatrix}
  -1 && \sqrt{2} \\
  -\sqrt{2}  && 1
  \end{pmatrix} \]
\ldots  wie die Form aus dem Skriptum (könnte man wieder direkt anschreiben wenn man es auswendig kann)
  \[ B=\begin{pmatrix} 0\\ 1/2\end{pmatrix} \]
  
\[ \tilde{B}=V^{-1}B=\begin{pmatrix} -\frac{1}{6}\\ \frac{1}{\sqrt{2}6 }\end{pmatrix} \]

      \[ C=\begin{pmatrix} 1&& 0\end{pmatrix} \]
      \[ \tilde{C}=C^{T}V= \begin{pmatrix} 1&& \sqrt{2} \end{pmatrix}\]
     
      \[ \tilde{\Phi}(t)=e^{-\t}\begin{pmatrix}
      \cos(\sqrt{2}t ) && \sin(\sqrt{2}t )\\
      -\sin(\sqrt{2}t ) && \cos(\sqrt{2}t )
      \end{pmatrix}  \]

\end{itemize}

\subsection*{Aufgabe 2.4 Laplacetransformation}
\[ \dot x =Ax+Bu \]
\[ y=Cx+Du \]
Laplacetransformiert
\[ sXs)-x_{0}= AX(s)+Bu(s) \]
\[ Y(s)=CX(s)+Du(s) \]
\[ \implies (sE-A)X(s)-x_{0=Bu(s)} \]
\[ X(s)=(sE-A)^{-1}(Bu(s)+x_{0}) \]
\[ y(s)=C(sE-A)^{-1}x_{0}+C(sE-A)^{-1}(Bu(s))+du(s)\]
\[ \implies x_{0}=0, y(s)={C(sE-A)^{-1}B+D}u(s) \]
\[ G(s)=\frac{y(s)}{u(s)}=C(sE-A)^{-1}+D \]

Alternativ:
\[ x(t)=\Phi(t)x_{0}+\int_{0}^{t} \Phi(t-\tau)Bu(\tau)d\tau  \]
\[ X(s)=\Phi(s)x_{0}+\Phi(s)Bu(s) \]
\[\implies \Phi(s)=(sE-A)^{-1} \]


\[ (sE-A)=\begin{pmatrix}
s+3 && 4\\
-2 && s-1
\end{pmatrix}  \]
\[ \Phi(s)=(sE-A)^{-1}=\frac{1}{(s+3)(s-1)8}\begin{pmatrix}
    (s-1) && -4\\
2 && (s+3)
\end{pmatrix}  \]

Rücktransformation:
\[ \frac{1}{s^{2}+2s+5}=\frac{1}{(s+1)^{2}+4}\implies \frac{1}{2}e^{-t}\sin(2t)  \]
\[ \frac{s-1}{(s+1)^{2}+4}\implies \frac{1}{2}e^{-t}\sin(2t)+e^{-t}2\cos(2t))-\frac{1}{2}e^{-t}\sin(2t) \]
\[ \frac{s-3}{(s+1)^{2}+4}\implies \frac{1}{2}e^{-t}\sin(2t)+e^{-t}2\cos(2t))-3\frac{1}{2}e^{-t}\sin(2t)  \]

\[ x(s)=\Phi(s)x_{0}+\Phi(s)Bu(s)=0+Bu(s) \]
\[ y(s)=Cx(s)+Du(s)=Cx(s)+0 \]
\[ \implies y(s)=C\Phi(s)Bu(s)=\frac{s-1}{(s+1)^{2}+4}\frac{1}{s-1}=\frac{1}{N} \]
\[ \text{Rücktransformation:} y(t)=\frac{1}{2}e^{-t}\sin(2t) \]

\subsection*{Bonusbeispiel, weil wir erst seit 2h ohne Pause dran sind}
\[ A=\begin{pmatrix}
1 && 0 && -s^{2}\\
0 && -2 && -2s\\
1 && s && -1\\
\end{pmatrix}  \]
Wir haben 5min Zeit bis er uns das Ergebnis zeigt. Wir brauchen ja keine Pause, aber er kann gern im Gang chillen.

\[ A^{-1}= \frac{1}{2} \begin{pmatrix}
2+2s && -2s && 2\\
-s^{3} && s^{2}-1 && -s\\
-2s^{2} && 2s && -2\\
\end{pmatrix}\]
\ldots Alles falsch\ldot ich hab i und j in der Formel vertauscht

Eine 5min Pause, nach 2h und 10min oida. Der Rechenteil ist abgeschlossen. Jetzt macht er noch eine Fragestunde und zeigt den Umgang mit Maple und Matlab.

\section*{MatLab}
Alle Programme die er hier zeigen wird, sind auf der Homepage verfügbar.
Hilfreiche Befehle:
\begin{itemize}
    \item "doc ss"
    \item "doc c2d"
\end{itemize}
\section*{Maple}
Alle Programme die er hier zeigen wird, sind auf der Homepage verfügbar.
Es sind nicht Programme zu den exakt gleichen Beispielen aber trotzdem das gleiche.
\subsection*{Beispiel 2.1 in Maple}
\subsection*{Beispiel 2.2 in Maple}
\subsection*{Beispiel 2.3 in Maple}
\subsection*{Beispiel 2.4 in Maple}
\subsection*{Beispiel 2.5 in Maple}

Oida, er hat sich jetzt genau 10min für die Programme genommen. Hat sich voll gelohnt so lang zu warten\ldots 
\section*{Fragestunde}
Hab ich mir nicht gegeben.

\end{document}
