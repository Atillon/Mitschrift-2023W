\documentclass[a4paper]{article}

\usepackage[a4paper,top=2cm,bottom=2cm,left=3cm,right=3cm,marginparwidth=1.75cm]{geometry}
\usepackage[utf8]{inputenc}
\usepackage[T1]{fontenc}
\usepackage{textcomp}
\usepackage[ngerman]{babel}
\usepackage{amsmath, amssymb}
\usepackage{accents}

% figure support
\usepackage{import}
\usepackage{xifthen}
\pdfminorversion=7
\usepackage{pdfpages}
\usepackage{transparent}
\newcommand{\incfig}[1]{%
    \def\svgwidth{\columnwidth}
    \import{./figures/}{#1.pdf_tex}
}

\pdfsuppresswarningpagegroup=1

\begin{document}

\subsection*{17.10.2023}
Notiz: erste stunde fehlt

\section*{2.5 Linearisierung nicht-linearer Systeme}
Beispiel: Mathematischs Pendel
\[\frac{d\phi}{dt}= \omega\]
\[\frac{d\omega}{dt} = -\frac{g}{l.sin(\phi)}\]

\[\frac{dx}{dt} = f(x)\text{,} x_{0}  x=\begin{pmatrix} \phi\\ \omega\end{pmatrix}\]
 \[ w_{R}=0\]

 $\sin(\phi_{R})=0$ folglich $\phi_{R1}=0$, $\phi_{R2}=\pi$ (hier gibt es eigendlich unendlich viele lösungen für \phi, weil der \R^2 verwendet wurde
der \R^2 ist der falsche Zustandsraum, der richtige Raum wäre eigendlich der SO3, der Zylinder?)

\[\frac{d}{dt}\begin{pmatrix} \delta\phi\\ \delta\omega\end{pmatrix}=\begin{pmatrix}
0 && 1\\
-\frac{g}{l\cdot \cos(\phi_{R})} && 0
\end{pmatrix}
\cdot \begin{pmatrix} \delta\phi\\ \delta\omega\end{pmatrix}\]

Ausgangslage (Linearisierung um die untere Ruhelage):
\[ w_{R}=0\text{,} \phi_{R1}=0 \]
\[ A=\begin{pmatrix}
0 && 1\\
-\frac{g}{l} && 0
\end{pmatrix}\]
\[det(A-\lambda\cdot I)=det\begin{pmatrix}
-\lambda && 1\\
-\frac{g}{l} && -\lambda
\end{pmatrix} =0\]
$\lambda_{1,2}=\pm I\sqrt(\frac{g}{l})$  - Klassische Eigenwerte für einen ewig schwingendes, ungedämpftes System
$I=\sqrt(-1)$  um Verwechslungen mit Laufindices zu vermeiden

Ausgangslage (linearisierung um die obere Ruhelage)
\[w_{R}=0, \phi_{R2}=\pi\]
\[A=\begin{pmatrix}
0 && 1\\
\frac{g}{l} && 0
\end{pmatrix} \]
\[det(A-lambda.I)=det\begin{pmatrix}
-\lambda && 1\\
\frac{g}{l} && -\lambda
\end{pmatrix}
=0\]
$\lambda_{1,2}=\pm\sqrt(\frac{g}{l})$  - Klassische Eigenwerte für einen ewig schwingendes, ungedämpftes System
Linearisierung ist ein Werkzeug, das sehr systematisch und sehr einfach angewandt werden kann.
In so einem einfachen Fall, kann natürlich auch mit der "Kleine-Winkel"-Näherung linearisiert werden.
Doch dann muss auf den Winkel geachtet werden, um den linearisiert wird (Näherung nur bei phi=0 gültig).
Mit dieser Methode mit der Jakobimatrix liegt man jedoch immer am richtigen Weg.
    -Fazit: diese Methode (Linearisierung mittels Jakobimatrix) ist bombensicher

\section*{inearisierung um die Trajektorie}
Annahme: Eine Trajektorie eines Endeffektors wird idealisiert bestimmt. Was passiert, wenn Störungen
auf dem Pfad auftreten?

\[\frac{dx}{dt}=f(x,u), x0\]
\[y=h(x,u)\]

\[\tilde{u}(t)\implies  \tilde{x}(t) \implies \tilde{y}(t)\]
\[x(t) = x~+deltax\]
u(t) = u~#deltau
y(t) = y~+deltay

Taylorentwicklung
dx~/dt + ddeltax/dt = f(x~+dektax, u~+deltau)
                    ~= f(x~, u~) + pard/
analoges für die Ausgangsgröße
siehe Skriptum (Folie 95)
zwischennotiz: Im Anhang gibt es auch detailreiche Beweise (nicht Prüfungsstoff)

\subsubsection*{Beispiel} Rakete (zeitlich veränderliche Masse)
siehe Skriptum

Zusammenfassung:
\begin{itemize}
    \item was ist eine Ruhelage?
    \item bei linearen Systemen gibt es entweder 0, 1, oder inf Ruhelagen
\end{itemize}
Nächste Stunde:
\begin{itemize}
    \item was für Systemverhalten können aus den Eigenwerten gelesen werden?
    \item Tools: Jordan-Normalformen, Linearkombinationen, ...
\end{itemize}
Wir befinden uns nochimmer am besten Weg zum Regelkreis!


\section*{Wiederholung}
    \begin{itemize}
        \item x\ldots Eingang
        \item u \ldots Zusatnd
        \item y \ldots Ausgang
    \end{itemize}
Ein LTI-SISO-System ist definiert als
\[\dot x=A\cdot x + b\cdot u, x_0\]
\[y=c^{T} + d\xdot u\]
\[\implies x=V\dot z \ldots V \text{ist regulär}\]
\[\dot x = V \cdot \dot z = A\cdot V\cdot z + b\cdot u\]
\[\dot z = V^{-1}\cdot A\cdot V\cdot z + V^{-1}\cdot b\cdot u\]
\[y= c^{T}\cdot V\cdot z + d\cdot u\]
Bestimmung der Eigenwerte:
\[det(\lambda \cdot E - A)=0\]
\[det(V^{-1})\cdot det(\lambda \cdot E - A)\cdot det(V)=0\]
\[det(\lambda \cdot V^{-1}\cdot V - V^{-1}\cdot A\cdot V)=0\]
\[det(\lambda \cdot E - \tilde A)=0\]
\[\bold{det(V\cdot A)=det(V)\cdot det(A)}\]

Die Gleichungen sind äquivalent, die Dynamik des Systems bleibt unverändert.\newline
Die Nullstellen des charakterisitischen Polynoms werden als die "Wurzel des Polynoms" bezeichnet. Das liegt daran, dass die Nullstellen des Nennerpolynoms Polstellen und die des Zählerpolynoms Nullstellen genannt werden.
Die Eigenwerte charakterisieren die Dynamik, und diese wird sich nicht ändern, man wird nur mal anders draufschauen.\newline
Fallunterscheidungen der Eigenwerte im Rahmen dieser VO:
\begin{itemize}
    \item relle Eigenwerte
    \item konjugiert komplexe Eigenwerte
    \item Vielfachheiten (geom. und algebraisch)
        Die Geometrische Vielfachheit ist beschrieben, durch den Rangeinbruch der Matrix A. (Man setzt also einen Eigenwert in $det(\lambda E-A)$ ein, und schaut, wie weit der Rang zurückgeht; siehe Beispiel)
\end{itemize}

Warum brauchen wir Diagonalmatrizen? Es ist viel einfacher beim Lösen, die Eigenwerte spiegeln sich so in Exponentialfunktionen wieder. Die Lösungen von z und x lassen sich ineinander umrechnen.\newline
\section*{Wichtiges beispiel 3.1:}
\[\dot x = A \cdot x = 
\begin{pmatrix}
3 && 2 && -2\\
0 && 1 && 0\\
0 && 0 && 1\\
\end{pmatrix}\cdot x\]
Die Eigenwerte einer Dreiecksmatrix liegen IMMER auf der Diagonale. (wichtig für Prüfungen)
\[(A-\lambda_{i}\cdot E)\cdot v_i=0$, $i \in {1,2,3}\]
Für $\lambda_1=3$:
\[\implies v_{13}=0, v_{1,2}=0, v_{11}=beliebig\]
\[\begin{pmatrix} v_{11}\\ 0\\ 0\end{pmatrix}, v_{11}\neq 0\]
Für $\lambda_2=1$:
Durch einsetzen sieht man:
\[rang(A-1\cdot E)=1\]
$dim(Kern(A-1\cdot E))=3-rang(A-1\cdot E)=2$  \ldots der "Verlust" den man hat, ist immer die Dim des Kerns
analog wie zuvor ergibt sich:
\[2\cdot v_{21}+2\cdot v_{22}-2\cdot v_{23}=0\]
\[[-v_{22}+v_{23}, v_{22}, v_{23}]\]
Es können einige Variablen gewählt werden. Gewählt wurden
\[V=\begin{pmatrix} v_{1}&& v_{2}&& v_{3}\end{pmatrix}=\begin{pmatrix}
1 && 0 && -1\\
0 && 1 && 1\\
0 && 1 && 0\\
\end{pmatrix}\]
\[\dot z =V^{-1}\cdot A\cdot V\cdot z=\tilde A=\begin{pmatrix}
3 && 0 && 0\\
0 && 1 && 0\\
0 && 0 && 1\\
\end{pmatrix}\cdot z, z(9)=z_0=V^{-1}\cdot x_0\]
\[\tilde \Phi=\begin{pmatrix}
\exp(3t) && 0 && 0\\
0 && \exp(t) && 0\\
0 && 0 && \exp(t)\\
\end{pmatrix}\]

\section*{3.2.2 Notwendigkeit von Hauptvektoren}
$(A-\lambda \cdot E)v_1=$ wird gelöst, woraus dann die restlichen Hauptvektoren bestimmt werden können:
\[(A-\lambda E)v_{j+1}=v\]

Nach Herleitung wie im Skriptum folgt:
\[A\cdot V=v\cdot (\lambda\cdot E+N)\]
 Beispiel:  Berechne $V\cdot N$, wobei $V \in R^{n x n}$
N ist eine nilpotente Matrix, die wie ein Schieberegister funktioniert, sie verschiebt alle Spaltenvektoren um eine Stelle nach rechts. Nach genügend vielen Schritten kommt immer eine Nullmatrix raus!
\newline
Wenn gilt $A\cdot B=B\cdot A$, dann gilt $\exp(A+B)=\exp(A)\exp(B)$. Damit folgt aus 3.24 die Lösung 2.25.
Was sehen wir: Das Lösungsverhalten wird durch den Eigenwert dominiert. Hiermit wurde das Lösungsverhalten gezeigt, wenn die algebraische Vielfachheit größer als die geometrische ist.

Keine Ahnung was das hier werden soll:
 \[
\begin{pmatrix}
    v_{11,R}+I\cdot v_{11,R} && v_{12,R}+I\cdot v_{12,R}\\
v_{11,R}-I\cdot v_{11,R} && v_{12,R}-I\cdot v_{12,R}\\
\end{pmatrix}  
\cdot \frac{1}{2}\cdot
\begin{pmatrix}
1 && 1\\
-I && I
\end{pmatrix} =
\begin{pmatrix}
v_{11,R} && v_{12,R}\\
v_{11,I} && v_{12,I}
\end{pmatrix} 
\]

 \subsection*{3.irgendwas Zusammenfassung}
 Man betrachte wieder die Fallunterscheidung der Eigenwerte. Die Eigenwerte bestimmen nämlich das Lösungsverhalten.\newline

Beispiel: Unbedingt selbstständig durchrechnen (er hat das Beispiel nur durchbesprochen)

Mit der Dynamikmatrix, im speziellen mit ihrer Eigenwerte, identifizieren wir das Verhalten. Im Weiteren wird die Stabilität beurteilt. Wir haben heute die Fallunterscheidungen der Eigenwerte untersucht, weil uns die Lage der Eigenwerte extrem viel sagen.
\section*{3.3 Allgemeines Lösungsverhalten}
Ein lineares System kann nicht, in endlicher Zeit nach unendlich oder gegen 0 gehen. Ein nichtlineares System kann das. Siehe Satz 3.3. Mit Satz 3.4 haben wir den ersten Stabilitätsbegriff kennengelernt.

\section*{Nächstes Mal:}
Was ist die Bedeutung des Eigenvektors. Im Beispiel der Flugtechnik ist weiters die Richtung notwendig, da wird der EigenVEKTOR notwendig.

\section*{31.10.2023}
\section*{Wiederholung}
\section*{3.5 Realisierungsproblem}
Gegeben sei ein LTI-System:
\[ \dot x = Ax+bu\text{,}x_{0} \]
\[ dot y = c^{T}x+du \]
Daraus kann mit einem Ausgangszustand $x_{0}$ eine Übertragungsfunktion im Laplacebereich hergeleitet werden:
\[ G(s)=\frac{\hat{Y(s)}}{\hat{U}(s)}=c^{T}(SE-A)^{-1b+d}\]
Der reverse Weg beschreibt ein Realisierungsproblem.

Wenn der Zählergrad grösser als der Nennergrad ist, hat man einen DIfferenzierer, daher macht es Sinn, dass man da keinen Zustand haben kann.
Es werden Eigenschaften mit den englischen Begriffen
\begin{itemize}
    \item proper: Zählergrad \ge  Nenner
    \item strictly proper: Zählergrad > Nennergrad
\end{itemize}

\subsubsection*{Beispiel:}
\[ \tilde{b_{l}}=b_{l}-a_{l}b_{n} \]
\ldots bitte nachlesen

\subsection*{Stabilität}

\subsubsection*{BIBO-Stabilität}
\[ \dot x=Ax\text{,} x_{0} \]
\[ G_{s}=\frac{\hat{Y}}{\hat{U}}\]
\[ \lim_{t \to \infty} x(t)=0 \]
\subsubsection*{Satz 3.7: BIBO-Stabilität anhand der Impulsantwort}
Impuls-Eingang Laplace-transformiert ergibt
\[ G(s)=1\]
daraus folgt
\[ \hat{Y}(s)=G(s)*1\text{,}y(t)=\mathbb{L}^{-1}\{G(s)\}=g(t) \]

\subsubsection*{Satz 3.8: BIBO-Stabilität anhand der Impulsantwort}
\[ G(s)=\frac{\hat{Y}}{\hat{U}}=c^{T}(SE-A)^{-1}b+d=\frac{Z(s)}{N(s)} \]
Hieraus lässt sich erkennen, dass sich die Eigenwerte im Nenner befinden müssen.

\section*{3.7 Kontinuierlicher Frequenzgang}
Wir beschränken uns auf Harmonische Eingangs- und Ausgangsgrössen. Wenn eine 
harmonische Grösse aufgeschaltet wird, schwingen nach ausreichend langer Zeit
alle transienten Funktionen im Ausgang ab und es bleibt eine harmonische
Schwingung mit der gleichen Frequenz.

\subsubsection*{Beispiel 3.4} Im Skriptum durchbesprochen
Man sieht, wir erhalten wieder eine harmonische Schwingung, es ändern sich nur 
Amplitude und Phase. Am Beispiel $Y=U^{2}$ sieht man auch direkt, dass das nur
für lineare Systeme gilt.

Rückblick auf die komplexen Zahlen: $z_{1}=a_{1}+Ib_{1}=Betrag(z_{1})\cdot e^{Iarg(z_{1})}$ 
\[ Betrag(z_{1})=\sqrt{a_{1}^{2}+b_{1}^{2}}  \]
\[ arg(z_{1}) =\arctan(\frac{b_{1}}{a_{1}})\]
Für die Prüfung: $\arctan(\frac{1}{1})\neq \arctan(-\frac{1}{-1})$ das ist KEIN
Rechenfehler, das wird bei der Prüfung als ein normaler Fehler angerechnet.\newline
HÜ: Rechne auch $z_{1}\cdot z_{2}$ und $\frac{z_{1}}{z_{2}}$ aus.
Der Prof hat lang und breit erklärt, dass wir bei der Prüfung nachdenken müssen,
es werden keine 40 Zeilen zum umformen erwartet. Das dividieren zweier komplexer
Zahlen ist kein Trick sondern eine Grundlage.

Beim Aufbau eines Frequenzgangs könnte man eine Frequenz einstellen, und aufs
Einschwingen warten, und das für ganz viele Frequenzen wiederholen. Wird aber
in der Praxis nicht so gemacht. Professionell wird Eingang und Ausgang
irgendwie bestimmt angeregt, beides FF-transformiert und die beiden FFTs mit-
einander dividiert.

Zur Darstellung werden Bodediagram und Ortskurve verwendet.

Eine Nyquist-Ortskurve wird mit Betrag und Winkel geplottet (siehe Skriptum)

Das Bodediagram:
\begin{itemize}
    \item Amplitudengang: doppellogarithmisch aufgetragener Betrag
    \item Phasengang
\end{itemize}

Wieso wird das so gemacht? Siehe Gleichungen 3.117 und 3.118
\[ \log(\frac{a}{b})=\log(a)-\log(b) \]
\[ \log(a\cdot b)=\log(a)+\log(b) \]

Zu Gleichung 3.119:
Es sind einige Erkenntnisse zu den reellen und konj. kompl. Nullstellen zu sehen.
Weiters:
\begin{itemize}
    \item $\xi=0$ \implies  $1+(\frac{s}{\omega_{z}^{2}}=0$ 
    \item $\xi=1$ \implies  $ 1+ 2(\frac{s}{\omega_{z}+(\frac{s}{\omega_{z}})^2=(1+\frac{s}{\omega_{z}})^{2}=0}$
\end{itemize}
Die Nullstellen lassen sich sehr gut in der komplexen Ebene abbilden.
Diese Erkenntnis ist sehr fundamental für die gesamte Vorlesung.

$V$ wird als Verstärkungsfaktor bezeichnet.

\[ G(s)=\frac{\hat{Y}}{\hat{U}} \]
\[ u(t)=\sigma(t) \text{laplacetransformiert} \hat{U}=\frac{1}{s}\]
\[ \sigma(t) \text{\ldots Heavyside-Funktion} \]
\[ \hat{Y}=G(s)\hat{U}=G(s) \frac{1}{s} \]
Endwertsatz: $\lim_{t \to \infty} y(t)=\lim_{s \to 0} s \hat{y}=\lim_{s \to 0} G(s)=G(0)=V \frac{Z(0)}{N(0)}=V$ 

$(I\omega)'$ wird als Integrator/Differenzierer bezeichnet.
Integrator
\[ \dot x=u \text{,} x_{0}=0\]
\[ y=x \]
\[ s \hat{x}=\hat{u} \]
\[ \hat{y}=\hat{x} = \frac{\hat{u}}{s} \implies \frac{\hat{y}}{\hat{u}}=\frac{1}{s}\]

Differenzierer
\[ y=\dot u \]
\[ \hat{y}=s \hat{u} \]
\[ \frac{\hat{y}}{u}=s=G(s) \]

\[ G(s)=\frac{1}{s}=\frac{1}{I\omega}\]
\[ Betrag(G(s))_{dB}= \text{Im Bodediagram plotten (Gerade mit k=-45°)}\]
\[ arg(\frac{1}{I\omega})=0-arg(I\omega)=-90° \text{\ldots const}\]


\[ G(s)=\frac{1}{s^{2}}=-\frac{1}{\omega^{2}}\]
\[ Betrag(G(s))_{dB}= \text{Im Bodediagram plotten}\]
\[ arg(\frac{1}{s^{2}})=\ldots \]

Der Differenzierer steigt stattdessen. Für sehr hohe Frequenzen geht der Differenzierer gegen unendlich.
Da jedes reale System Rauschen hat, macht es den Differenzierer sehr unangenehm.
Der Differenzier macht in echt einfach keinen Sinn, und MatLab weigert sich, ihn aus der Toolbox zu entfernen.

\section*{Wiederholung}
\begin{itemize}
    \item LTI
    \item SISO
    \item jedes Laplace-transformierbare Signal kann auf ein System aufgeschaltet werden, um die Systemantwort zu bestimmen
        Beispiele: Motor: Strom wird aufgeschaltet und Drehzahl bestimmt
    \item Letzte Stunde: Wenn s auf die Imaginäre Achse eingeschränkt wird, hat man einen kontinuierlichen Frequenzgang
        $G(s)\vert_{s=I\omega}=G(I\omega)=Re(G(I\omega)+I\cdot Img(G(I\omega))=\|G(I\omega)e^{I\cdot arg(G(I\omega))}\|$ 
        Mit logarithmischen Frequenzgängen zeigen wir harmonische Eingangs und Ausgangsgrössen.\newline
        Siehe weiters Integrierer, Differenzierer, etc. und siehe weiters die logarithmischen Rechenregeln.
\end{itemize}

\subsubsection*{3) Linearer Term $G_{3}(I\omega)=1+I \frac{\omega}{\omega_{K}}$}
\subsubsection*{Betrag}
Fallunterscheidung
\begin{itemize}                    
    \item für $\frac{\omega}{\omega_{K} \le 1}$ : nähert sich $0$ an
    \item für $\frac{\omega}{\omega_{K} = 1}$: ist etwa 3,0103
    \r $\frac{\omega}{\omega_{K} \bigger 1}$
\end{itemize}
Hierzu bitte den Frequenzgang für $G(s)=1+\frac{s}{10}$ plotten, dann hat man den Betrag von Lineartermen verstanden.

Anmerkung für Prüfung: Transienten müssen im Bodediagram nicht angenähert werden, Knickzüge sind komplett ausreichend.

\subsubsection*{Phase}
Fallunterscheidung
\begin{itemize}
    \item alle 3 fälle wiederholen und für $G(s)=1+\frac{s}{10}$ und $G(s)=\frac{1}{1+\frac{s}{10}}$ plotten
\end{itemize}
Wichtig - Interpretation des Phasengangs: Wenn das Vorzeichen negativ ist, kommen negative Winkel raus. Das trifft für Nullstellen zu, die auf der instabilen (positiven) hälfte der Komplexen Ebene zu. Daher werden negative Phasengänge als instabil interpretiert (definiert).
Wenn jedoch der Term im Nenner stehen würde, z.B. $\frac{1}{1+\frac{s}{10}}$, dann geht die Phase für stabile Systeme nach unten.\newline
Hierzu irgendwie eine Eselsbrücke aneignen und nochmal im Skriptum nachlesen.

\subsubsection*{4) Quadratischer Term $G_{4}(I\omega)=1-(\frac{\omega}{\omega_{K}})^{2}+I2\Xi\frac{\omega}{ \omega_{K}}$, $\omega_{K}\bigger0$ :}
\subsubsection*{Betrag}
\begin{itemize}
    \item wieder 3 Fälle durchführen und für $G(s)=1+2\Xi\frac{s}{10} +(\frac{s}{10}^{2}$ und $G(s)=\frac{1}{1+2\Xi\frac{s}{10} +(\frac{s}{10}^{2}}$ plotten
    \item 4. Fall: Variation vom $\Xi=(0, 1]$. Man erkennt Resonanz- und Antiresonanz (und alles dazwischen)
\end{itemize}
Beispiel: Feder-Masse-Schwingkreis
Aufbau: Wand-Feder-Masse-Kraft
Mit Position x zwischen Wand und Massenzentrum
\[ m \dot\dot x = F-cx \]
\[ y=x \]
\[ \implies s^{2}m \hat{x}=\hat{F}-c \hat{x} \]
\[ \hat{y}=\hat{x} \]
\[ \implies G(S)=\frac{\hat{y}}{\hat{x}}=\frac{1}{s^{2}m+c}=\frac{1}{c}\frac{1}{1+s^{2}\frac{m}{c}}=\frac{1}{c}\frac{1}{1+(\frac{s}{\sqrt{\frac{c}{m}} })^{2}}=\frac{1}{c}\frac{1}{1+(\frac{s}{\omega_{K}})^{2}}\text{,} \Xi_{K}=0\]

Funfakt: Jede Struktur hat unendlich viele Resonanz- und Antiresonanzfrequenzen. Diese befinden sich je nach Struktur extrem weit oben (rechts). Da aber jede Struktur ein Tiefpassverhalten aufweist, sind die hohen Resonanzen rechts stark geschwächt. 

Beispiel: $G(s)=\frac{1}{1+s^{2}}$ 
Betrag sollte einfach sein und wird übersprungen.
Phase:
\[ arg(G(I\omega))=arg(1)-arg(1-\omega^{2})=\arctan(\frac{0}{1-\omega^{2}}) \]
Wieder für $\Xi=(0, 1]$ plotten. Siehe Abbildung 3.17 im Skriptum. Ist ein wenig komplizierter.

\subsection*{Darstellung von Bodediagrammen - Durchführung}
\subsection*{Amplitudengang}
\[ G(s)=\frac{10^{-2} 10}{0.01} \frac{1-\frac{s}{10}}{s(1+2 \frac{1}{2} \frac{s}{0.1}+(\frac{s}{0.1})^{2})}\]

\subsubsection*{normierte Form anschreiben}
\[ G(s)=\frac{10^{-2} 10}{0.01} \frac{1-\frac{s}{10}}{s(1+2 \frac{1}{2} \frac{s}{0.1}+(\frac{s}{0.1})^{2})}\]

\subsubsection*{Algebraisch Teilfunktionen}
\[ G_{1}=10, G_{2}=\frac{1}{s}, G_{3}=1-\frac{s}{10}, G_{4}=\frac{1}{1+2 \frac{1}{2} \frac{s}{0.1} }\]
Diese Teilfunktionen können separat aufgetragen und addiert werden.

\subsubsection*{Grafisch Teilfunktionen}
Weil die unterschiedlichen Knickzüge, nach der Knickfrequenz sortiert sind, können sie nacheinander kontinuierlich grafisch addiert werden.
Das geht natürlich schwer, wenn die Knickfrequenzen sehr knapp anneinander liegen. In dem Fall ergeben sich die Transienten durch Überlagerung. In solchen Fällen sollten die algebraischen Teilfunktionen aufgezeichnet werden. Die Addition dieser Teilfunktionen ist dann nicht mehr so wichtig und kann bei Prüfungen übersprungen werden.

\subsection*{Phasengang}
Hier kann sehr ähnlich gearbeitet werden, siehe Skriptum, kein Bock mitzuschreiben.

\subsection*{Phasenminimale Übertragungsfunktionen}
Eine Übertragungsfunktion $G(s)$ ist phasemminimal, wenn sich alle Pole in der linken offenen Halbebene befinden.

Beispiel: Die Sprungantwort eines Systems wird kurzzeitig negative, bevor sie sich im Positiven einstellt (siehe Versatz eines Pendels, das in der oberen Ruhelage liegt). Die kurzzeitige negative Phase zeigt, dass es eine rechte Polstelle gibt \implies das System ist nicht phasenminimal


\end{document}
