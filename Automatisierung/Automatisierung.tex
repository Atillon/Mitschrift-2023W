\documentclass[a4paper]{article}

\usepackage[a4paper,top=2cm,bottom=2cm,left=3cm,right=3cm,marginparwidth=1.75cm]{geometry}
\usepackage[utf8]{inputenc}
\usepackage[T1]{fontenc}
\usepackage{textcomp}
\usepackage[ngerman]{babel}
\usepackage{amsmath, amssymb, nccmath}
\usepackage{accents}

% figure support
\usepackage{import}
\usepackage{xifthen}
\pdfminorversion=7
\usepackage{pdfpages}
\usepackage{transparent}
\newcommand{\incfig}[1]{%
    \def\svgwidth{\columnwidth}
    \import{./figures/}{#1.pdf_tex}
}

\pdfsuppresswarningpagegroup=1

\title{Mitschrift\\Automatisierung - Wintersemester 2023}
\author{DINC Atilla (11917652)}

\begin{document}
\normalsize
\maketitle
%\tableofcontent\newpage

% ~~~~~~~~~~~~~~~~~~~~~~~~~~~~ Start of the document ~~~~~~~~~~~~~~~~~~~~~~~~~~~~

\subsection*{17.10.2023}
Notiz: erste stunde fehlt

\section*{2.5 Linearisierung nicht-linearer Systeme}
Beispiel: Mathematischs Pendel
\[\frac{d\phi}{dt}= \omega\]
\[\frac{d\omega}{dt} = -\frac{g}{l.sin(\phi)}\]

\[\frac{dx}{dt} = f(x)\text{,} x_{0}  x=\begin{pmatrix} \phi\\ \omega\end{pmatrix}\]
 \[ w_{R}=0\]

 $\sin(\phi_{R})=0$ folglich $\phi_{R1}=0$, $\phi_{R2}=\pi$ (hier gibt es eigendlich unendlich viele lösungen für \phi, weil der \R^2 verwendet wurde
der \R^2 ist der falsche Zustandsraum, der richtige Raum wäre eigendlich der SO3, der Zylinder?)

\[\frac{d}{dt}\begin{pmatrix} \delta\phi\\ \delta\omega\end{pmatrix}=\begin{pmatrix}
0 && 1\\
-\frac{g}{l\cdot \cos(\phi_{R})} && 0
\end{pmatrix}
\cdot \begin{pmatrix} \delta\phi\\ \delta\omega\end{pmatrix}\]

Ausgangslage (Linearisierung um die untere Ruhelage):
\[ w_{R}=0\text{,} \phi_{R1}=0 \]
\[ A=\begin{pmatrix}
0 && 1\\
-\frac{g}{l} && 0
\end{pmatrix}\]
\[det(A-\lambda\cdot I)=det\begin{pmatrix}
-\lambda && 1\\
-\frac{g}{l} && -\lambda
\end{pmatrix} =0\]
$\lambda_{1,2}=\pm I\sqrt(\frac{g}{l})$  - Klassische Eigenwerte für einen ewig schwingendes, ungedämpftes System
$I=\sqrt(-1)$  um Verwechslungen mit Laufindices zu vermeiden

Ausgangslage (linearisierung um die obere Ruhelage)
\[w_{R}=0, \phi_{R2}=\pi\]
\[A=\begin{pmatrix}
0 && 1\\
\frac{g}{l} && 0
\end{pmatrix} \]
\[det(A-lambda.I)=det\begin{pmatrix}
-\lambda && 1\\
\frac{g}{l} && -\lambda
\end{pmatrix}
=0\]
$\lambda_{1,2}=\pm\sqrt(\frac{g}{l})$  - Klassische Eigenwerte für einen ewig schwingendes, ungedämpftes System
Linearisierung ist ein Werkzeug, das sehr systematisch und sehr einfach angewandt werden kann.
In so einem einfachen Fall, kann natürlich auch mit der "Kleine-Winkel"-Näherung linearisiert werden.
Doch dann muss auf den Winkel geachtet werden, um den linearisiert wird (Näherung nur bei phi=0 gültig).
Mit dieser Methode mit der Jakobimatrix liegt man jedoch immer am richtigen Weg.
    -Fazit: diese Methode (Linearisierung mittels Jakobimatrix) ist bombensicher

\section*{inearisierung um die Trajektorie}
Annahme: Eine Trajektorie eines Endeffektors wird idealisiert bestimmt. Was passiert, wenn Störungen
auf dem Pfad auftreten?

\[\frac{dx}{dt}=f(x,u), x0\]
\[y=h(x,u)\]

\[\tilde{u}(t)\implies  \tilde{x}(t) \implies \tilde{y}(t)\]
\[x(t) = x~+deltax\]
u(t) = u~#deltau
y(t) = y~+deltay

Taylorentwicklung
dx~/dt + ddeltax/dt = f(x~+dektax, u~+deltau)
                    ~= f(x~, u~) + pard/
analoges für die Ausgangsgröße
siehe Skriptum (Folie 95)
zwischennotiz: Im Anhang gibt es auch detailreiche Beweise (nicht Prüfungsstoff)

\subsubsection*{Beispiel} Rakete (zeitlich veränderliche Masse)
siehe Skriptum

Zusammenfassung:
\begin{itemize}
    \item was ist eine Ruhelage?
    \item bei linearen Systemen gibt es entweder 0, 1, oder inf Ruhelagen
\end{itemize}
Nächste Stunde:
\begin{itemize}
    \item was für Systemverhalten können aus den Eigenwerten gelesen werden?
    \item Tools: Jordan-Normalformen, Linearkombinationen, ...
\end{itemize}
Wir befinden uns nochimmer am besten Weg zum Regelkreis!


\section*{Wiederholung}
    \begin{itemize}
        \item x\ldots Eingang
        \item u \ldots Zusatnd
        \item y \ldots Ausgang
    \end{itemize}
Ein LTI-SISO-System ist definiert als
\[\dot x=A\cdot x + b\cdot u, x_0\]
\[y=c^{T} + d\xdot u\]
\[\implies x=V\dot z \ldots V \text{ist regulär}\]
\[\dot x = V \cdot \dot z = A\cdot V\cdot z + b\cdot u\]
\[\dot z = V^{-1}\cdot A\cdot V\cdot z + V^{-1}\cdot b\cdot u\]
\[y= c^{T}\cdot V\cdot z + d\cdot u\]
Bestimmung der Eigenwerte:
\[det(\lambda \cdot E - A)=0\]
\[det(V^{-1})\cdot det(\lambda \cdot E - A)\cdot det(V)=0\]
\[det(\lambda \cdot V^{-1}\cdot V - V^{-1}\cdot A\cdot V)=0\]
\[det(\lambda \cdot E - \tilde A)=0\]
\[\bold{det(V\cdot A)=det(V)\cdot det(A)}\]

Die Gleichungen sind äquivalent, die Dynamik des Systems bleibt unverändert.\newline
Die Nullstellen des charakterisitischen Polynoms werden als die "Wurzel des Polynoms" bezeichnet. Das liegt daran, dass die Nullstellen des Nennerpolynoms Polstellen und die des Zählerpolynoms Nullstellen genannt werden.
Die Eigenwerte charakterisieren die Dynamik, und diese wird sich nicht ändern, man wird nur mal anders draufschauen.\newline
Fallunterscheidungen der Eigenwerte im Rahmen dieser VO:
\begin{itemize}
    \item relle Eigenwerte
    \item konjugiert komplexe Eigenwerte
    \item Vielfachheiten (geom. und algebraisch)
        Die Geometrische Vielfachheit ist beschrieben, durch den Rangeinbruch der Matrix A. (Man setzt also einen Eigenwert in $det(\lambda E-A)$ ein, und schaut, wie weit der Rang zurückgeht; siehe Beispiel)
\end{itemize}

Warum brauchen wir Diagonalmatrizen? Es ist viel einfacher beim Lösen, die Eigenwerte spiegeln sich so in Exponentialfunktionen wieder. Die Lösungen von z und x lassen sich ineinander umrechnen.\newline
\section*{Wichtiges beispiel 3.1:}
\[\dot x = A \cdot x = 
\begin{pmatrix}
3 && 2 && -2\\
0 && 1 && 0\\
0 && 0 && 1\\
\end{pmatrix}\cdot x\]
Die Eigenwerte einer Dreiecksmatrix liegen IMMER auf der Diagonale. (wichtig für Prüfungen)
\[(A-\lambda_{i}\cdot E)\cdot v_i=0$, $i \in {1,2,3}\]
Für $\lambda_1=3$:
\[\implies v_{13}=0, v_{1,2}=0, v_{11}=beliebig\]
\[\begin{pmatrix} v_{11}\\ 0\\ 0\end{pmatrix}, v_{11}\neq 0\]
Für $\lambda_2=1$:
Durch einsetzen sieht man:
\[rang(A-1\cdot E)=1\]
$dim(Kern(A-1\cdot E))=3-rang(A-1\cdot E)=2$  \ldots der "Verlust" den man hat, ist immer die Dim des Kerns
analog wie zuvor ergibt sich:
\[2\cdot v_{21}+2\cdot v_{22}-2\cdot v_{23}=0\]
\[[-v_{22}+v_{23}, v_{22}, v_{23}]\]
Es können einige Variablen gewählt werden. Gewählt wurden
\[V=\begin{pmatrix} v_{1}&& v_{2}&& v_{3}\end{pmatrix}=\begin{pmatrix}
1 && 0 && -1\\
0 && 1 && 1\\
0 && 1 && 0\\
\end{pmatrix}\]
\[\dot z =V^{-1}\cdot A\cdot V\cdot z=\tilde A=\begin{pmatrix}
3 && 0 && 0\\
0 && 1 && 0\\
0 && 0 && 1\\
\end{pmatrix}\cdot z, z(9)=z_0=V^{-1}\cdot x_0\]
\[\tilde \Phi=\begin{pmatrix}
\exp(3t) && 0 && 0\\
0 && \exp(t) && 0\\
0 && 0 && \exp(t)\\
\end{pmatrix}\]

\section*{3.2.2 Notwendigkeit von Hauptvektoren}
$(A-\lambda \cdot E)v_1=$ wird gelöst, woraus dann die restlichen Hauptvektoren bestimmt werden können:
\[(A-\lambda E)v_{j+1}=v\]

Nach Herleitung wie im Skriptum folgt:
\[A\cdot V=v\cdot (\lambda\cdot E+N)\]
 Beispiel:  Berechne $V\cdot N$, wobei $V \in R^{n x n}$
N ist eine nilpotente Matrix, die wie ein Schieberegister funktioniert, sie verschiebt alle Spaltenvektoren um eine Stelle nach rechts. Nach genügend vielen Schritten kommt immer eine Nullmatrix raus!
\newline
Wenn gilt $A\cdot B=B\cdot A$, dann gilt $\exp(A+B)=\exp(A)\exp(B)$. Damit folgt aus 3.24 die Lösung 2.25.
Was sehen wir: Das Lösungsverhalten wird durch den Eigenwert dominiert. Hiermit wurde das Lösungsverhalten gezeigt, wenn die algebraische Vielfachheit größer als die geometrische ist.

Keine Ahnung was das hier werden soll:
 \[
\begin{pmatrix}
    v_{11,R}+I\cdot v_{11,R} && v_{12,R}+I\cdot v_{12,R}\\
v_{11,R}-I\cdot v_{11,R} && v_{12,R}-I\cdot v_{12,R}\\
\end{pmatrix}  
\cdot \frac{1}{2}\cdot
\begin{pmatrix}
1 && 1\\
-I && I
\end{pmatrix} =
\begin{pmatrix}
v_{11,R} && v_{12,R}\\
v_{11,I} && v_{12,I}
\end{pmatrix} 
\]

 \subsection*{3.irgendwas Zusammenfassung}
 Man betrachte wieder die Fallunterscheidung der Eigenwerte. Die Eigenwerte bestimmen nämlich das Lösungsverhalten.\newline

Beispiel: Unbedingt selbstständig durchrechnen (er hat das Beispiel nur durchbesprochen)

Mit der Dynamikmatrix, im speziellen mit ihrer Eigenwerte, identifizieren wir das Verhalten. Im Weiteren wird die Stabilität beurteilt. Wir haben heute die Fallunterscheidungen der Eigenwerte untersucht, weil uns die Lage der Eigenwerte extrem viel sagen.
\section*{3.3 Allgemeines Lösungsverhalten}
Ein lineares System kann nicht, in endlicher Zeit nach unendlich oder gegen 0 gehen. Ein nichtlineares System kann das. Siehe Satz 3.3. Mit Satz 3.4 haben wir den ersten Stabilitätsbegriff kennengelernt.

\section*{Nächstes Mal:}
Was ist die Bedeutung des Eigenvektors. Im Beispiel der Flugtechnik ist weiters die Richtung notwendig, da wird der EigenVEKTOR notwendig.

\section*{31.10.2023}
\section*{Wiederholung}
\section*{3.5 Realisierungsproblem}
Gegeben sei ein LTI-System:
\[ \dot x = Ax+bu\text{,}x_{0} \]
\[ dot y = c^{T}x+du \]
Daraus kann mit einem Ausgangszustand $x_{0}$ eine Übertragungsfunktion im Laplacebereich hergeleitet werden:
\[ G(s)=\frac{\hat{Y(s)}}{\hat{U}(s)}=c^{T}(SE-A)^{-1b+d}\]
Der reverse Weg beschreibt ein Realisierungsproblem.

Wenn der Zählergrad grösser als der Nennergrad ist, hat man einen DIfferenzierer, daher macht es Sinn, dass man da keinen Zustand haben kann.
Es werden Eigenschaften mit den englischen Begriffen
\begin{itemize}
    \item proper: Zählergrad \ge  Nenner
    \item strictly proper: Zählergrad > Nennergrad
\end{itemize}

\subsubsection*{Beispiel:}
\[ \tilde{b_{l}}=b_{l}-a_{l}b_{n} \]
\ldots bitte nachlesen

\subsection*{Stabilität}

\subsubsection*{BIBO-Stabilität}
\[ \dot x=Ax\text{,} x_{0} \]
\[ G_{s}=\frac{\hat{Y}}{\hat{U}}\]
\[ \lim_{t \to \infty} x(t)=0 \]
\subsubsection*{Satz 3.7: BIBO-Stabilität anhand der Impulsantwort}
Impuls-Eingang Laplace-transformiert ergibt
\[ G(s)=1\]
daraus folgt
\[ \hat{Y}(s)=G(s)*1\text{,}y(t)=\mathbb{L}^{-1}\{G(s)\}=g(t) \]

\subsubsection*{Satz 3.8: BIBO-Stabilität anhand der Impulsantwort}
\[ G(s)=\frac{\hat{Y}}{\hat{U}}=c^{T}(SE-A)^{-1}b+d=\frac{Z(s)}{N(s)} \]
Hieraus lässt sich erkennen, dass sich die Eigenwerte im Nenner befinden müssen.

\section*{3.7 Kontinuierlicher Frequenzgang}
Wir beschränken uns auf Harmonische Eingangs- und Ausgangsgrössen. Wenn eine 
harmonische Grösse aufgeschaltet wird, schwingen nach ausreichend langer Zeit
alle transienten Funktionen im Ausgang ab und es bleibt eine harmonische
Schwingung mit der gleichen Frequenz.

\subsubsection*{Beispiel 3.4} Im Skriptum durchbesprochen
Man sieht, wir erhalten wieder eine harmonische Schwingung, es ändern sich nur 
Amplitude und Phase. Am Beispiel $Y=U^{2}$ sieht man auch direkt, dass das nur
für lineare Systeme gilt.

Rückblick auf die komplexen Zahlen: $z_{1}=a_{1}+Ib_{1}=Betrag(z_{1})\cdot e^{Iarg(z_{1})}$ 
\[ Betrag(z_{1})=\sqrt{a_{1}^{2}+b_{1}^{2}}  \]
\[ arg(z_{1}) =\arctan(\frac{b_{1}}{a_{1}})\]
Für die Prüfung: $\arctan(\frac{1}{1})\neq \arctan(-\frac{1}{-1})$ das ist KEIN
Rechenfehler, das wird bei der Prüfung als ein normaler Fehler angerechnet.\newline
HÜ: Rechne auch $z_{1}\cdot z_{2}$ und $\frac{z_{1}}{z_{2}}$ aus.
Der Prof hat lang und breit erklärt, dass wir bei der Prüfung nachdenken müssen,
es werden keine 40 Zeilen zum umformen erwartet. Das dividieren zweier komplexer
Zahlen ist kein Trick sondern eine Grundlage.

Beim Aufbau eines Frequenzgangs könnte man eine Frequenz einstellen, und aufs
Einschwingen warten, und das für ganz viele Frequenzen wiederholen. Wird aber
in der Praxis nicht so gemacht. Professionell wird Eingang und Ausgang
irgendwie bestimmt angeregt, beides FF-transformiert und die beiden FFTs mit-
einander dividiert.

Zur Darstellung werden Bodediagram und Ortskurve verwendet.

Eine Nyquist-Ortskurve wird mit Betrag und Winkel geplottet (siehe Skriptum)

Das Bodediagram:
\begin{itemize}
    \item Amplitudengang: doppellogarithmisch aufgetragener Betrag
    \item Phasengang
\end{itemize}

Wieso wird das so gemacht? Siehe Gleichungen 3.117 und 3.118
\[ \log(\frac{a}{b})=\log(a)-\log(b) \]
\[ \log(a\cdot b)=\log(a)+\log(b) \]

Zu Gleichung 3.119:
Es sind einige Erkenntnisse zu den reellen und konj. kompl. Nullstellen zu sehen.
Weiters:
\begin{itemize}
    \item $\xi=0$ \implies  $1+(\frac{s}{\omega_{z}^{2}}=0$ 
    \item $\xi=1$ \implies  $ 1+ 2(\frac{s}{\omega_{z}+(\frac{s}{\omega_{z}})^2=(1+\frac{s}{\omega_{z}})^{2}=0}$
\end{itemize}
Die Nullstellen lassen sich sehr gut in der komplexen Ebene abbilden.
Diese Erkenntnis ist sehr fundamental für die gesamte Vorlesung.

$V$ wird als Verstärkungsfaktor bezeichnet.

\[ G(s)=\frac{\hat{Y}}{\hat{U}} \]
\[ u(t)=\sigma(t) \text{laplacetransformiert} \hat{U}=\frac{1}{s}\]
\[ \sigma(t) \text{\ldots Heavyside-Funktion} \]
\[ \hat{Y}=G(s)\hat{U}=G(s) \frac{1}{s} \]
Endwertsatz: $\lim_{t \to \infty} y(t)=\lim_{s \to 0} s \hat{y}=\lim_{s \to 0} G(s)=G(0)=V \frac{Z(0)}{N(0)}=V$ 

$(I\omega)'$ wird als Integrator/Differenzierer bezeichnet.
Integrator
\[ \dot x=u \text{,} x_{0}=0\]
\[ y=x \]
\[ s \hat{x}=\hat{u} \]
\[ \hat{y}=\hat{x} = \frac{\hat{u}}{s} \implies \frac{\hat{y}}{\hat{u}}=\frac{1}{s}\]

Differenzierer
\[ y=\dot u \]
\[ \hat{y}=s \hat{u} \]
\[ \frac{\hat{y}}{u}=s=G(s) \]

\[ G(s)=\frac{1}{s}=\frac{1}{I\omega}\]
\[ Betrag(G(s))_{dB}= \text{Im Bodediagram plotten (Gerade mit k=-45°)}\]
\[ arg(\frac{1}{I\omega})=0-arg(I\omega)=-90° \text{\ldots const}\]


\[ G(s)=\frac{1}{s^{2}}=-\frac{1}{\omega^{2}}\]
\[ Betrag(G(s))_{dB}= \text{Im Bodediagram plotten}\]
\[ arg(\frac{1}{s^{2}})=\ldots \]

Der Differenzierer steigt stattdessen. Für sehr hohe Frequenzen geht der Differenzierer gegen unendlich.
Da jedes reale System Rauschen hat, macht es den Differenzierer sehr unangenehm.
Der Differenzier macht in echt einfach keinen Sinn, und MatLab weigert sich, ihn aus der Toolbox zu entfernen.

\section*{Wiederholung}
\begin{itemize}
    \item LTI
    \item SISO
    \item jedes Laplace-transformierbare Signal kann auf ein System aufgeschaltet werden, um die Systemantwort zu bestimmen
        Beispiele: Motor: Strom wird aufgeschaltet und Drehzahl bestimmt
    \item Letzte Stunde: Wenn s auf die Imaginäre Achse eingeschränkt wird, hat man einen kontinuierlichen Frequenzgang
        $G(s)\vert_{s=I\omega}=G(I\omega)=Re(G(I\omega)+I\cdot Img(G(I\omega))=\|G(I\omega)e^{I\cdot arg(G(I\omega))}\|$ 
        Mit logarithmischen Frequenzgängen zeigen wir harmonische Eingangs und Ausgangsgrössen.\newline
        Siehe weiters Integrierer, Differenzierer, etc. und siehe weiters die logarithmischen Rechenregeln.
\end{itemize}

\subsubsection*{3) Linearer Term $G_{3}(I\omega)=1+I \frac{\omega}{\omega_{K}}$}
\subsubsection*{Betrag}
Fallunterscheidung
\begin{itemize}                    
    \item für $\frac{\omega}{\omega_{K} \le 1}$ : nähert sich $0$ an
    \item für $\frac{\omega}{\omega_{K} = 1}$: ist etwa 3,0103
    \r $\frac{\omega}{\omega_{K} \bigger 1}$
\end{itemize}
Hierzu bitte den Frequenzgang für $G(s)=1+\frac{s}{10}$ plotten, dann hat man den Betrag von Lineartermen verstanden.

Anmerkung für Prüfung: Transienten müssen im Bodediagram nicht angenähert werden, Knickzüge sind komplett ausreichend.

\subsubsection*{Phase}
Fallunterscheidung
\begin{itemize}
    \item alle 3 fälle wiederholen und für $G(s)=1+\frac{s}{10}$ und $G(s)=\frac{1}{1+\frac{s}{10}}$ plotten
\end{itemize}
Wichtig - Interpretation des Phasengangs: Wenn das Vorzeichen negativ ist, kommen negative Winkel raus. Das trifft für Nullstellen zu, die auf der instabilen (positiven) hälfte der Komplexen Ebene zu. Daher werden negative Phasengänge als instabil interpretiert (definiert).
Wenn jedoch der Term im Nenner stehen würde, z.B. $\frac{1}{1+\frac{s}{10}}$, dann geht die Phase für stabile Systeme nach unten.\newline
Hierzu irgendwie eine Eselsbrücke aneignen und nochmal im Skriptum nachlesen.

\subsubsection*{4) Quadratischer Term $G_{4}(I\omega)=1-(\frac{\omega}{\omega_{K}})^{2}+I2\Chi\frac{\omega}{ \omega_{K}}$, $\omega_{K}\bigger0$ :}
\subsubsection*{Betrag}
\begin{itemize}
    \item wieder 3 Fälle durchführen und für $G(s)=1+2\Chi\frac{s}{10} +(\frac{s}{10}^{2}$ und $G(s)=\frac{1}{1+2\Chi\frac{s}{10} +(\frac{s}{10}^{2}}$ plotten
    \item 4. Fall: Variation vom $\Chi=(0, 1]$. Man erkennt Resonanz- und Antiresonanz (und alles dazwischen)
\end{itemize}
Beispiel: Feder-Masse-Schwingkreis
Aufbau: Wand-Feder-Masse-Kraft
Mit Position x zwischen Wand und Massenzentrum
\[ m \dot\dot x = F-cx \]
\[ y=x \]
\[ \implies s^{2}m \hat{x}=\hat{F}-c \hat{x} \]
\[ \hat{y}=\hat{x} \]
\[ \implies G(S)=\frac{\hat{y}}{\hat{x}}=\frac{1}{s^{2}m+c}=\frac{1}{c}\frac{1}{1+s^{2}\frac{m}{c}}=\frac{1}{c}\frac{1}{1+(\frac{s}{\sqrt{\frac{c}{m}} })^{2}}=\frac{1}{c}\frac{1}{1+(\frac{s}{\omega_{K}})^{2}}\text{,} \Chi_{K}=0\]

Funfakt: Jede Struktur hat unendlich viele Resonanz- und Antiresonanzfrequenzen. Diese befinden sich je nach Struktur extrem weit oben (rechts). Da aber jede Struktur ein Tiefpassverhalten aufweist, sind die hohen Resonanzen rechts stark geschwächt. 

Beispiel: $G(s)=\frac{1}{1+s^{2}}$ 
Betrag sollte einfach sein und wird übersprungen.
Phase:
\[ arg(G(I\omega))=arg(1)-arg(1-\omega^{2})=\arctan(\frac{0}{1-\omega^{2}}) \]
Wieder für $\Chi=(0, 1]$ plotten. Siehe Abbildung 3.17 im Skriptum. Ist ein wenig komplizierter.

\subsection*{Darstellung von Bodediagrammen - Durchführung}
\subsection*{Amplitudengang}
\[ G(s)=\frac{10^{-2} 10}{0.01} \frac{1-\frac{s}{10}}{s(1+2 \frac{1}{2} \frac{s}{0.1}+(\frac{s}{0.1})^{2})}\]

\subsubsection*{normierte Form anschreiben}
\[ G(s)=\frac{10^{-2} 10}{0.01} \frac{1-\frac{s}{10}}{s(1+2 \frac{1}{2} \frac{s}{0.1}+(\frac{s}{0.1})^{2})}\]

\subsubsection*{Algebraisch Teilfunktionen}
\[ G_{1}=10, G_{2}=\frac{1}{s}, G_{3}=1-\frac{s}{10}, G_{4}=\frac{1}{1+2 \frac{1}{2} \frac{s}{0.1} }\]
Diese Teilfunktionen können separat aufgetragen und addiert werden.

\subsubsection*{Grafisch Teilfunktionen}
Weil die unterschiedlichen Knickzüge, nach der Knickfrequenz sortiert sind, können sie nacheinander kontinuierlich grafisch addiert werden.
Das geht natürlich schwer, wenn die Knickfrequenzen sehr knapp anneinander liegen. In dem Fall ergeben sich die Transienten durch Überlagerung. In solchen Fällen sollten die algebraischen Teilfunktionen aufgezeichnet werden. Die Addition dieser Teilfunktionen ist dann nicht mehr so wichtig und kann bei Prüfungen übersprungen werden.

\subsection*{Phasengang}
Hier kann sehr ähnlich gearbeitet werden, siehe Skriptum, kein Bock mitzuschreiben.

\subsection*{Phasenminimale Übertragungsfunktionen}
Eine Übertragungsfunktion $G(s)$ ist phasemminimal, wenn sich alle Pole in der linken offenen Halbebene befinden.

Beispiel: Die Sprungantwort eines Systems wird kurzzeitig negative, bevor sie sich im Positiven einstellt (siehe Versatz eines Pendels, das in der oberen Ruhelage liegt). Die kurzzeitige negative Phase zeigt, dass es eine rechte Polstelle gibt \implies das System ist nicht phasenminimal

\section*{09.11.2023}
\subsection*{Wiederholung/Rechentips?}
Gegeben sei
\[ G(s)=\frac{-(s-2)}{s^{2}+s+1} \].
Er plottet uns die Sprungantwort in Matlab. Zeigen Sie, wie man auf die gleiche Sprungantwort kommt.
\[ \lim_{t \to +0} y(t)=\lim_{s \to \infty} s \hat{y}=\lim_{s \to \infty} sG(s) \frac{1}{s}=0 \]
\[ \lim_{t \to \infty} y(t)=\lim_{s \to 0} s \hat{y}=\lim_{s \to 0} sG(s) \frac{1}{s}=G(0)=0 \]
\[ \lim_{t \to 0} \frac{d}{dt}y(t) =\lim_{s \to \infty} s(s \hat{y})=\lim_{s \to \infty} s^{2}G(s) \frac{1}{s}=\lim_{s \to \infty} \frac{-s^{2}+2s}{s^{2}+s+1}= -1\]auttex
Diese Eigenschaften können zur Konstruktion der Sprungantwort herangezogen werden.

\subsection*{3.8 Regelungstechnische Übertragungsglieder}
In der Praxis können natürlich nicht immer Sprungantworten herangezogen werden, da diese bereits destrukltiv für das zu untersuchende System wirken können.

\subsection*{3.8.1 PT1-Glied}
\subsubsection*{Anmerkung: Polstellen und Zeitkonstanten hängen direkt miteinander zusammen, daher sind unterschiedliche Problemstellungen möglic. (Polvorgabe oder Dynamikvorgabe)}

\subsection*{3.8.2 PT2-Glied}
\subsection*{3.8.3 P-Regler}
\subsection*{3.8.4 I-Regler}
\subsection*{3.8.5 PI-Regler}
\[ G(s)=\frac{V_{I}(1+sT_{I})}{s}=\frac{V_{I}}{s}+V_{I}T_{I} \]
Die Sprungantwort ist direkt aus der Geradengleichung ablesbar.
\subsection*{3.8.6 Differenzierer}
\subsection*{3.8.7 Proportional-Differential-Glied}
\subsection*{3.8.8 Lead-Glied (PD-T_{1}-Glied)}
Kommt immer wieder zur Prüfung!
\[ G=V \frac{1+sT}{1+s\eta T}\text{,} 0 < \eta < 1\]
Plot: Frequenzgang (V-kosnt -> gerade -> konstant$\frac{1}{T}$ und $\frac{1}{\eta T}$) Phasengang (Gauss-Glocke für $\frac{1}{T}$ und $\frac{1}{\eta T}$)
\[ \lim_{t \to +0} y(t)=\lim_{s \to \infty} s \frac{1}{s}G(s)=V \frac{T}{\eta T} \]
\[ \lim_{t \to \infty} y(t)=\lim_{s \to 0} s \frac{1}{s}G(s)=V \]
Sprungantwort des Lead-Glieds konstruieren: Sprung auf $\frac{V}{\eta}$ und exponentieller abstieg zu $V$.
Der Nenner dominiert immer die Dynamik, daher kann die Zeitkonstante ohne Zeichnen der Tangente durch Zerlegung der Funktion abgelesen werden.
Das Lag-Glied schaut ähnlich aus, startet jedoch unterhalb von V und nähert sich nach oben an.
Zur Bestimmung der maximalen Phasenanhebung kann die erste Ableitung des Arguments bestimmt werden. Solche Formeln stehen aber im Formelheft.

\section*{3.8.9}
\subsection*{3.8.10 Totzeit-Glied}
Er meinte, er wird sich im Rahmen dieser VO nicht mit der Lösung partieller Differentialgleichungen beschäftigen. Beim Totzeiglied macht er die Ausnahme.
Keine Ahnung was er macht aber Transport durch ein Rohr:
\[ k_{P}+\frac{K_{I}}{s}+k_{D}s \]
\[ y(t)=u(t-\frac{L}{V}) \]
\section*{Story-Time: Water-Hammer im Pumpspeicherkraftwerk}
Es treten sehr lange Transportleitungen auf, weshalb es zu Totzeiten im Sekundenbereich kommt. Daher muss man stark aufpassen, keinen Überdruck aber auch keinen Unterdruck zu erzeugen. Beim Unterdruck können Kavitäten erzeugt werden, die durch das System wandern und dann zum Beispiel an der Schaufel Schäden erzeugen.

\subsubsection*{Bodediagram}
\[ G(s)=e^{-sT_{t}} \]
\[ G(I\omega)=e^{-I\omega T_{t}} \]
\[ arg(G(I\omega))=-\omega T_{z}\]

\subsection*{3.9 Schaltungstechnische Realisierung}
Dieses Kapitel wird nicht geprüft.
Er wollte uns nur mal demonstrieren, wie elegant man Sachen machen kann.

\subsection*{3.10 Pol-Nullstellen Diagram}
\[ T=\frac{1}{\sqrt{\alpha^{2}+\beta^{2}} } \]
Hier sieht man sehr schön, weshalb Matlab Halbkreise für das Pol-Nullstellen-Diagramm einzeichnet.

Wenn als Problemstellung jetzt eine Polvorgabe gefordert ist, können diese Pole nun in diesem Diagramm eingetragen werden.
Durch Verschiebung entlang der Halbkreise kann die Dämpfung des Systems beeinflusst werden (siehe letztes Kapitel).

\section*{4 Der Regelkreis}
Regelkreis hier abbilden.
\begin{itemize}
    \item Führungsregelung
    \item Störregelung: Störgrössenaufschaltung -  Messung der Temperatur der Nordwand, da sonst die Totzeit zu hoch wäre
    \item Rauschen des Sensors
\end{itemize}

Die Identifikation eines Regelkreies ist sehr wichtig. Was ist die Strecke? Was ist der Regler? Was ist die Führungsgrösse? Alle Elemente sollten intuitiv erkennbar sein und das kann geübt werden.

Fun-Fact: Die englische Sprache hat keine eigenen Begriffe für Steuerung und Regelung
\begin{iteimize}
    open-loop-control: Steuerung
    closed-loop-control: Regelung
\end{iteimize}

Anmerkung zu Abb. 4.2: Sowas muss natürlich nicht auswendig gelernt werden und ist auch nicht Stoff der Vorlesung, es geht hier nur um die regelungstechnische Analyse.
In diesem System kommen einige Komponenten vor, unsere Ausgangsgrösse ist hier die drehzal $\omega_{2}$. Man sieht, dass die Drehzahl als Überlagerung von zwei Eingängen darstellen kann. Diese Strecke muss nun um ein Regelungsglied erweitert werden.

Die wesentlichen Aufgaben der Regelung sind folgende:
\begin{itemize}
    \item 1) Stabilisierung einer instabilen Strecke
    \item 2) das Ausgangssignal $y(t)$ soll dem Referenzsignal $r(t)$ möglichst gut folgen
    \item 3) die Störübertragungsfunktion kann durch $R(s)$ prinzipiell gezielt verändert werden
    \item 4) die Parameterschwankungen der Strecke $\Delta G(s)$ können in  $T_{r,y}(s)$ prinzipiell unterdrückt werden
    \item 5) das Sensorrauschen wirkt sich im Bereich guten Führungsverhaltens direkt auf die Ausgangsgröße aus 
\end{itemize}

\section*{14.11.2023}
\section*{Wiederholung}
Historische Zusammenfassung der Regelkreise:
\begin{itemize}
    \item Ägypter: Korkschwimmer als erste P-Regler
    \item Dampfmaschine: Erster moderner Regler. Hat sich noch nicht stark durchgesetzt, da die Fabrikation der Regler noch zu aufwendig war
    \item vor wenigen Jahrzehnten waren Echtzeit-regler npoch wunschdenken
    \item Heutzutage sind Echtzeitregler ein Kernthema, Stichwort: Cyber-Physical-Connections
\end{itemize}

Wir brauchen zur Regelung einer Strecke, einen Aktor. Wenn die Dynamik des Aktors in der gleichen Grössenordunng wie die Stecke liegt, so ist der Aktor Teil der Regelstrecke. Aktoren sind in der Praxis oftmals stark überdimensioniert. die Eingangsgösse des Aktors ist immer die Ausgangsgrösse des Reglers. Am Ausgang der Strecke findet man die Stellgrösse, mit der ein Soll-Ist-Vergleich durchgeführt werden kann. Der Sensor übersetzt die Ausgangsgrösse für den Regler.

Weiters wurde alles am Beispiel einer Antriebsregelstrecke besprochen. Hier sieht man wieder: Dynamik ist relativ, verglichen zum  Anker ist die Leistungselektronik beliebig schnell. Zur Abbildung 4.4 wird heute der Regler entworfen.

Siehe Abb. 4.5: Regelkreis als Übertragungssystem
Der Regler erhält eine verrauschte Rückkopplung und stellt eine Führungsgre $u$ ein. Da wir mit linear Funktionen arbeiten können wir Übertragungsfunktionen verwenden.

Wir müssen nun die Aufgaben einer unserer Regelstrecke erfüllen:
\begin{itemize}
    \item 1) solllte die Strecke instabil ien, müssen wir diese stabilisieren.
        Man denke an das AKW Fukushima oder an die Boeing die nur ein einem extern angebrachten emfpindlichen Sensor stabilisiert wurde
    \item 2) $T_{r,y}(I\oemga)=1$
    \item 3) $T_{d,y}(I\omega)=0$ 
    \item 4) $\frac{\Delta T_{r,y}(I\omega)}{T_{r,y}(I\omega)}=1$ 
    \item 5) $T_{n,y}(I\omega)=1$
\end{itemize}
Dies ist nicht sehr realistisch, daher wandeln wir das um auf:

\begin{itemize}
    \item 1) solllte die Strecke instabil ien, müssen wir diese stabilisieren.
        Man denke an das AKW Fukushima oder an die Boeing die nur ein einem extern angebrachten emfpindlichen Sensor stabilisiert wurde
    \item 2) $\|T_{r,y}(I\oemga)\|\ll1$
    \item 3) $\|T_{d,y}(I\omega)\|\ll0$ 
    \item 4) $\|\frac{\Delta T_{r,y}(I\omega)}{T_{r,y}(I\omega)}\|\ll1$ 
    \item 5) $\|T_{n,y}(I\omega)\|\ll=1$
\end{itemize}
\[ T_{r,y}(s)=\frac{\hat{Y}}{\hat{r}}, \frac{R(s)(G(s)+\Delta G(s))-R(s)G(s)}{R(s)G(s)}=\frac{\DeltaG}{G} \]
\[ T_{r,y}(s)=G(s)R(s) \]
Man erinnere sich: BIBO-Stabilität - Alle Polstellen haben Realteil kleiner 0? das hab ich fix falsch verstanden.

\subsection*{4.2.1 Einfache Steuerung}
Er zeigt uns ein Matlab-Projekt: Steuerung eines GleichstrommoterEr zeigt uns ein Matlab-Projekt: Steuerung eines Gleichstrommoters.
Wie kann man eine Steuerung entwerfen.
Man nehme an, es gelte $G(s)=\frac{1}{(sT)^{2}+2\xi sT+1}$
\[ R(s)=V \frac{(sT)^{2}+2\Xi sT+1}{(s+1)^{3}} \]
Es wird gewählt: $V=1\text{,}\ldots  $

zu Abb. 4.7: Steuerung mit Störgrössenaufschaltung
\[ T_{dy}(s)=\]
\[ R_{d}(s)=\frac{G_{d}(s)}{G(s)}\imples T_{dy}(s)=0 \text{\ldots dies ist nicht möglich}\]
\[ G_{d}(s)=\frac{Z_{d}(s)}{N_{d}(s)} \]
\[ G(s)=\frac{Z(s)}{N(s)} \]
\[ G(s)=\frac{1}{s^{2}+s+1} \]
\[ G_{d}(s)=\frac{1}{s+1} \]
\[ \frac{G_{d}(s)}{G(s)}=\frac{1}{s+1}(s^{2}+s+1) \]
\[ R_{d}(s)=\frac{Z_{d}(s)N(s)}{N_{d}(s)Z(s)} \]
Es darf zu keiner Kürzung kommen. (Kürzungen sind nämlich niemals praktisch möglich, da man nicht alle Nachkommerstellen kennen kann)

\subsection*{4.3 Regelung}
\[ T_{d,y}=\frac{R(s)G(s)}{1+R(s)G(s)}\]
\[ T_{d,y}(s)=\frac{G_{d}(s)}{1+R(s)G(s)} \]
\[ T_{n,y}(s)=\frac{-R(s)G(s)}{1+R(s)G(s)} \]
\[ irgendwas \]
\[ irgendwas \]

Demonstriert Implemantation in Matlab.

Merksatz: Als Konsequenz ist zu beachten, dass die Rückkopplung immer mit dem Stabilitätsproblem verbunden ist.

Def.: Interne Stabilität
Die Übertragungsfunktion des Geschlssenen Kreises muss stabil sein. We\ldots stimmt nicht ganz, bitte überarbeitenn
Beispiel: Einschleifiger Standardregelkreis
Strecke: $\frac{1}{s-1}$
Regler: $\frac{s-1}{s+1}$
Blockschaltbild:
      |
r--O--R--O--S--.--y
  |----------|
\[ T_{r,y}=\frac{
        \frac{s-1}{s+1}
        \frac{1}{s-1}
}{
    1+ \frac{s-1}{s+1}\frac{1}{s-1}
}=\frac{
    \frac{1}{s+1}
}{
    \frac{s+2}{s+1}
}=\frac{1}{s+2}\text{BIBO}\]

\[ T_{d,y}=
\frac{
    \frac{1}{s-1}
}{
    1+
    \frac{1}{s-1}
    \frac{s-1}{s+1}
}=\frac{s+1}{(s+2)(s-1)}\text{nicht BIBO wegen (s-1)}\]

Ein intuitives Verständis zur Interne Stabilität ist wichtig.

Nun folgt der Anschluss an die klassische Methodik, die oft an einer HTL übermittelt wird.
Blockschaltbild:
         |
r--O--R--G--O--.--y
  |----------|
Das R-G-Glied wird als offener Regelkreis bezeichnet $L(s)=R(s)G(s)$
\[ T_{r,y}=\frac{RG}{1+RG}=\frac{L}{1+L} \]
\[ T_{dy}=\frac{1}{1+RG}=\frac{1}{1+L} \]
Der Amplitudengang des offenen Regelkreises geht exponentiell nach unten und schneidet die Achse bei $\omega_{C}$
 \[ \omega\ll\omega_{c}:\|L\|\gg1\implies\|T_{r,y}\|approx 1, \|T_{dy}\|approx \frac{1}{\|L\|} \]
 \[ \omega\gg\omega_{c}:\|L\|\ll1\implies\|T_{r,y}\|approx \|L\|, \|T_{dy}\|approx \frac{1}{\|L\|} \]
Hiermit können nun $\|T_{ry}\|_{dB}$ und $ \|T_{dy}\|_{dB}$ plotten

nun könnte der Regler so entworfen werden, der im offenen Kreis an der Stelle $\omega_{s}$ einen Peak einbaut, welcher in $\|T_{dy}\|_{dB}$ den gleichen negativen Peak erzeugt.
Siehe Tafelbild

Im weitesten Sinne entspricht die Durchtrittsfrequenz der Bandbreite wie in Folie 285 beschrieben. Nun könnte man sich denken, dass man die Durchtrittsfrequenz beliebig erhöht und die Bandbreite somit beliebig wählt. Woran wirds scheitern? Am Aktor.

Als Vorbereitung für die nächste Stunde wurd in Matlab ein Antriebsreglerentwurf mit Sensorrauschen betrachtet. Die Durchtrittsfrequenz wurde so weit erhöht, dass das Sensorrauschen nicht mit im Unterdrückungsbereich der offenen Regelschleife lag. 

\section*{16.11.2023}
\section*{Wiederholung}
Korrektur: Bandbreite bedeutet, die Führungsübertragungsfunktion hat $3dB$ Abfall. In der englischen Literatur wir das ein wenig anders definiert.
Flext schonwieder wie ein König - Das Nobelpreiskommitee hat um einen Vortrag gebeten. Er hat das 3-fach Pendel vorgestellt und 2 Physiker haben ihn persönlich angesprochen und haben gemeint, dass das Video fake sei. Das 3-fach Pendel hat zwar nicht vorhersagbares Verhalten mit einer nicht lösbaren Differentialgleichung, kann aber in einem Closed-Loop-System stabilisiert werden.
Zur Erinnerung, jede Schwächung kann noch durch einen entsprechenden Aktor ausgeglichen werden, abe bereits bei  $-60dB$ bräuchte man einen 1000-mal stärkeren Aktor. Hochdynamische Systeme haben einen Faktor 100 an Überdimensionierung im Vergleich zu ihren stationären Betrieben.\newline
Das Sensorrauschen und die Führungsübertragungsfunktion haben bis auf ein Vorzeichen das gleiche Verhalten. Daher kann das Rauschen nie seperat unterdrückt werden und man muss bei der Wahl des Sensors entsprechend wählen. Wir leben von Information und wenn uns der Sensor die falsche Information liefert kann man dagegen nichts mehr machen!

 \section*{Sensitivitätsfunktion}
 \section*{Führungsübertragungsfunktion}
 
Es werden gerne glatte Führungssignale aufgeschaltet (am liebsten unendlich oft stetig differentierbar), ist aber oft nicht möglich. Beliebt sind auch Rampenfunktionen wegen ihrer einfachen superositionierbarkeit.

Kaskadenregelkreise werden mit sogenannten "Inneren Regelkreisen" innerhalb von Regelkreisen gebaut. Der innere Regelkreis sollte mit einer Dekade schneller als der äussere Regelkreis entworfen werden. Der innere Regelkreis entspricht einer Verstärkung von 1 und ist für den äusseren Regelkreis ideal. Er beeinflusst also den äusseren Regelkreis nicht und kein unsere gewünschte Zwischengrösse ideal erfassen. Nachteil: wir brauchen mehr Messgrössen (wir müssen die Zwischenngrösse auch messen)

\[ I \dot\dot\phi M_{el}-M_{L}=k_{A}i_{A}-M_{L} \]
\[ \dot\phi=\omega \]
\[ \dot\omega=\frac{k_{A}}{I}i_{A}-\frac{1}{I}M_{L} \]
Bitte hier Blockschaltbild von diesem Gleichungssystem hinzufügen.
Je nachdem was nun geregelt werden soll, kann man beispielsweise $i_{A}$, $\omega$ oder $\phi$ rückgeführt werden.

Wir sind nun fast soweit einen Reglerentwurf zu machen, vorher kommt noch die Stabilität.

Wenn alle Nullstellen des Nenners in der linken offenen Halbene liegen, nennt man dieses Polynom ein Hurwitzpolynom.
Notwendige Bedingung für ein Hurwitzpolynom: Die Koeffizienten sind alle von 0 verschieden und haben gleiches Vorzeichen.

\subsection*{Beispiel 4.1}
bis Gl. 4.34 sollte man das selbstständig hinbekommen.

Routh-Schema:
\[ \begin{bmatrix}
s^{3} && 1 && 9 && 0\\
s^{2} && 3 && 9K && 0\\
s1 && \frac{3\cdot 9-9k\cdot 1}{3}=9-3K && 0\\
s^{0} && \frac{(9-3K)9K-3\cdot 0}{9-3K}=9K\\
\end{bmatrix}  \]
Gleichungsystem lösen.

\subsection*{Beispiel}
\[ s^{4}+2s^{3}+3s^{2}4s+5 \]
\[]
\begin{bmatrix}
s^{4} && 1 && 3 && 5\\
s^{3} && 2 && 4 && 0\\
s^{2} && 1 && 5\\
s^{1} && -6 && 0\\
s^{0} && 5
\end{bmatrix} 
\]
Laut der -6 ist dies kein Hurwitzpolynom, falsches Vorzeichen.
Wir wissen, dass beim Abbauverfahren die Spalten nach und nach verschwinden. Werte die zur Berechnung des nächsten Wertes fehlen, werden zu 0 angenommen. 

\subsection*{Beispiel}
Wir haben einen Regelkreis R->G->-y rückgekoppelt
\[ T_{ry}(s)=\frac{\frac{V_{p}V}{1+s^{2}}}{1+\frac{V_{p}V}{1+s^{2}}}=\frac{V_{P}V}{s^{2}+1+V_{P}V} \]

\subsection*{Nyquist-Kriterium: Graphisches Kriterium}
\subsubsection*{Vorkenntnisse: Stetige Winkeländerung}
Was ist die stetige Winkeländerung. Wir betrachten vom Punkt 0 aus, wie sich der Winkel ändert. Am Beispiel der ortskruve aus Abb. 4.15 sieht man
\begin{iteimize}
\item P(-\infty): Pfeil dreht sich im Uhrzeigersinn und stoppt bei $-\frac{3\pi}{2}$
\item P(-0)->P(+0): wird nicht gezählt,  da ein Sprung nicht stetig ist
\item P(+0)->P(+\infty): Pfeil dreht sich im Uhrzeigersinn und stoppt bei $-\frac{3\pi}{2}$
\end{iteimize}
stetige Winkeländerung=$ 2\cdot \frac{-3\pi}{2}=-3\pi $

Für $s+1$ mit  $s_{i}=-1$
 stetige Winkeländerung = $\pi$
Für $s+1$ mit  $s_{i}=0$
 stetige Winkeländerung = $0$
Für $s+1$ mit  $s_{i}=1$
 stetige Winkeländerung = $-\pi$

\subsubsection*{Satz 4.5 Nyquist-Kriterium}
Gilt bedingung 4.45, so ist der geschlossene Regelkreis BIBO-stabil.

\subsubsection*{Beispiel}
 Für das Nyquist-Kriterium wird die stetige Winkeländerung immer vom Punkt -1 aus betrachtet. Das Beispiel wird im Skriptum ausführlich beschrieben.

 Das Nyquistkriterium gehört heutzutage nicht mehr wirklich zur Praxis. Bei transzendenten Funktion schaut das noch anders aus.

 \section*{Nächstes Mal}
 Wir werden die Phase auslegen, um Stabilität zu garantieren. Das wird ein Derivat des Nyquist-Kriteriums.Wir werden die Phase auslegen, um Stabilität zu garantieren. Das wird ein Derivat des Nyquist-Kriteriums.

 \section*{21.11.2023}
 \section*{Wiederholung}
 Sofern wir einen gekürzten Ausdruck fpr eine Übertragungsfunktion gegeben haben, können wir dessen Stabilität bestimmen.
 \[ G(s)=\frac{z(s)}{n(s)} \]
 \begin{itemize}
     \item Routh-Hurwitz-Kriterium
         Analyse des Nennerpolynoms
     \item Nyquistkriterium
         Analyse der Ortskurve des offenen Regelkreises
 \end{itemize}
 
 \section*{Vereinfachtes Nyquist-Kriterium}
Wenn die 4 Bedingungen lt. Skriptum gegeben sind gilt: Phasenreserve muss positiv sein \implies stabil

\section*{Kenngrössen der Sprungantwort}
\subsection*{Anstiegszeit $t_{r}$}
Es gibt einige Definition zu Anstiegszeiten. Eine Norm definiert dies als Zeit von 10% bis 90%. In der VO verwenden wir eine Tangenten am Wendepunkt und schneiden diese mit 0% und 100%.
\subsection*{Überschwingweite $ü$}
\subsection*{bleibende Regelabweichung}
\subsubsection*{Empirische Zusammenhänge}
\[ \omega_{C} t_{r}\approx 1.5 \]
\[ \Phi[°]+ü[%]\approx 70% \]
bleibende Regelabweichung: siehe Tabelle 5.10
\subsubsection*{Herleitung: Bleibende Regelabweichung}
Rückkoppelkreis:
r--+--R--G(s)--+--.--y
   |--------------|
\[ \hat{e}=T_{de}(s) \hat{d} \]
\[ T_{de}(s)=\frac{-1}{1+R(s)G(s)} \text{\ldots Mittels Formel (gilt natürlich nur für einschleifige Standardregelkreise)}\]
\[ -\hat{e}=\hat{y}=\hat{d}+R(s)G(s)\implies \frac{\hat{e}}{\hat{d}}=\frac{-1}{1+RG} \text{\ldots selbst hergeleitet}\]

\[ \lim_{t \to \infty} e(t)=0=\lim_{s \to 0}\hat{e}=\lim_{s \to 0} s \frac{1}{1+R(s)G(s)} \frac{1}{s}=\lim_{s \to 0} \frac{-1}{1+\frac{Z_{R}}{N_{R}} \frac{Z_{G}}{N_{G}}}=\lim_{s \to 0} \frac{-N_{R}(s)N_{G}(s)}{N_{R}(s)N_{G}(s)+Z_{R}(s)Z_{G}(s)}=0  \]
\[ \implies N_{R}(0)N_{G}(0)=0 \]

Rückkoppelkreis:
         |$d=\sigma(t)$
r--+--R(s)--+--G(s)--.--y
  |-            |
  |-------------|
  \[ T_{de}(s)=\frac{-G(s)}{1+R(s)G(s)}=\frac{\hat{e}}{\hat{d}} \]
  \[ \lim_{t \to \infty} e(t)=\lim_{s \to 0} s \frac{-G(s)}{1+R(s)G(s)} \frac{1}{s}=\lim_{s \to 0} \frac{-\frac{Z_{G}}{N_{G}}}{1+ \frac{Z_{R}}{N_{R}} \frac{Z_{G}}{N_{G}}}=\lim_{s \to 0}  \frac{-Z_{G}N_{R}}{N_{G}N_{R}+Z_{G}Z_{R}}=0 \]
  \[ \implies Z_{G}(0)N_{R}(0)=0 \]
  
\subsection*{Annahme: $d(t)=\sind(2t)$}
\[ e(t)=\|T_{de}(I2)\|\sin(2t+arg(T_{de}(I2))) \]
\[ T_{de}(s)= \frac{-Z_{g}N_{R}}{N_{G}N_{R}+Z_{G}Z_{R}} \]
\[ N_{R}(s)=(1+\frac{s^{4}}{4})\tilde{n}_{R} \]
\[ N_{R}(2I)=0 R(s)=\frac{V Z_{R}}{(1+\frac{s^{2}}{4})\tilde{N}_R}\]

\subsection*{5.1 PI-Reglerentwurf}
Streckenübertragungsfunktion: $ G(s)= \frac{5}{1+2 \cdot 0.707s+s^{2}} $

\[ L_{1}(s)=\frac{5}{s(1+2\cdot 707+)} \]
Im Skriptum nachlesen. Ist perfekt dokumentiert. Beispiel auch als Matlab-Projekt verfügbar.

\subsection*{5.2 Lead-Lag-Reglerentwurf}
Streckenübertragungsfunktion: $G(s)=\frac{
    irgendwas
}{
    irgendwas
}$ 

siehe Skriptum

Wir sehen, dass die Phase gehoben und die Amplituden gesenkt werden muss. Normalerweise steigen/fallen die beiden Gren Zusammen (denke an Bedeutungen im Zähler und Nenner) \implies das schreit nach einem Lead-Lag-Reglerentwurf!

Wir heben zunächst die Phase an, aber heben es um 10°(Richtwert: 15° und 20° sind auch noch möglich) zu weit an.\newline
\implies Lead-Glied

Nun sind sowohl Amplitude alsauch Phase zu hoch und können gemeinsam gesenkt werden.\newline
\implies Lag-Glied

\subsection*{PID-Regler per Hand einstellen}
Integralanteil abschalten: liefert nämlich -90° und verschlechtert die Phasenreserve
P-Regler einstellen, damit es sich schön einschwingt
D-Anteil einstellen, damit der Schwung entfernt wird (liefert nämlich +90° und verbessert die Phasenreserve)
I-Anteil hochdrehen, P-Anteil senken und D-Anteil erhöhen

Ihm ist folgendes wichtiger: wir verstehen, dass wir hier Phasenreserven und Knickfrequenzen verschieben

\section*{Nächstes Mal}
Kompensationsreglerentwurf mit Notchfilter

Danach ist der erste Teil des Skriptums durch. Kapitel 6 ist wieder ein kompletter Neuanfang!

\section*{28.11.2023}
\section*{Wiederholung}
Beim letzten Mal haben wir uns die Dynamik angeschaut und Kenngrössen der Sprungantwort mit Kenngrössen des Bodediagrams in Verbindung gebracht.

\section*{5.3 Kompensationsreglerentwurf am Beispiel der Antriebsregelstrecke aus Abb. 4.3}
Wir erkennen, dass der offene Regelkreis eine Resonanzerhöhung hat, diese wollen wir mit einem Regler kompensieren. Wir kürzen mit unserem Regler also den Resonanzterm weg.
Damit der gewählte Regler realisierbar ist, müssen wir die Ordnung des Nenners erhöhen. \newline
Hinweis: Wenn der Zählergrad und Nennergrad gleich sind, ist der Regler sprungfähig. Wir können stabile Nullstellen kürzen wie wir sollen und auch beliebige Regler vorgeben wie wir wollen, solange der Nennergrad grössergleich Zählergrad ist, aber sprungfähige Regler sind schlecht.\newline
Nun müssen wir mit unserem Regler noch die gewünschte Phasenreserve berücksichtigen.

\section*{6 Der Digitale Regelkreis}
\subsection*{6.1 Allgemeines}
Es ändert sich nicht viel, nur die Schnittstelle zum Aktor und vom Sensor zurück zum digitalen Regler.

Notation: $y_{k}=y(kT_{a})$

Sprünge werden mit dem rechtsseitigen Grenzwert definiert.

Zur Quantisierung wird ein Halteglied mit einem Abtastwert benutzt. Das "zero-oder-hold" Halteglied hält den zuletzt abgetasteten Wert bis zum nächsten Abtastzeitpunkt aufrecht.

\subsection*{6.2.1 Der nichtlineare Fall}
Wir schalten eine Folge $(u_{k})$ auf und erhalten eine Ausgangsfolge $(y_{k})$.
Für $kT_{a}<t<(k+1)T_{a}$ gilt $u_{k}=u(kT_{a})$\ldots siehe Skriptum.

\subsection*{6.2.2 Linear, zeitinvariant}
Integral aus Gleichung (6.18a) läasst sich vereinfachen:
\[ \int_{kTa}^{(k+1)T_{a}} \Phi((k+1)T_{a}-\tau)d\tau  \]
Zur Supstitution:
\[ \overline{\tau}=(k+1)T_{a}-\tau \]
\[ \tau=kT_{a} \to \overline{\tau}=\tau_{a}\]
\[ \tau=(k+1)T_{a} \implies \overline{\tau}=0 \]

\[ \int_{T_{a}}^{0} \Phi(\overline{\tau})(-d \overline{\tau})=\int_{0}^{T_{a}} \Phi(\overline{\tau}) d \overline{\tau}  \]

\[ \dot x=Ax+Bu \]
\[ y=Cx+Du \]

\[ x_{k+1}=\Phi x_{k} + \Gamma u_{k} \]
\[ y_{k}=Cx_{k}+Du_{k} \]

\[ \Phi=e^{AT_{a}} \]
\[ \Gamma=\int_{0}^{T_{a}e^{A\tau}d\tau B}   \]

Beispiel:
\[ \frac{d}{dt}x=Ax+bu=\begin{pmatrix}
0 && 1\\
0 && 0
\end{pmatrix} x+\begin{pmatrix} 0\\ 1\end{pmatrix}u \]
\[ y=c^{T}x=\begin{pmatrix} 1&& 0\end{pmatrix}x \]
Das Beispiel ist so einfach, dass es im Skriptum im Schnellformat mit einer Abkürzung gerechnet wird.

\section*{6.3 Die Transistionsmatrix}
\[ x_{k+1}=\Phi x_{k} \text{,}x_{0} \]
Durch Iterations erhält man die Transitionsmatrix des Abtastsystems
\[ x_{1}=\Phi x_{0} \]
\[ x_{2}=\Phi\Phi x_{0} \]
\[ x_{k}=\Phi^{k}x_{0}=\Psi(k)x_{0} \]

Aus dieser Erkenntnis folgen sofort die Beziehungen:
\begin{iteimize}
\item $\Psi(0)=E$
\item $\Psi(k+i)=\Psi(k)\Psi(i)$
\item \ldots  (siehe Skriptum
\end{iteimize}

Bald folgt die Finite Elemente Methode, ich sags dir.


Beim nächsten Mal kommt die Z-Trasnformation.

Dann haben wir die Zusammenhänge.
Differentialgleichung \implies s-Übertragungsfunktion\newline
Differenzengleichung \implies z-Übertragungsfunktion


\end{document}
