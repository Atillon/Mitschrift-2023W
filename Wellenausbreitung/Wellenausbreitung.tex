\documentclass[a4paper]{article}

\usepackage[a4paper,top=2cm,bottom=2cm,left=3cm,right=3cm,marginparwidth=1.75cm]{geometry}
\usepackage[utf8]{inputenc}
\usepackage[T1]{fontenc}
\usepackage{textcomp}
\usepackage[ngerman]{babel}
\usepackage{amsmath, amssymb, nccmath}
\usepackage{accents}

% figure support
\usepackage{import}
\usepackage{xifthen}
\pdfminorversion=7
\usepackage{pdfpages}
\usepackage{transparent}
\newcommand{\incfig}[1]{%
    \def\svgwidth{\columnwidth}
    \import{./figures/}{#1.pdf_tex}
}

\pdfsuppresswarningpagegroup=1

\title{Mitschrift\\Wellenausbreitung - Wintersemester 2023}
\author{DINC Atilla (11917652)}

\begin{document}
\normalsize
\maketitle
%\tableofcontent\newpage

% ~~~~~~~~~~~~~~~~~~~~~~~~~~~~ Start of the document ~~~~~~~~~~~~~~~~~~~~~~~~~~~~
\section*{2.1 Ladungskontinuität}
Maxwell ergänzt in der letzten Maxwellschen Gleichung die Summe der Leitungsströme mit der Änderung der Verschiebungsstromdichte.

\[
\nabla \cdot (\nabla \times H) = \nabla \cdot S + \nabla\cdot\frac{dD}{dt}
= \nabla\cdot S + \frac{d\nabla\cdot D}{dt}
= \nabla\cdot S + \frac{d\rho}{dt}
\]

\section*{2.2.1 Hilfsgleichung}
Problem: 8 Maxwellgleichungen für 12 unbekannte, daher werden Hilfgleichungen benötigt.\newline
el. und mag. FLussdichtegleichungen gelten in isotropen Medien (linear \& homogen?)


\subsection*{$\rho$ (ausgeprochen Rohacel)}
\[\nabla\cdot(\nabla x H)=\nabla\cdot (S+\frac{dD}{dt})\]
\[\nabla xE = -\frac{dB}{dt}\]
\[\nabla \cdot D=\rho\]

(effektive Ladungsfreiheit bei $f\le 10^18 Hz$ - bei diesen Wellenlängen ist $\lambda$ kleiner als der Atomradius, da hören die Kontinuitätsannahmen auf)\newline
$\nabla\cdot B=0$\newline
aus 1 und 3:

\[0 = \nabla\cdot S+\frac{d\rho}{dt} \text{folgt} \sigma \nabla\cdot E+\frac{drho}{dt}=0\]
\[\sigma \rho\epsilon+\frac{d\rho}{dt}=0\]
\[\rho (x,y,z,t) \text{with} (x,y,z) \in \mathbb{R}^{3}, t\in 0, inf \]
\[\rho(x,y,z,t)=\rho(x,y,z)\cdot e^{-\frac{t}{\tau_{D}}}\]
\[S = \sigma E\]
\[D=\epsilon E\]
\[\frac{\sigma}{\epsilon} \rho_{0}(x,y,z)\cdot e^{\frac{-t}{\tau_{D}}} + \rho_{0}(x,y,z)(\frac{-1}{\tau_{D}})e^{\frac{-t}{\tau_{D}}}=0\]
\[\frac{\sigma}{\epsilon}= \frac{1}{\tau_{D}}, \tau_{D}=\frac{\epsilon}{\sigma}\]
allgemeine Lösung = homogen + partikulär
\[\rho(x,y,z) = \rho_{0}(x,y,z)\cdot e^(-\frac{t}{\tau_{D}}) + \text{partikuläre Lösung}\]
\[\int_{\mathbb{R}^3}(\rho(x,y,z,t))dxdydz = \int_{\mathbb{R}^2}(\rho_{0}(x,y,z)\cdot e^\frac{-t}{\tau_{D}})dxdydz\]
\[= e^\frac{-t}{\tau_{D}}, Q(t)=e^\frac{-t}{\tau_{D}}\cdot Q_{0}\]
Anfangsladungsverteilung:
\[Q_{0}=rect_{-t_{0}}(t) - rect_{t_{0}}(t)\]
Wobei bei Kupfer $\tau_{D} = 8,854.10^-12 F/m / 60.10^6S/m~=10^-18s$ gilt. 

$H(x,y,z,t)=Re\{\underline{H}(x,y,z)e^{jwt}\}$ ...Notiz: Spitzenwertzeiger $\underline{H}$

\[\nabla \times \underline{H} = \underline{S} + j\omega \underline{D} = \sigma\cdot \underline{E} + j\omega\epsilon \underline{E} = (\sigma + j\omega\epsilon)\cdot \underline{E} = j\omega\epsilon(1+\frac{\sigma}{j\omega\epsilon})\cdot \underline{E} = j\omega\epsilon(1 - \frac{j\sigma}{\omega\epsilon})\cdot \underline{E}\]
$\nabla \times \underline{H}= j\omega \delta \underline{E}$...$\delta$ = effektive Permitivität
\[\nabla \times  \underline{H} = -j\omega\mu \underline{H}\]
\[\nabla\times\underline{H} = -j\omega \underline{B}\]
\[\nabla\cdot\underline{D} = -j\omega \underline{B}\]
\[\nabla\cdot\underline{B} = 0\]

\section*{Poyntingscher Satz}
\begin{itemize}
    \item Gleichungen mit $H$ und $E$ multiplizieren
    \item Gleichungen subtrahieren
    \item den Term $\nabla\cdot (E\times H)$ mittels Produktregel + Spatprodukt umformen auf $H\cdot (\nabla\times E) - E\cdot (\nabla\times H)$
    \item Weiter Produktregel an $E$ und $H$ Termen anwenden
\end{itemize}

\[x\cdot \frac{dx}{dt}= |x|^2/2\]
\[\nabla\cdot (E\times H) + \sigma |E|^2 = -\frac{d}{dt}(\frac{\epsilon}{2} |E|^2 + \frac{\mu}{2}|H|^2)\]
Abstrahlung + Heizung = elektromagnetische\newline
Die dielektrischen Verluste wachsen direkt proportional zur Frequenz.

\[E(x,y,z,t)=Re\{E(x,y,z,t)\cdot e^{j\omega t}\}\]
\[H(x,y,z,t)=\frac{1}{2}(\underline{H}(x,y,z,t)\cdot e^{j\omega t} + \underline{H}\ast(x,y,z)\cdot e^{-j\omega t})\]
\[E(x,y,z,t)\times H(x,y,z,t)=Re\{\underline{E}(x,y,z)\cdot e^{j\omega t}\} \times \frac{1}{2}(\underline{H}\cdot e^{j\omega t}+\underline{H}\ast\cdot e^{-j\omega t})\]
\[=Re\{\frac{1}{2}\underline{E}\times \underline{H}\ast + \frac{1}{2}\underline{E}\times \underline{H}\cdot e^{2j\omega t}\} = \text{stationärer + pendelnder Anteil}\]
\[=Re\{\frac{1}{2}\underline{E}\times \underline{H}\ast\} + \ldots\]

komplexer Poyntingvektor: \[\underline{T}=\frac{1}{2}\underline{E}\times \underline{H}\ast= \underline{T}_{\omega}+j \underline{T}_{b}\]

\section*{Tägliche Trivia}
Quellen und Wellen hängen irgendwie zusammen, Antennen sind gerne mal Quellen.
Der Begriff der Antenne kommt aus dem italienischen Wort für Zeltstange.
Der Dipol ist die elementare Antenne.\newline
Merksatz: Die niedrigste Strahlungsordnung ist die Dipolladung.
Eine Druckwelle einer Explosion ist eine Monopolstrahlung, die Erdbebenwelle ist
eine Dipol- oder sogar Quadropolstrahlung.
Die Kräfte einer Explosion gehen radial weg, während im Zentrum eines Erbebens
Platten mit entgegengesetzten Kräften aneinander reiben.\newline

Dinge des Tages:
\begin{itemize}
    \item Gerade Einzeldrahtantenne mit induktiver Kopplung in der Mitte. Die Kopplung teilt die Antenne in zwei Teile für kurz- und langwellige Signale.\newline
       Skizze: ------a------===------b------
                $\lambda_{a}~=70cm | \lambda_{b}~=100cm$
    \item Koaxialkabel
    \item Wifi-Antenne: Diese haben "gespiegelte" Stecker, damit normale Koaxialantennen nicht in einen Router passen. Mehr dazu in Kapitel 11
\end{itemize}

Also, was war nochmal eine Antenne? Wandler zwischen raumgebundenen und leitungsgebundenen Wellen.

\section*{2.4 Randbedingungen}
Wiederholung: Faraday'sches Gesetz

\section*{2.1 Grenzflächen zwischen Dielektrika}
Beispiel: Oberflächenintegral über kleines Rechteck an Grenzfläche.
$\Delta l$ ist größer als $\Delta x$, aber kurz genug, damit $E_{tangential}$ an den beiden kurzen Kanten konstant bleibt. Siehe Resultat im Skriptum.
Analoges kann für die magnetische Feldstärke durchgeführt werden.\newline

Ähnliche infinitesimale Methode für die Divergenzgleichungen.
Die elektrischen Flächenladungen die in den Ergebnissen auftreten, spielen in
dieser Vorlesung keine Rolle und daher gibt es auch keine Beispiele dazu.
Die Flächenstromdichte spielt sehr wohl eine Rolle (Skineffekt).\newline\newline
Sommerfeld'sche Ausstrahlungsbedingung: Wellen verschwinden in der Unendlichkeit\newline
(Mathematische Formulierung wird erstmal nicht durchgeführt)

\section*{2.5 Lösung der Wellengleichung in kartesischen Koordinaten}
\bold{Sei angemerkt:} unter hinreichenden Bedingungen gilt der Satz von Schwartz.
\[\nabla \times H = \sigma E +\epsilon \frac{\partial}{\partial t}  E\]
\[\nabla x (\nabla x H) = sigma(\nablaxE)+epsilon partd/partdt(\nabla x E)\]
\[\nabla(\nabla H)-\delta H = \sigma-\mu \frac{\partial H}{\partial t} + \epsilon \frac{\partial }{\partial t}(-\mu \frac{\partial H}{\partial t})\]
\[\delta H = \mu \sigma \frac{\partial H}{\partial t} + \epsilon\mu \frac{\partial ^2 H}{\partial t^2}\] \ldotsTelegraphengleichung für $H$,  $B$, $E$ \& $D$\newline
In dieser VO wird hauptsächlich auf den Separationsansatz eingegangen.
end
$H=Re\{\underline{H}(x,y,z)\cdot e^{j\omega t}\}$ \ldots $\frac{\partial }{\partial t}$ wird Laplace-transformiert\newline
$\delta \underline{H} = j\omega\mu\sigma\underline{H}\epsilon\mu\omega^2 \underline{H}\ldots \delta \underline{H}+(\omega^2 \mu\epsilon j\omega\mu\sigma)\underline{H} = \underline{0}$? \ldots Helmholtzgleichung\newline

HÜ: Empfehlung: Das gleiche für E wiederholen.
Nächste Woche Di: Seperationsansatz (sollte man aus Mathe noch kennen)

\subsection*{17.10.2023}
\section*{2.5.1 Lösungsansätze}
\begin{itemize}
    \item Methode 1: $E=a\cdot \psi$\newline
    \item Methode 2:\newline
    Reines Wirbelfeld E bzw H=\nabla x (a.psi)=-a x \nabla.psi
    Durch Wiederholung: \nabla x ( \nabla x (a.psi))
    Mit genügend Wiederholungen kann die Elektrodynamik gelöst werden.
    \item Methode 3: Ansatz mittels elektrodynamischer Potentiale
    Ansatz mittels Vektorpotential
\end{itemize}

\section*{2.5.2 Separationsansatz}
Zu lösen:
\[\frac{\partial^2 \psi}{\partial x^2} + \frac{\partial ^2\psi}{\partial y^2} +\frac{\partial ^2\psi}{\partial z^2}  + \omega^2\mu\sigma\psi=0\]
\[\psi(x,y,z)=X(x)\cdot Y(y)\cdot Z(z)\]
\[\frac{\partial ^2\psi}{\partial x^2}\cdot YZ + X\cdot \frac{\partial ^2\psi}{\partial y^2}\cdot Z + XY\cdot \frac{\partial ^2\psi}{\partial z^2} + \omega^2 \mu\sigma\psi\cdot XYZ=0\]
\[\frac{1}{X(x)} \cdot \frac{\partial ^2\psi}{\partial x^2} + \frac{1}{Y(y)}\cdot \frac{\partial ^2\psi}{\partial y^2} + \frac{1}{Z(z)}\cdot \frac{\partial ^2\psi}{\partial z^2} + \omega^2\mu\sigma=0\]\newline
Aufgrund des konstanten Terms folgt:
\[k\cdot x^2 + k\cdot y^2 + k\cdot z^2 = \omega^2\mu\sigma\]\ldots Seperationsansatz

\[\frac{1}{X}\cdot \frac{\partial ^2X(x)}{\partial x^2} = -k\cdot x^2\]\newline
folglich erhält man die Schwingungsgleichung. Analog können die Schwingungsgleichungen für $y$ und $z$ bestimmen und man erhält das Gleichungssystem
\[X''(x)+k\cdot x^2\cdot X(x) = 0\]
\[Y''(x)+k\cdot y^2\cdot Y(y) = 0\]
\[Z''(x)+k\cdot z^2\cdot Z(z) = 0\]

Zur Bestimmung einer Lösungen können unterschiedliche Formen angesetzt werden. Die nützlichsten sind:
\begin{itemize}
    \item \bold{Fundamentalsystem:} $\{\sin(k_{x}x), \cos(k_{x}x)\}$\ldots häufig verwendet
    \item \bold{Komplexes Fundsystem:} $\{e^{-jk_{x}x}, e^{jk_{x}x}\}$\ldotshäufig verwendet
\item \bold{Mischform:} $\{\sin(k_{x}x)m e^{jk_{x}x}\}$, etc.
    \item \bold{Hyperbelfunktion:} $\{\cosh(k_{x}x), \sinh(k_{x}x)\}$
\end{itemize}

Physikalische Bedeutungen in Tabelle 2.1 können durch einsetzen in unsere Wellenfunktion $Re\{\psi(x,y,z)\cdot e^{j\omega t}\}=Re\{\sin(k_{z}z)\cdot e^{j\omega t}\}$ überprüft werden. Die Wellenfront verläuft in z-Richtung.\newline
Fortan wird die verkürzte Schreibweise $\frac{\partial}{\partial x}  = \partial _{x}$ verwendet.\newline

\[(2.39) \partial_{y} H_{z} - \partial_{z} H_{y}= \frac{j\omega\sigma E_{x}}{j\omega\mu}\]
\[(2.43) \partial_{z}E_{x} - \partial_{x}E_{z} = -j\omega\muH_{y} \partial_{z}\]
Über 2 Schritte folgt: $j\omega\mu\cdot \partial_{y}H_{z}+\partial_{z}^2 E_{x}-\partial_{z}\partial_{x}E_{z} = -\omega^2\mu\sigma\cdot E_{x}$\newline
Annahme Wanderwelle nach $z=\infty$\ldots $j\omega\mu \partial_{y}H_{z} - k_{z}^2\cdot E_{x} + jk_{z}\cdot \partial_{x}E_{z}=-\omega^2\mu\sigma E_{y}$
\[(\omega^2\mu\sigma-k_{z}^2)E_{x} = -j\omega\mu \partial_{y}\cdot H_{z}-jk_{z} \partial_{y}E_{z}\]
\[E_{x}=-j\frac{1}{\kappa^2}\cdot (k_{z}.\cdot \partial_{x}E_{z}+\omega\mu\partial_{y}H_{z})\]\ldots sollte die erste Gleichung (2.45 ergeben)
HÜ: Für 2.46 bis 2.48 wiederholen

\[\kappa^2=0\]
\[\omega^2\musigma-k_{z}^2=0\]
\[k_{z}=\pm \frac{\omega}{\sqrt(\mu\epsilon)}\]
\[\kappa^2\neq 0\]
\[\omega^2\mu\sigma-k_{z}^2=k_{x}^2+ k_{y}^2\]\ldots Der Betrag der k-Vektors bleibt gleich, daher wird die $k_{z}$ Komponente wegen $k_{x}$ und $k_{y}$ kleiner
Aus dieser Erkenntnis zeigt sich: Die TEM-Welle ist die schnellste Welle.
Notiz: Eingerahmte Formeln müssen auswendig gelernt werden, außer die Modalen Lösungen in kartesischen Koordinaten.\newline
Stattdessen das $\kappa^2$ auswendig lernen!

\section*{3. Die homogene ebene Welle}
\section*{3.1 Die HEW im idealen Dielektrikum}
$e$\ldots elektrische Feldstärke\newline
$h$\ldots magnetische Feldstärke\newline
Kleinbuchstaben zur Andeutung der Zeitabhängigkeit.

\[\partial_{z}^2 e_{x}(z,t) - \mu\epsilon\partial_{z}^2 e_{x}(z,t) = 0\]
mit $e_x(z,t)=f_{1}(z-vt)+f_{2}(z+vt)$ eingesetzt folgt (nur mal für f1):
\[f_{1}''(z-vt)-\mu\epsilon f_{1}''(z-vt)(-v)^2=0\]
\[1-\mu\epsilon v^2 = 0 \implies v=\pm\frac{1}{sqrt(\mu\epsilon)}\]

\[h_{y}^+=\frac{1}{\nu}\cdot e_x^+\ldots \nu=\sqrt(\frac{\mu}{\epsilon}), \nu_{0}=\sqrt(\frac{µ_{0}}{\epsilon_{0}})=\sqrt((4\cdot \pi\cdot 10^{-z}\frac{H}{m})^2\cdot c_{0}^2)=12\cdot \pi\cdot 10 \Omega = 120\cdot \pi\Omega\] \ldots Feldwellenwiderstand (hat nichts mit ohmschen Verlusten zu tun)
\[\epsilon_{0}\cdot µ_{0}=\frac{1}{c_{0}^2} = \frac{1}{\mu_{0}\cdot c_{0}^2}\]

\section*{19.10.2023}
\section*{Ding des Tages}
Der Prof. baute sich einmal eine Antenne, um mit der ISS beim FAQ mit einer Schule zuzuhören. Eine 50cm Stange nach links, eine 50cm Stange nach rechts. Die Stangen gehen per Bananenstecker in eine mysteriöse graue Box. Von unten wird ein Koax-Kabel angesteckt, bissl qualitativer, nicht so wie das vom grindigen Fernsehen.\newline
Es wird ein Balun benötigt (Bal=balanced, un=unbalanced). Dieses schließt eine balanzierte Leitung an eine unbalanzierte Leitung an.\newline
Abb.: Tafelbild + Diashow
Weil die Ströme gegensinnig, die Wicklung aber gleichsinnig ist, heben sich die Flüsse auf \implies $L=0, \omega_0=0$
Da bereits eine geringe Stromdifferenz eine relativ hohe Impedanz erzeugt, bestraft dieser Balun Differenzströme.\newline
Weil die Schule unterhalb des Horizontes lag, konnte der Prof. nur der ISS, nicht aber der Schule zuhören.
Der Prof. hat auch die Aufnahme zur Verfügung gestellt.

\subsection*{3.1.3 Energiedichte der HEW, Poyntingscher Vektor}
Zu jedem Zeitpunkt, an jedem Ort, ist die elektrische Energiedichte gleich groß wie die magnetische Energiedichte: sie gehen synchron.\newline
Abb. 3.1 Man sieht die Energiedichte eines Photons. Dort wo die Pfeile näher zusammen liegen, ist die Feldstärke erhöht. Man kann einen sinusförmigen Verlauf der Feldstärke erkennen.\newline
Weiters ist die Homogene Elektromagnetische Welle im freien Raum dargestellt. Unbedingt einprägen!\newline
\subsection*{3.1.4 Wellenzahl und Wellenlänge}
Unterschied zwischen $k$ und $\omega$.
\section*{Polarisation}
Alle haben eine Polaristation. Unpolarisierte sind nur im mittel nicht polarisiert.
\subsection*{Polaristationsarten}
\subsubsection*{elliptische Polarisations (allgemeine Form)}

\section*{07.11.2023}
\section*{Ding des Tages}
Ehemalige Diplomarbeit:\newline
Eine helixförmige Kunststofflinse wandelt eine ebene harmonische Welle in einen Gauss-Laguerre-Beam um. Die "Wendeltreppe" verzögert wellen an unterscheidlcihen Stellen unterschiedlich lang, wodurch eine ebene harmonische Welle spiralförmig Phasenverschobene ebene harmonische Welle umgewandelt wird.
Durch diese Konstruktion kann ein Empfänger-Linsen Paar in 4 unterschiedlichen Kombinationen aufgebaut werden, wodurch die Bandbreite vervierfacht werden kann. Siehe Anhang für mehr.

Er überspringt ein wenig Stoff und niemand traut sich was zu sagen
\section*{4.1 Grenzfläche  zwischen \ldots}
Wir betrachten den TM-Fall.

Die geometrischen Zusammenhänge werden erstmal festgelegt. Die Annahme, dass der Winkel der einfallenden und ausfallenden Welle gleich ist, wird noch ausgelassen.
\subsubsection*{Ansätze anschreiben}
\begin{itemize}
    \item einfallende Welle $\vec{E_{e}}$
    \item reflektierte Welle $\vec{E_{r}}$
    \item transmittierte Welle $\vec{E_{t}}$
    \item Randbedinungen bei $z=0$:
        \[ E_{tang}\text{stetig bei} z=0: \ldots \]
        \[ D_{norm} \text{ stetig bei} z=0; \ldots\]
        \[ \implies \text{Reflexionsgesetz:} \Theta_{e}=\Theta_{r}=\Theta_{1}\]
        \[ \text{Snelliussches Brechungsgesetz:} \frac{\sin(\Theta_{1})}{\sin(\Theta_{2})}=\sqrt{\frac{\epsilon_{2}}{\epsilon_{1}}}=\frac{n_{2}}{n_{1}}  \]
\end{itemize}
Durch Einsetzung:
\[ \begin{pmatrix}
-\cos(\Theta_{1}) && \cos(\Theta_{2})\\
\epsilon_{1}\sin(\Theta_{1}) && -\epsilon_{2}\sin(\Theta_{2})
\end{pmatrix} \cdot 
\begin{pmatrix} \Gamma_{TM}\\ T_{TM}\end{pmatrix}
=\begin{pmatrix} \cos(\Theta_{1})\\ -\epsilon_{1}\sin(\Theta_{1})\end{pmatrix}
\]

Im Skriptum findet man die Fresnelschen Formeln für den TE-Fall und dne TM-Fall. Muss nicht auswendiggelernt werden aber sollte verstanden werden.
Er legt uns stark ans Herz, dass es einmal selbstständig hergeleitet/angwandt werden soll. Hört sich einfach an, aber für den Prüfungsfall ist das eine sehr wichtige Übung.

Zu merken: $\tan(\THeta_{B}) =\sqrt{\frac{n^{2}-1}{n^{4}-1}} $

Denkübung: Eine Taschenlampe leuchtet auf eine Fläche im Brooster Winkel $\Theta_{B}$, dann ist die reflektierte Welle TE-polarisiert. Sollte mittlerweile verständlich sein.

\[ \Gamma(\Theta_{1}=0°)&=\frac{\sqrt{\epsilon_{2}-\epsilon_{1}} }{\sqrt{\epsilon_{2}+\epsilon_{1}}}=\frac{n-1}{n+1}\]
\[ \Gamma_{TE}=-\Gamma_{TM}=\frac{1-n}{1+n} \]
Siehe Diagram im Skriptum, die Nullstelle ist der Brooster Winkel. Die Abbildungen wurden ausführlich besprochen.
Alles links vom Grenzwinkel der totalen Reflektion $\Theta_{1,T}$ findet man auch in der anderen Abbildung, alles rechts davon widerum nicht.

In der nächsten Abbildung sieht man, wie das einfallende und das reflektierte Feld bei einer totalen Reflektion  lokal miteinander interferieren. Es sollte nichts transmittieren, es entsteht jedoch ein evalenszentes Feld in -x-Richtung. Der Energietransport für den Feldaufbau des evalescenten Feldes finden im Einschwungvorgang auf.

Er demonstriert an den Beispielen Reflektion am Auge und Reflektion am Mikrowellen-Gitter, wie nach man rangehen muss, um das evaleszente Licht zu spüren/messen.

Weitere Übung: Dämpfung durch Mikrowellengitter: $D=\frac{P_{legal}}{P_{N}}=\frac{2W}{1000W}=\frac{1}{500}\implies -27dB$

\subsection*{4.1.2 Nichtideale Dielektrika}
Kann wie gehabt berechnet werden. Im Realteil der Wellenzahl findet man die Dämpfung und im Imaginärteil das effektive Lambda.
Siehe Quasi-Brewsterwinkel im Skriptum.

\subsection*{4.2}
\subsubsection*{4.2.1 TM-Fall}
Kann man durchrechnen, ist aber sehr einfach.

Siehe Ergebnis in Abbildung. Diesmal tritt kein evaleszentes Feld im unteren Halbraum auf. Im oberen Halbraum findet sich wieder die Superposition aus einfallender und reflektierter Welle vor.
Weiters sieht man, dass eine Metallplatte oberhalb der ursprünglichen Metallplatte die Randbedingung nicht verletzt. Siehe Abbildung mit rot markierter Platte.

Weiters sieht man, dass eine Metallplatte oberhalb der ursprünglichen Metallplatte die Randbedingung nicht verletzt. Siehe Abbildung mit rot markierter Platte.

Im praktischen Fall, werden die parallelen Platten vorgegeben im Eingang muss wird die Welle mit einem Winkel eingeprägt, der die Randbedingungen erfüllt.

\subsection*{4.3 Parallelplattenleitung}
\subsection*{4.3.2 Die TEM-Welle und die TE-Welle}
Die Feldstärke breitet sich in der horizontalen Ebene aus.

Durch mathematische Herleitung im Skriptum folgt:
\[ \lambda_{H,m}=\frac{\lambda_{0}}{\sin(\Theta_{m})} \]

\section*{09.11.2023}
\section*{Ding des Tages}
Der Richtkoppler - zwei parallelplattenleiter umgeben von einem gemeinsamen Gehäuse nähern sich im Zentrum möglichst weit an. Dadurch überlagern sich ihre H-Felder, wodurch eine Kopplung entsteht. Die Die Leitung haben zwar 90° Winkel, welche mit einer Phase versehen sind. Diese Phase ist eine freundliche Erinnerung an die Welle, dass sie abbiegen soll.
Mit solch einem Richtkoppler, kann also in der Hochfrequenztechnik eine geringe ausgekoppelte Vorwärtsleistung des anderen Leiters gemessen werden. Weiters kann am eingehenden Ende auch die reflektierte Leistung messen. Der Richtkoppler kommt wieder beim Hohlleiter.
Siehe Tafelbild und Foto vom Ding des Tages.

Siehe weiters Tafelbild von sexy Verstärkerschaltung. Der linke Transistor ist thermisch an den Wiederstand gekoppelt.

\subsection*{3.3 Der Begriff des Modus}
\subsection*{3.3.4 Die Grenzfrequenz}
\[ d=m \frac{\lambda_{0}}{2} \]
Das kleinste $m$ ergibt sich aus dem Durchmesser des Hohlleiters: $g_{G,m}=\frac{mc}{2d}$ 

Siehe Tafelbild:
Diagram der Wellenzahl abhängig von der Frequenz. Eine Gerade - unterhalb der Gerade ist Informationsaustausch möglich, oberhalb der Garde ist kein Informationstausch notwendig. Die Steigung ist falsch ausgerechnet kann aber easy hergeleitet werden.
links von der ersten Hyperpel (< m=1, also unterm ersten Modus) befindet sich auschliesslich die TEM-Welle und ist somit die schnellste (hat immer die höchste Gruppengeschwindigkeit)
Diagram der Gruppengeschwindigkeit abhängig von der Frequenz. Unterhalb der konstante ist ein Informationsaustausch möglich, oberhalb jedoch nicht!
Es gibt unendlich viele evaleszente Moden, wenig Energie aber Kleinvieh macht auch Mist.

WICHTIG: Kommenden Dienstag kommt eine Übung, Beispiel 7 Parallelplattenleiter wird besprochen.

Die Begriffe wie Wanderwelle, evaeszentes Feld, evaleszente Moden, etc. sind wenig überraschend wichtig und werden bei der Prüfung erwartet.

\section*{16.11.2023}
\section*{Geschichte des Tages}
Nobelpreisverleih an Marconis für eine transatalantische Übertragung. Die Übertragung wird zwar angezweifelt, der Verleih wurde jedoch nicht widerufen.

Fig. 2: Marconis Empfänger 1901

Die moderne Erklärung läuft auf eine Oberflächenwelle über dem schlecht leitfähigem Atlantis durch die feuchte Luft als Dielektrikum.

\section*{Der Oberflächenwelle}
Wir leiten Schritt für Schrit folgenden Ansatz her:
\[ E_{z,1}=e^{jk_{x,1}x} e^{-jk_{z}z} \]
\[ E_{z,2}=e^{jk_{x,2}x} e^{-jk_{z}z}\]
\[ H_{z,1}=H_{z,2}=0 \text{\ldots bedeutet TM-Welle} \]
\[ \frac{\partial}{\partial y}=0\]
\[ k_{x,i}^{2}+k_{y}^{2}+k_{z}^{2}=\omega^2\mu_{i}\epsilon_{i}\text{,} i \in {1,2} \]
Modale Lösung: siehe Skriptum
Randbedingungen: siehe Skriptum

Aus dem Gleichungssystem erhalten wir eine Gleichung mit den beiden Amplituden $A_{1}$ und $A_{2}$
 \[ 
 \begin{pmatrix}
 \frac{\omega\delta_{1}}{k_{x,1}} && -\frac{\omega\delta_{2}}{k_{x,2}}\\
 1 && -1
 \end{pmatrix} 
\begin{pmatrix} A_{1}\\ A_{2}\end{pmatrix}
= \begin{pmatrix} 0\\ 0\end{pmatrix}
\]
Wir fordern von der Matrix $det M \overset{!}{=}0$
Nach some Algebra erhalten wir:
\[ k_{x1}^{2}= \omega^{2}\delta_{1}^{2} \frac{\delta_{1}\mu_{1}-\delta_{2}\mu_{2}}{\delta_{1}^{2}-\delta_{2}^{2}}\]
\[ k_{x2}^{2}= \omega^{2}\delta_{2}^{2} \frac{\delta_{1}\mu_{1}-\delta_{2}\mu_{2}}{\delta_{1}^{2}-\delta_{2}^{2}}\]
\[ k_{z}^{2}= \omega^{2}\delta_{1}^{2}\delta_{2}^{2} \frac{\delta_{1}\mu_{1}-\delta_{2}\mu_{2}}{\delta_{1}^{2}-\delta_{2}^{2}}\]

Für das Besipiel der transatlantischen Übertragung:
\begin{itemize}
    \item Medium 1 ist Quasileiter: $ s_{1}=\frac{\sigma_{1}}{\omega\epsilon_{1}}\gg1$
    \item Medium 2 ist Quasidielektrikum: $ s_{2}=\frac{\sigma_{2}}{\omega\epsilon_{2}}\ll1$
\end{itemize}

\subsection*{5.2 Feldbilder}
Achtung: das Feld ist planar. Es geht nicht in die y-Richtung.

\subsection*{5.3 Wellenwiderstände und Leistungsumsatz}

\section*{Nächste Woche: 5.3.2 Der Leitungswellenwiderstand}

\section*{21.11.2023}
\section*{Ding des Tages}
\begin{itemize}
    \item Hohlleiter (rechteckiger Querschnitt)
        Kann beliebige Frequenzen ausser Gleichstrom leiten. Für gleichstrom gibt es keinen Rückleiter, für alle anderen Frequenzen kann man einen beliebigen Schnitt ziehen und die beiden Teile beliebig als Hin- und Rückleiter definieren.\item Hohlleiter (rechteckiger Querschnitt)
        Kann beliebige Frequenzen ausser Gleichstrom leiten. Für gleichstrom gibt es keinen Rückleiter, für alle anderen Frequenzen kann man einen beliebigen Schnitt ziehen und die beiden Teile beliebig als Hin- und Rückleiter definieren. Je nach Schnitt ist die Definition von Strom und Spannung anzupassen. Für schräge Schnitte gilt auch nicht mehr die Wegunabhängigkeit für $\intE ds$.
        Skizze:
        |---------|
        | a |
        |   |b $f_{g}=\frac{c_{0}}{2a}$
        |---------|
\end{itemize}

\section*{5.3.2 Der Leitungswellenwiderstand}
Wir erhalten den Ansatz
\[ dP = \frac{1}{2} \|I_{z}\|^{2}dZ\text{\ldots P = Scheinleistung!P = Scheinleistung!}\]
mit dem Integral
\[ I_{z}=\int_{\sum} \vec{S}_{1} d\vec{F}=\sigma_{1} \int_{x=0}^{\infty} \int_{y=0}^{b} E_{z1}dxdy  
=j \frac{\sigma_{1}A_{1}b}{k_{x1}}e^{-jk_{z}z}\]
Man bedenke, dass für $x=0$ der Grenzwert verwendet werden muss.
Der Leitungswellenwiderstand charakterisiert den Leistungsumsatz. Zu sehen aus $ dU_{z}=irgendwas \text{ siehe skriotum}$

Durch Einsetzen erhält man das erste Zwischenergebnis:
\[ dP=\frac{1}{2}\ldots \text{siehe Skriptum} \]

Weiters kann der Poyntingvektor herangezogen werden.
\[ \vec{T}=\frac{1}{2}\vec{E}x \vec{H}^{*}=\frac{1}{2}\begin{pmatrix} -E_{z}H_{y}^{*}\\ 0\\ E_{z}H_{y}^{*}\end{pmatrix} \]
Durch Einsetzen des Ansatzes erhalten wir das zweite Zwischenergebnis.

Durch Gleichsetzen der Zwischenergebnisse erhalten wir: $dZ_{L}\approx \eta_{1} \frac{dz}{b}$
\[ \eta_{1}\text{\ldots Mediumswellenwiderstand} \]
\[ \frac{dz}{b} \text{\ldots Geometrie} \]

\section*{5.3.3 Berechnung der Leitungsverluste: Power Loss Method}
Über die Sprungbedingung der tangentialen Feldstärke auf der Leiteroberfläche kann ein Ausdruck für die Oberflächenstromdichte hergeleitet werden.
\[ \implies I_{z}=-b H_{y1}(Oberfläche) \]
\[ p=\frac{1}{b} \frac{dP_{W}}{dz}=\frac{1}{2}\|H_{tang}(Oberfläche)\|^{2}b\R_{1} \]
\[ \R_{1}\text{\ldots Realteil des zugehörigen Leitungswellenwiderstandes} \]


\section*{5.3.4 Der Oberflächenwiderstand}
\[ R=\frac{l}{\sigma_{1}A}=\frac{l}{\sigma_{1}d_{1}b}=R_{square}\frac{l}{b} \]

\section*{5.3.5 Der Skin-Effekt}
$R=\frac{1}{2\pi a}\sqrt{\frac{\omega\mu}{2\sigma_1}} $
Wenn wir das nun plotten, können wir das erste Grundgesetz der Nachrichtentechnik überprüfen: Alle Verluste steigen mit der Frequenz!
\subsubsection*{Einschub Exponentialfunktion}
Die Fläche unter $\int_{0}^{\infty}   a\cdot e^{-bx} dx$ kann durch Konstruktion einer Gerade durch die Achsen bestimmt werden.

\section*{23.11.2023}
\section*{Ding des Tages}
\begin{itemize}
    \item Hohlleiter: Merklich dünn, ausgelegt für das D-Band; Mechanischer Aufbau: Klemmverbindung mittels trichterförmigem Flansch
    \item Quadratischer Trichter
    \item SIW (Substrate Integrated Wareguide): Durch eine Serie an Vias wird aus einem Bereich einer doppelteitig beschichteten Platine wird ein Hohlleiter geformt. Die Lücke zwischen den Vias darf maximal ein vierte der Hohlleiterwellenlänge betragen
\end{itemize}

\section*{6.1 Der Rechteckhohlleiter}
die transversalen Feldkomponenten
$\nabla^{2} \begin{pmatrix} E_{x}\\ E_{y}\\\ H_{x}\\ H_{y}\end{pmatrix}
+(\omega^{2}\epsilon\mu-k_{z}^{2})
\begin{pmatrix} E_{x}\\ E_{y}\\\ H_{x}\\ H_{y}\end{pmatrix}=0$
mit $k_{z}=\frac{\omega}{c}=\omega\sqrt{\epsilon\mu} $ für TEM-Wellen
\implies Die TEM-Welle ist nicht ausbreitungsfähig im Hohlleiter

Wegen den Randbedingungen für die H-Welle wird der ein $\cos$-Ansatz für die H-Welle gemacht.
Wegen den Randbedingungen für die E-Welle wird der ein $\sin$-Ansatz für die E-Welle gemacht.

Die Randbedingungen ergeben $k_{z}=\frac{m\pi}{a}$ und $k_{y}=\frac{n\pi}{b}$ mit Breite $a$ und Höhe $b$ Wellen, die entweder von ihren Wellenbäuchen oder Wellenknoten angrenzen.

Wir leiten also einen Ausdruck für die Grenzwellenlänge $\lambda_{G}$ her. Dadurch kann die Hohlleiterwellenlänge $\lambda_{H}=\frac{1}{\sqrt{1-(\frac{\lambda}{\lambda_{G}})^{2}} }$ berechnet werden. Diese Formel für die Hohlleiterwellenlänge ist gültig für sehr viele Hohlleiterformen.

Wichtige Formeln:
\begin{itemize}
    \item Grenzwellenlänge: $\lambda_{G}$
    \item Gruppengeschwindigkeit: $yo$
    \item Phasengeschwindigkeit:  $yo$
\end{itemize}
Wichtig: Die Gruppengeschwindigkeit ist immer kleiner als die Phasengeschwindigkeit. (vielleicht hab ichs vertauscht\ldots )

\section*{27.11.2023}
\section*{Ding des Tages}
\begin{itemize}
\item David Eurocom
\item Hohlleiter
    Die Stummelantenne ragt in das Zentrum des hornförmigen Trichters (Megafon, siehe Tafelbild). Die von der Antenne ausgehende Welle breitet sich in alle Richtungen aus und wird auch an der Rückwand des Trichters reflektiert. Die Reflektion erzeugt einen Phasensprung
    \[ \Delta\phi=lk_{z} +\pi+2lk_{z}=\pi+2l \frac{2\pi}{\lambda_{H}}=\pi+\frac{4\pi}{\lambda_{H}}\overset{!}{=}2\pi \].
    Zur Einstellung kommen solche Antennen Werkstjustiert oder mit einer verstellbaren Rückwand (+100€).
    
    Man kann die Antenne zum Beispiel am Gehäuse kurzschliessen\implies Stromfluss regt ein H-Feld an\newline
    Weiters kann man die Bandbreite mit einer Stufe (oder Schräge) erhöhen.
\item Parallele Rechteckhohlleiter:
    Die beiden Hohlleiter werden von einer Wand getrennt. Wenn nun Löcher an dieser Wand angebracht werden, können sich die Wellen in den Nebenraum ausbreiten. Beim Durchgang durch eines der Löcher erzeugt einen uns unbekannten Phasensprung, die Wegstrecke ist jedoch unabhängig von der Position des Lochs, daher sind alle durch die Löcher gewanderten Teilwellen phasengleich. Wir können nun die Ausgänge links und rechts abhängig von der Eingangswelle vergleichen:
    Eingangswelle von links: wenn Lochabstand $d=\frac{\lambda_{H}}{4}$ \implies rechter Ausgang konstruktiv und linker Ausgang destruktiv
    Eingangswelle von rechts: wenn Lochabstand $d=\frac{\lambda_{H}}{4}$ \implies rechter Ausgang destruktiv und linker Ausgang konstruktiv 

    Wir erkennen also, dass wir hier einen Richtkoppler gegeben haben.

\item Zirkulator: Leitet Wellen einmal im Kreis um (siehe Tafelbild: 2->3->1->2, von 2 kommt aber hoffentlich nicht im Beispiel des Handys)
\item Wandler (siehe Tafelbild)
\item ein paar Sachen mit noch mehr Löchern, hat er zum selbst nachdenken durchgereicht
\end{itemize}

\subsection*{TM und TE Ausbreitung}
Modenkarte wie im Skriptum:
Die gesamte Karte enthält nur Hyperbeln, die Geraden sind hier einfach nur degenerierte Hyperbeln.
Die rote Gerade bestimmt sowohl die Reihenfolge alsauch Positionen der Durchstosspunkt.
Je Flacher die Gerade um so besser, weil wir dann noch länger im Monomodebetrieb bleiben.
\end{document}
