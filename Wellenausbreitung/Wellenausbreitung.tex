\documentclass[a4paper]{article}

\usepackage[a4paper,top=2cm,bottom=2cm,left=3cm,right=3cm,marginparwidth=1.75cm]{geometry}
\usepackage[utf8]{inputenc}
\usepackage[T1]{fontenc}
\usepackage{textcomp}
\usepackage[ngerman]{babel}
\usepackage{amsmath, amssymb}
\usepackage{accents}

% figure support
\usepackage{import}
\usepackage{xifthen}
\pdfminorversion=7
\usepackage{pdfpages}
\usepackage{transparent}
\newcommand{\incfig}[1]{%
    \def\svgwidth{\columnwidth}
    \import{./figures/}{#1.pdf_tex}
}

\pdfsuppresswarningpagegroup=1

\begin{document}
\section*{2.1 Ladungskontinuität}
Maxwell ergänzt in der letzten Maxwellschen Gleichung die Summe der Leitungsströme mit der Änderung der Verschiebungsstromdichte.

\[
\nabla \cdot (\nabla \times H) = \nabla \cdot S + \nabla\cdot\frac{dD}{dt}
= \nabla\cdot S + \frac{d\nabla\cdot D}{dt}
= \nabla\cdot S + \frac{d\rho}{dt}
\]

\section*{2.2.1 Hilfsgleichung}
Problem: 8 Maxwellgleichungen für 12 unbekannte, daher werden Hilfgleichungen benötigt.\newline
el. und mag. FLussdichtegleichungen gelten in isotropen Medien (linear \& homogen?)


\subsection*{$\rho$ (ausgeprochen Rohacel)}
\[\nabla\cdot(\nabla x H)=\nabla\cdot (S+\frac{dD}{dt})\]
\[\nabla xE = -\frac{dB}{dt}\]
\[\nabla \cdot D=\rho\]

(effektive Ladungsfreiheit bei $f\le 10^18 Hz$ - bei diesen Wellenlängen ist $\lambda$ kleiner als der Atomradius, da hören die Kontinuitätsannahmen auf)\newline
$\nabla\cdot B=0$\newline
aus 1 und 3:

\[0 = \nabla\cdot S+\frac{d\rho}{dt} \text{folgt} \sigma \nabla\cdot E+\frac{drho}{dt}=0\]
\[\sigma \rho\epsilon+\frac{d\rho}{dt}=0\]
\[\rho (x,y,z,t) \text{with} (x,y,z) \in \mathbb{R}^{3}, t\in 0, inf \]
\[\rho(x,y,z,t)=\rho(x,y,z)\cdot e^{-\frac{t}{\tau_{D}}}\]
\[S = \sigma E\]
\[D=\epsilon E\]
\[\frac{\sigma}{\epsilon} \rho_{0}(x,y,z)\cdot e^{\frac{-t}{\tau_{D}}} + \rho_{0}(x,y,z)(\frac{-1}{\tau_{D}})e^{\frac{-t}{\tau_{D}}}=0\]
\[\frac{\sigma}{\epsilon}= \frac{1}{\tau_{D}}, \tau_{D}=\frac{\epsilon}{\sigma}\]
allgemeine Lösung = homogen + partikulär
\[\rho(x,y,z) = \rho_{0}(x,y,z)\cdot e^(-\frac{t}{\tau_{D}}) + \text{partikuläre Lösung}\]
\[\int_{\mathbb{R}^3}(\rho(x,y,z,t))dxdydz = \int_{\mathbb{R}^2}(\rho_{0}(x,y,z)\cdot e^\frac{-t}{\tau_{D}})dxdydz\]
\[= e^\frac{-t}{\tau_{D}}, Q(t)=e^\frac{-t}{\tau_{D}}\cdot Q_{0}\]
Anfangsladungsverteilung:
\[Q_{0}=rect_{-t_{0}}(t) - rect_{t_{0}}(t)\]
Wobei bei Kupfer $\tau_{D} = 8,854.10^-12 F/m / 60.10^6S/m~=10^-18s$ gilt. 

$H(x,y,z,t)=Re\{\underline{H}(x,y,z)e^{jwt}\}$ ...Notiz: Spitzenwertzeiger $\underline{H}$

\[\nabla \times \underline{H} = \underline{S} + j\omega \underline{D} = \sigma\cdot \underline{E} + j\omega\epsilon \underline{E} = (\sigma + j\omega\epsilon)\cdot \underline{E} = j\omega\epsilon(1+\frac{\sigma}{j\omega\epsilon})\cdot \underline{E} = j\omega\epsilon(1 - \frac{j\sigma}{\omega\epsilon})\cdot \underline{E}\]
$\nabla \times \underline{H}= j\omega \delta \underline{E}$...$\delta$ = effektive Permitivität
\[\nabla \times  \underline{H} = -j\omega\mu \underline{H}\]
\[\nabla\times\underline{H} = -j\omega \underline{B}\]
\[\nabla\cdot\underline{D} = -j\omega \underline{B}\]
\[\nabla\cdot\underline{B} = 0\]

\section*{Poyntingscher Satz}
\begin{itemize}
    \item Gleichungen mit $H$ und $E$ multiplizieren
    \item Gleichungen subtrahieren
    \item den Term $\nabla\cdot (E\times H)$ mittels Produktregel + Spatprodukt umformen auf $H\cdot (\nabla\times E) - E\cdot (\nabla\times H)$
    \item Weiter Produktregel an $E$ und $H$ Termen anwenden
\end{itemize}

\[x\cdot \frac{dx}{dt}= |x|^2/2\]
\[\nabla\cdot (E\times H) + \sigma |E|^2 = -\frac{d}{dt}(\frac{\epsilon}{2} |E|^2 + \frac{\mu}{2}|H|^2)\]
Abstrahlung + Heizung = elektromagnetische\newline
Die dielektrischen Verluste wachsen direkt proportional zur Frequenz.

\[E(x,y,z,t)=Re\{E(x,y,z,t)\cdot e^{j\omega t}\}\]
\[H(x,y,z,t)=\frac{1}{2}(\underline{H}(x,y,z,t)\cdot e^{j\omega t} + \underline{H}\ast(x,y,z)\cdot e^{-j\omega t})\]
\[E(x,y,z,t)\times H(x,y,z,t)=Re\{\underline{E}(x,y,z)\cdot e^{j\omega t}\} \times \frac{1}{2}(\underline{H}\cdot e^{j\omega t}+\underline{H}\ast\cdot e^{-j\omega t})\]
\[=Re\{\frac{1}{2}\underline{E}\times \underline{H}\ast + \frac{1}{2}\underline{E}\times \underline{H}\cdot e^{2j\omega t}\} = \text{stationärer + pendelnder Anteil}\]
\[=Re\{\frac{1}{2}\underline{E}\times \underline{H}\ast\} + \ldots\]

komplexer Poyntingvektor: \[\underline{T}=\frac{1}{2}\underline{E}\times \underline{H}\ast= \underline{T}_{\omega}+j \underline{T}_{b}\]

\section*{Tägliche Trivia}
Quellen und Wellen hängen irgendwie zusammen, Antennen sind gerne mal Quellen.
Der Begriff der Antenne kommt aus dem italienischen Wort für Zeltstange.
Der Dipol ist die elementare Antenne.\newline
Merksatz: Die niedrigste Strahlungsordnung ist die Dipolladung.
Eine Druckwelle einer Explosion ist eine Monopolstrahlung, die Erdbebenwelle ist
eine Dipol- oder sogar Quadropolstrahlung.
Die Kräfte einer Explosion gehen radial weg, während im Zentrum eines Erbebens
Platten mit entgegengesetzten Kräften aneinander reiben.\newline

Dinge des Tages:
\begin{itemize}
    \item Gerade Einzeldrahtantenne mit induktiver Kopplung in der Mitte. Die Kopplung teilt die Antenne in zwei Teile für kurz- und langwellige Signale.\newline
       Skizze: ------a------===------b------
                $\lambda_{a}~=70cm | \lambda_{b}~=100cm$
    \item Koaxialkabel
    \item Wifi-Antenne: Diese haben "gespiegelte" Stecker, damit normale Koaxialantennen nicht in einen Router passen. Mehr dazu in Kapitel 11
\end{itemize}

Also, was war nochmal eine Antenne? Wandler zwischen raumgebundenen und leitungsgebundenen Wellen.

\section*{2.4 Randbedingungen}
Wiederholung: Faraday'sches Gesetz

\section*{2.1 Grenzflächen zwischen Dielektrika}
Beispiel: Oberflächenintegral über kleines Rechteck an Grenzfläche.
$\Delta l$ ist größer als $\Delta x$, aber kurz genug, damit $E_{tangential}$ an den beiden kurzen Kanten konstant bleibt. Siehe Resultat im Skriptum.
Analoges kann für die magnetische Feldstärke durchgeführt werden.\newline

Ähnliche infinitesimale Methode für die Divergenzgleichungen.
Die elektrischen Flächenladungen die in den Ergebnissen auftreten, spielen in
dieser Vorlesung keine Rolle und daher gibt es auch keine Beispiele dazu.
Die Flächenstromdichte spielt sehr wohl eine Rolle (Skineffekt).\newline\newline
Sommerfeld'sche Ausstrahlungsbedingung: Wellen verschwinden in der Unendlichkeit\newline
(Mathematische Formulierung wird erstmal nicht durchgeführt)

\section*{2.5 Lösung der Wellengleichung in kartesischen Koordinaten}
\bold{Sei angemerkt:} unter hinreichenden Bedingungen gilt der Satz von Schwartz.
\[\nabla \times H = \sigma E +\epsilon \frac{\partial}{\partial t}  E\]
\[\nabla x (\nabla x H) = sigma(\nablaxE)+epsilon partd/partdt(\nabla x E)\]
\[\nabla(\nabla H)-\delta H = \sigma-\mu \frac{\partial H}{\partial t} + \epsilon \frac{\partial }{\partial t}(-\mu \frac{\partial H}{\partial t})\]
\[\delta H = \mu \sigma \frac{\partial H}{\partial t} + \epsilon\mu \frac{\partial ^2 H}{\partial t^2}\] \ldotsTelegraphengleichung für $H$,  $B$, $E$ \& $D$\newline
In dieser VO wird hauptsächlich auf den Separationsansatz eingegangen.
end
$H=Re\{\underline{H}(x,y,z)\cdot e^{j\omega t}\}$ \ldots $\frac{\partial }{\partial t}$ wird Laplace-transformiert\newline
$\delta \underline{H} = j\omega\mu\sigma\underline{H}\epsilon\mu\omega^2 \underline{H}\ldots \delta \underline{H}+(\omega^2 \mu\epsilon j\omega\mu\sigma)\underline{H} = \underline{0}$? \ldots Helmholtzgleichung\newline

HÜ: Empfehlung: Das gleiche für E wiederholen.
Nächste Woche Di: Seperationsansatz (sollte man aus Mathe noch kennen)

\subsection*{17.10.2023}
\section*{2.5.1 Lösungsansätze}
\begin{itemize}
    \item Methode 1: $E=a\cdot \psi$\newline
    \item Methode 2:\newline
    Reines Wirbelfeld E bzw H=\nabla x (a.psi)=-a x \nabla.psi
    Durch Wiederholung: \nabla x ( \nabla x (a.psi))
    Mit genügend Wiederholungen kann die Elektrodynamik gelöst werden.
    \item Methode 3: Ansatz mittels elektrodynamischer Potentiale
    Ansatz mittels Vektorpotential
\end{itemize}

\section*{2.5.2 Separationsansatz}
Zu lösen:
\[\frac{\partial^2 \psi}{\partial x^2} + \frac{\partial ^2\psi}{\partial y^2} +\frac{\partial ^2\psi}{\partial z^2}  + \omega^2\mu\sigma\psi=0\]
\[\psi(x,y,z)=X(x)\cdot Y(y)\cdot Z(z)\]
\[\frac{\partial ^2\psi}{\partial x^2}\cdot YZ + X\cdot \frac{\partial ^2\psi}{\partial y^2}\cdot Z + XY\cdot \frac{\partial ^2\psi}{\partial z^2} + \omega^2 \mu\sigma\psi\cdot XYZ=0\]
\[\frac{1}{X(x)} \cdot \frac{\partial ^2\psi}{\partial x^2} + \frac{1}{Y(y)}\cdot \frac{\partial ^2\psi}{\partial y^2} + \frac{1}{Z(z)}\cdot \frac{\partial ^2\psi}{\partial z^2} + \omega^2\mu\sigma=0\]\newline
Aufgrund des konstanten Terms folgt:
\[k\cdot x^2 + k\cdot y^2 + k\cdot z^2 = \omega^2\mu\sigma\]\ldots Seperationsansatz

\[\frac{1}{X}\cdot \frac{\partial ^2X(x)}{\partial x^2} = -k\cdot x^2\]\newline
folglich erhält man die Schwingungsgleichung. Analog können die Schwingungsgleichungen für $y$ und $z$ bestimmen und man erhält das Gleichungssystem
\[X''(x)+k\cdot x^2\cdot X(x) = 0\]
\[Y''(x)+k\cdot y^2\cdot Y(y) = 0\]
\[Z''(x)+k\cdot z^2\cdot Z(z) = 0\]

Zur Bestimmung einer Lösungen können unterschiedliche Formen angesetzt werden. Die nützlichsten sind:
\begin{itemize}
    \item \bold{Fundamentalsystem:} $\{\sin(k_{x}x), \cos(k_{x}x)\}$\ldots häufig verwendet
    \item \bold{Komplexes Fundsystem:} $\{e^{-jk_{x}x}, e^{jk_{x}x}\}$\ldotshäufig verwendet
\item \bold{Mischform:} $\{\sin(k_{x}x)m e^{jk_{x}x}\}$, etc.
    \item \bold{Hyperbelfunktion:} $\{\cosh(k_{x}x), \sinh(k_{x}x)\}$
\end{itemize}

Physikalische Bedeutungen in Tabelle 2.1 können durch einsetzen in unsere Wellenfunktion $Re\{\psi(x,y,z)\cdot e^{j\omega t}\}=Re\{\sin(k_{z}z)\cdot e^{j\omega t}\}$ überprüft werden. Die Wellenfront verläuft in z-Richtung.\newline
Fortan wird die verkürzte Schreibweise $\frac{\partial}{\partial x}  = \partial _{x}$ verwendet.\newline

\[(2.39) \partial_{y} H_{z} - \partial_{z} H_{y}= \frac{j\omega\sigma E_{x}}{j\omega\mu}\]
\[(2.43) \partial_{z}E_{x} - \partial_{x}E_{z} = -j\omega\muH_{y} \partial_{z}\]
Über 2 Schritte folgt: $j\omega\mu\cdot \partial_{y}H_{z}+\partial_{z}^2 E_{x}-\partial_{z}\partial_{x}E_{z} = -\omega^2\mu\sigma\cdot E_{x}$\newline
Annahme Wanderwelle nach $z=\infty$\ldots $j\omega\mu \partial_{y}H_{z} - k_{z}^2\cdot E_{x} + jk_{z}\cdot \partial_{x}E_{z}=-\omega^2\mu\sigma E_{y}$
\[(\omega^2\mu\sigma-k_{z}^2)E_{x} = -j\omega\mu \partial_{y}\cdot H_{z}-jk_{z} \partial_{y}E_{z}\]
\[E_{x}=-j\frac{1}{\kappa^2}\cdot (k_{z}.\cdot \partial_{x}E_{z}+\omega\mu\partial_{y}H_{z})\]\ldots sollte die erste Gleichung (2.45 ergeben)
HÜ: Für 2.46 bis 2.48 wiederholen

\[\kappa^2=0\]
\[\omega^2\musigma-k_{z}^2=0\]
\[k_{z}=\pm \frac{\omega}{\sqrt(\mu\epsilon)}\]
\[\kappa^2\neq 0\]
\[\omega^2\mu\sigma-k_{z}^2=k_{x}^2+ k_{y}^2\]\ldots Der Betrag der k-Vektors bleibt gleich, daher wird die $k_{z}$ Komponente wegen $k_{x}$ und $k_{y}$ kleiner
Aus dieser Erkenntnis zeigt sich: Die TEM-Welle ist die schnellste Welle.
Notiz: Eingerahmte Formeln müssen auswendig gelernt werden, außer die Modalen Lösungen in kartesischen Koordinaten.\newline
Stattdessen das $\kappa^2$ auswendig lernen!

\section*{3. Die homogene ebene Welle}
\section*{3.1 Die HEW im idealen Dielektrikum}
$e$\ldots elektrische Feldstärke\newline
$h$\ldots magnetische Feldstärke\newline
Kleinbuchstaben zur Andeutung der Zeitabhängigkeit.

\[\partial_{z}^2 e_{x}(z,t) - \mu\epsilon\partial_{z}^2 e_{x}(z,t) = 0\]
mit $e_x(z,t)=f_{1}(z-vt)+f_{2}(z+vt)$ eingesetzt folgt (nur mal für f1):
\[f_{1}''(z-vt)-\mu\epsilon f_{1}''(z-vt)(-v)^2=0\]
\[1-\mu\epsilon v^2 = 0 \implies v=\pm\frac{1}{sqrt(\mu\epsilon)}\]

\[h_{y}^+=\frac{1}{\nu}\cdot e_x^+\ldots \nu=\sqrt(\frac{\mu}{\epsilon}), \nu_{0}=\sqrt(\frac{µ_{0}}{\epsilon_{0}})=\sqrt((4\cdot \pi\cdot 10^{-z}\frac{H}{m})^2\cdot c_{0}^2)=12\cdot \pi\cdot 10 \Omega = 120\cdot \pi\Omega\] \ldots Feldwellenwiderstand (hat nichts mit ohmschen Verlusten zu tun)
\[\epsilon_{0}\cdot µ_{0}=\frac{1}{c_{0}^2} = \frac{1}{\mu_{0}\cdot c_{0}^2}\]

\section*{19.10.2023}
\section*{Ding des Tages}
Der Prof. baute sich einmal eine Antenne, um mit der ISS beim FAQ mit einer Schule zuzuhören. Eine 50cm Stange nach links, eine 50cm Stange nach rechts. Die Stangen gehen per Bananenstecker in eine mysteriöse graue Box. Von unten wird ein Koax-Kabel angesteckt, bissl qualitativer, nicht so wie das vom grindigen Fernsehen.\newline
Es wird ein Balun benötigt (Bal=balanced, un=unbalanced). Dieses schließt eine balanzierte Leitung an eine unbalanzierte Leitung an.\newline
Abb.: Tafelbild + Diashow
Weil die Ströme gegensinnig, die Wicklung aber gleichsinnig ist, heben sich die Flüsse auf \implies $L=0, \omega_0=0$
Da bereits eine geringe Stromdifferenz eine relativ hohe Impedanz erzeugt, bestraft dieser Balun Differenzströme.\newline
Weil die Schule unterhalb des Horizontes lag, konnte der Prof. nur der ISS, nicht aber der Schule zuhören.
Der Prof. hat auch die Aufnahme zur Verfügung gestellt.

\subsection*{3.1.3 Energiedichte der HEW, Poyntingscher Vektor}
Zu jedem Zeitpunkt, an jedem Ort, ist die elektrische Energiedichte gleich groß wie die magnetische Energiedichte: sie gehen synchron.\newline
Abb. 3.1 Man sieht die Energiedichte eines Photons. Dort wo die Pfeile näher zusammen liegen, ist die Feldstärke erhöht. Man kann einen sinusförmigen Verlauf der Feldstärke erkennen.\newline
Weiters ist die Homogene Elektromagnetische Welle im freien Raum dargestellt. Unbedingt einprägen!\newline
\subsection*{3.1.4 Wellenzahl und Wellenlänge}
Unterschied zwischen $k$ und $\omega$.
\section*{Polarisation}
Alle haben eine Polaristation. Unpolarisierte sind nur im mittel nicht polarisiert.
\subsection*{Polaristationsarten}
\subsubsection*{elliptische Polarisations (allgemeine Form)}



\end{document}
