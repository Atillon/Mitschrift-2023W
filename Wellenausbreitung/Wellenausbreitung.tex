\documentclass[a4paper]{article}

\usepackage[utf8]{inputenc}
\usepackage[T1]{fontenc}
\usepackage{textcomp}
\usepackage[dutch]{babel}
\usepackage{amsmath, amssymb}
\usepackage{accents}


% figure support
\usepackage{import}
\usepackage{xifthen}
\pdfminorversion=7
\usepackage{pdfpages}
\usepackage{transparent}
\newcommand{\incfig}[1]{%
    \def\svgwidth{\columnwidth}
    \import{./figures/}{#1.pdf_tex}
}

\pdfsuppresswarningpagegroup=1

\begin{document}
\section{2.1 Ladungskontinuität}
Maxwell ergänzt in der letzten Maxwellschen Gleichung die Summe der Leitungsströme mit der Änderung der Verschiebungsstromdichte.

\[
\nabla \cdot (\nabla \times H) = \nabla \cdot S + \nabla\cdot\frac{dD}{dt}
= \nabla\cdot S + \frac{d\nabla\cdot D}{dt}
= \nabla\cdot S + \frac{drho}{dt}
\]

\section{2.2.1 Hilfsgleichung}
Problem: 8 Maxwellgleichungen für 12 unbekannte, daher werden Hilfgleichungen benötigt

el. und mag. FLussdichtegleichungen gelten in isotropen Medien (linear & homogen?)

Rohacel
\nabla\cdot(\nabla x H)=\nabla.(S+dD/dt)
\nabla xE = -\frac{dB}{dt}
$\nabla . D=rho$ (effektive Ladungsfreiheit bei fkleinergleich 10^18 Hz, 
bei diesen Wellenl\"angen ist lambda kleiner als der Atomradius, da hören die Kontinuutaetsannahmen auf)
$\nabla.B=0$
aus 1 und 3:
0 = \nabla.S+drho/dt folgt sigma \nabla.E+drho/dt=0
                           sigma rho/epsilon+drho/dt=0
                                rho(x,y,z,t), (x,y,z)element aus R^3, t element[0, inf)
                                rho(x,y,z,t)=rho(x,y,z).e^(-t/tau_D)
S = sigmaE
D=epsilonE

sigma/epsilon rho_0(x,y,z)e^(-t/tau_D) + rho_0(x,y,z)(-1/tau_d)e^(-t/tau_D)=0
sigma/epsilon = 1/tau_D, tau_D=epsilon/sigma
allgemeine Lösung = homogen + partikulär
rho(x,y,z) = rho_0(x,y,z).e^(-t/tau_D) + partikuläre Lösung
int_R^3(rho(x,y,z,t))dxdydz = int_R^2(rho_0(x,y,z)e^-t/tau_D)
                            = e^(-t/tau_D)
                            Q(t)=e^(-t/tau_D).Q_0
Anfangsladungsverteilung
Q_0=rect-t_0(t) - rectt_0(t)

tau_D von Kupfer = 8,854.10^-12 F/m / 60.10^6S/m~=10^-18s

H(x,y,z,t)=Re{botH(x,y,z)e^jwt} ...Notiz: Spitzenwertzeiger botH

\nabla x botH = botS + jwbotD = sigma.botE + jwepsilon botE = (sigma + jwepsilon).botE = jwepsilon(1+sigma/jwepsilon).botE = jwepsilon (1 - jsigma/wepsilon).botE
        \nabla x botH = jw delta botE   ...delta = effektive Permitivität
        \nabla x botH = -jwµ botH
\nabla x botH = -jwbotB
\nabla.botD = -jwbotB
\nabla.botB = 0

\section{Poyntingscher Satz}
gleichungen mit H und E multiplizieren

Gleichungen subtrahieren
Den Term \nabla.(ExH) mittels Produktregel + Spatprodukt umformen auf H(\nablaxE) - E(\nablaxH)
Weiter Produktregel an E und H term anwenden (x.dx/dt=betragx^2/2)

\nabla.(ExH) + sigma betragE^2 = -d/dt(epsilon betragE^2/2 + µbetrag(H)^2/2)
Abstrahlung   Heizung         =          elektromagnetische


Die dielektrischen Veruste wachsen direkt proportional zur Frequenz.

E(x,y,z,t)=Re{E(x,y,z,t)e^jwt}
H(x,y,z,t)=1/2(botH(x,y,z,t)e^jwt + botHkomkonj(x,y,z)e^-jwt)
E(x,y,z,t)xH(x,y,z,t)=Re{botE(x,y,z)ejwt} x 1/2(botHe^jwt+botHkomkonj.e^-jwt)
                    =Re{1/2botExbotHkomkonj + 1/2*botExbotHe^2jwt}
                    stationärer Energifluss        Pendelt
                    =Re{1/2.botExbotHkompkonj} + ....

    komplexer Poyntingvektor botT=1/2botExbotHkompkonj = botT_w+j botT_b


\section{tägliche Trivia}
Quellen und Wellen hängen irgendwie zusammen, Antennen sind gerne mal Quellen.
Der Begriff der Antenne kommt aus dem italienischen Wort für Zeltstange.
Der Dipol ist die elementare Antenne.
Merksatz: Die niedrigste Strahlungsordnung ist die Dipolladung.
Eine Druckwelle einer Explosion íśt´éíńé´Monopolstrahlung, die Erbebenwelle ist
eine Dipol- oder sogar Quadropolwelle.
Die Kräfte einer Explosion gehen radial weg, während im Zentrum eines Erbebens
Platten mit entgegengesetzten Kräften aneinander reiben.
Dinge des Tages:
-) Gerade Einzeldrahtantenne mit induktiver Kopplung in der Mitte. Die Kopplung teilt die Antenne in zwei Teile für kurz- und langwellige Signale.
       Skizze: ------------===--------
                lambda=70cm   lambda~=100cm

-) Koaxkabel
-) Wifi-Antenne: Diese haben "gespiegelte" Stecker, damit normale koaxantennen 
    nicht in einen Router passen. Mehr dazu in Kapitel 11

Also, was war nochmal eine Antenne? Wandler zwischen raumgebundenen und
                                    leitungsgebundenen Wellen.

\section{2.4 Randbedingungen}
Wiederholung: Faraday'sches Gesetz

\section{2.1 Grenzflächen zwischen Dielektrika}
Beispiel: Oberflächenintegral über kleines Rechteck na Grenzfläche.
        delta l ist größer als delta x, aber kurz genug, damit E_tangential an
        den beiden kurzen Kanten konstant bleibt.
    siehe Resultat im Skriptum
Analoges zur magnetischen Feldstärke

Ähnliche infinitesimale Methode fpr die Divergenzgleichungen.
Die elektrischen Flächenladungen die in den Ergebnissen auftreten, spielen in
dieser Vorlesung keine Rolle und daher gibt es auch keine Beispiele dazu.
Die Flächenstromdichte spielt sehr wohl eine Rolle (Skinneffekt).

Sommerfeldsche Ausstrahlungsbedingung: Wellen verschwinden in der Unendlichkeit
        (Mathematische Formulierung wird erstmal nicht durchgeführt)

\section{2.5 Lösung der Wellengleichung in kartesischen Koordinaten}
Sei angemerkt: unter hinreichenden Bedingungen  gilt der Satz von schwartz.
\nabla x H = sigma E +epsilon partd/partdt E
\nabla x (\nabla x H) = sigma(\nablaxE)+epsilon partd/partdt(\nabla x E)
\nabla(\nabla H)-delta H = sigma()-µpartdH/partdt) + epsilon partd/partdt(-µpartdH/partdt)
delta H = µsigmapartdH/partdt+epsilon µ partd^2/partdt^2H          ... Telegraphengleichung für H, B, E, D
In dieser VO wird hauptsächlich auf den Seperationsansatz eingegangen
H=Re{botH(x,y,z).e^jwt}...partd/partdt wird laplace transformiert
delta botH = jwµsigmabotHepsilonµw^2botH ... delta botH+(w^2µepsilojwµsigma)botH = bot 0? ... Helmholtzgleichung

HÜ: Empfehlung: Das gleiche für E wiederholen.
Nächste Woche Di: Seperationsansatz (sollte man aus Mathe noch kennen)

17.10.2023
\section{2.5.1 Löungsansätze}
Methode 1: E=a.psi
Methode 2:
    Reines Wirbelfeld E bzw H=\nabla x (a.psi)=-a x \nabla.psi
    Durch Wiederholung: \nabla x ( \nabla x (a.psi))
    Mit genügend Wiederholungen kann die Elektrodynamik gelöst werden.

Methode 3: Ansatz mittels elektrodynamischer Potentiale
    Ansatz mittels Vektorpotential

\section{2.5.2 Seperationsansatz}
zu lösen:
    pard^2psi/pardx^2 # pard^2psi/pardy^2+ pard^2psi/pardz^2 + w^2µsigmapsi=0
        psi(x,y,z)=X(x)Y(y)Z(z)
    pard^2psi/pardx^2YZ # .Xpard^2psi/pardy^2.Z+ XYpard^2psi/pardz^2 + w^2µsigmapsi.XYZ=0
    1/X(x) . pard^2psi/pardx^2 + 1/Y(y) . pard^2psi/pardy^2 + 1/Z(z) . pard^2psi/pardz^2 + w^2µsigma=0
                                                                                           ^^^^^^^^^
    aufgrund des konstanten Terms folgt:
    1/X(x) . pard^2psi/pardx^2 + 1/Y(y) . pard^2psi/pardy^2 + 1/Z(z) . pard^2psi/pardz^2 + w^2µsigma=0
    ^^^^^^^^^^^^^^^^^^^^^^^^^^^  ^^^^^^^^^^^^^^^^^^^^^^^^^^   ^^^^^^^^^^^^^^^^^^^^^^^^^^
        kx^2                   +     ky^2                   +       kz^2                 = w^2µsigma ...Seperationsansatz

1/X . d^2X(x)/dx^2 = -kx^2
    folglich X''(x)+kx^2.X(x) = 0 ... Schwingungsgleichung
    ANALOG   Y''(x)+ky^2.Y(y) = 0
             Z''(x)+kz^2.Z(z) = 0
    Fundamentalsystem: {sin(kx.x), cos(kx.x)} ... häufig verwendet
    Komplexes Fundsys: {e^-jkx.x, e^+jkx.x}   ... häufig verwendet
    Mischform:         {sin(kx.x), e^+jkx.x}, etc.
    Hyperbelfunktion:  {cosh(kx.x), sinh(kx.x)}
Physikalische Bedeutungen in Tabelle 2.1 können durch einsetzen in unsere Wellenfunktion
Re{Psi(x,y,z).e^jwt}=Re{sin(kz.z).e^jwt} überprüft werden. Die Wellenfront verläuft in z-Richtung.

Fortan wird die verkürzte schreibweise pard/pardx = pard_x verwendet.

(2.39) pard_yH_z - pard_zH_y = jwsigmaE_x / .jwµ
(2.43) pard_zE_x - pard_xE_z = -jwµH_y / .pard_z
    über 2 Schritte folgt: jwµ.pard_yH_z+pard_z^2E_x-pard_zpard_xE_z = -w^2µsigma.E_x

Annahme wanderwelle nach z +inf     ..... jwµpard_yHz - k_z^2.E_x + jk_z.pard_xE_z=-w^2µsigmaE_y
                                          (w^2µsigma-kz^2)E_x = -jwµpard_y.Hz-jk_zpard_xE_z
                                          ^^^^^^^^^^^^^^^^
                                             kappa^2
                                          E_x=-j/kappa^2.(k_z.pard_xE_z+wµd_yH_z) ... sollte die erste Gleichung (2.45 ergeben)
HÜ: Für 2.46 bis 2.48 wiederholen

kappa^2=0
w^2µsigma-kz^2=0
    kz=+-w/sqrt(µ.epsilon)
kappa^2!=0
w^2µsigma-kz^2=kx^2+kz^2 ... Der betrag der k-Vektors bleibt gleich, daher wird die kz Komponente wegen kx und kz kleiner
                                Aus dieser Erkentniss zeigt sich: Die TEM-Welle ist die schnellste Welle.
Notiz: Einngerahmte Formeln müssen auswendig gelernt werden, außer die Modalen Lösungen in kartesischen Koordinaten.
        Stattdessen das kappa^2 auswendig lernen!

\section{3. Die homogene ebene Welle}
\section{3.1 Die HEW im idealen Dielektrikum}
e ... elektrische Feldstärke
h ... magnetische Feldstärke
Kleinbuchstaben als Anzeichen der Zeitabhängigkeit.

pard_z^2e_x(z,t) - µepsilon.pard_z^2e_x(z,t) = 0
    mit e_x(z,t)=f_1(z-vt)+f_z(z+vt)
eingesetzt (nur mal für f1):
    f_1''(z-vt)-µepsilonf_1''(z-vt)(-v).(-v)=0
    1-µepsilonv^2 = 0  folgt v=+-1/sqrt(µepsilon)

h_y^+=1/nü.e_x^+  ... nü=sqrt(µ/epsilon)  nü_0=sqrt(µ_0/epsilon_0)=sqrt((4pi.10^-zH/m)^2.c_0^2)=12pi.10 Ohm = 120piOhm
                        ...Feldwellenwiderstand     ^^^^^^^^^^^^^                     hat nichts mit ohmschen Verlusten zu tun
                                                   epsilon_0.µ_0=1/c_0^2  epsilon_0=1/(µ_0.c_0^2)
\section*{19.10.2023}
\section*{Ding des Tages}
Der Prof. baute sich einmal eine Antenne, um mit der ISS beim FAQ mit einer Schule zuzuhören. Eine 50cm Stange nach links, eine 50cm Stange nach rechts. Die Stangen gehen per Bananenstecker in eine mysteriöse graue Box. Von unten wird ein Koax-Kabel angesteckt, bissl qualitativer, nicht so wie das vom grindigen Fernsehen.\newline
Es wird ein Balun benötigt (Bal=balanced, un=unbalanced). Dieses schließt eine balanzierte Leitung an eine unbalanzierte Leitung an.\newline
Abb.: Tafelbild + Diashow
Weil die Ströme gegensinnig, die Wicklung aber gleichsinnig ist, heben sich die Flüsse auf \implies $L=0, \omega_0=0$
Da bereits eine geringe Stromdifferenz eine relativ hohe Impedanz erzeugt, bestraft dieser Balun Differenzströme.\newline
Weil die Schule unterhalb des Horizontes lag, konnte der Prof. nur der ISS, nicht aber der Schule zuhören.
Der Prof. hat auch die Aufnahme zur Verfügung gestellt.

\subsection*{3.1.3 Energiedichte der HEW, Poyntingscher Vektor}
Zu jedem Zeitpunkt, an jedem Ort, ist die elektrische Energiedichte gleich groß wie die magnetische Energiedichte: sie gehen synchron.\newline
Abb. 3.1 Man sieht die Energiedichte eines Photons. Dort wo die Pfeile näher zusammen liegen, ist die Feldstärke erhöht. Man kann einen sinusförmigen Verlauf der Feldstärke erkennen.\newline
Weiters ist die Homogene Elektromagnetische Welle im freien Raum dargestellt. Unbedingt einprägen!\newline
\subsection*{3.1.4 Wellenzahl und Wellenlänge}
Unterschied zwischen k und \omega.
\section*{Polarisation}
Alle haben eine Polaristation. Unpolarisierte sind nur im mittel nicht polarisiert.
\subsection*{Polaristationsarten}
\subsubsection*{elliptische Polarisations (allgemeine Form)}


\end{document}
