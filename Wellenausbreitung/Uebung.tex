\documentclass[a4paper]{article}

\usepackage[a4paper,top=2cm,bottom=2cm,left=3cm,right=3cm,marginparwidth=1.75cm]{geometry}
\usepackage[utf8]{inputenc}
\usepackage[T1]{fontenc}
\usepackage{textcomp}
\usepackage[ngerman]{babel}
\usepackage{amsmath, amssymb, nccmath}
\usepackage{accents}

% figure support
\usepackage{import}
\usepackage{xifthen}
\pdfminorversion=7
\usepackage{pdfpages}
\usepackage{transparent}
\newcommand{\incfig}[1]{%
    \def\svgwidth{\columnwidth}
    \import{./figures/}{#1.pdf_tex}
}

\pdfsuppresswarningpagegroup=1

\title{Rechenübung\\Wellenausbreitung}
\author{DINC Atilla (11917652)}

\begin{document}
\normalsize
\maketitle
%\tableofcontent\newpage

% ~~~~~~~~~~~~~~~~~~~~~~~~~~~~ Start of the document ~~~~~~~~~~~~~~~~~~~~~~~~~~~~
\section*{31.10.2023}
\section*{Beispiel 5}
Abbildung: siehe Beispielskriptun

\begin{itemize}
    \item 1. Ansatz für einfallende und ausfallende Welle finden
        \[ \vec{E_{i}}=E_{0}e^{-jk_{z}z} \vec{e_{x}} \]
        \[ \vec{E_{r}}=E_{0}e^{+jk_{z}z}\vec{e_{x'}} \]
    \item 2. Transponieren
        gemoetrisch ergibt sich $x'=x\cos(2\phi)-z\sin(\phi)$
        und $y'=y$ 
        und $z'=x\sin(2\phi)+z\cos(2\phi)$
        \[ \vec{E_{r}}=E_{0}e^{jk_{z}(x\sin(2\phi)+z\cos(2\phi))}(\vec{e_{x}}\cos(2\phi)-\vec{e_{z}}\sin(2\phi)) \]
        \[\vec{E_{ges}}=\vec{E_{i}}+\vec{E_{r}} \]
        \[ =E_{0}(e^{-jk_{z}z}+\cos(2\phi)e^{jk_{z}(x\sin(2\phi)+z\cos(2\phi))})\vec{e_{x}}
        -E_{0}\sin(2\phi)e^{jk_{z}(x\sin(2\phi)+z\cos(2\phi))}\vec{e_{z}}\]
    \item 3. Hüllkurve bestimmen
        \[ Betrag(E_{ges,x})=Betrag(E_{0})\sqrt{irgendwas}  \]
        \[ =Betrag(E_{0}) \sqrt{1+\cos^{2}(2\phi) + 2\cos(2\phi)\cos(k_{z}z(1+\cos(2\phi)))} \]
    \item 4. Fallunterscheidung#
        \[ \phi =0\deg \implies Betrag(E_{ges,x})=Betrag(E_{0})2\cos(k_{z}z)  \]
        \[ \phi =45\deg \implies Betrag(E_{ges,x})=Betrag(E_{0})  \]
        \[ \phi =90\deg \implies Betrag(E_{ges,x})=0  \]
    \item 5. Minimale und Maximale Feldstärke
        \[ Betrag(E_{ges,x})=Betrag(E_{0}) \sqrt{1+\cos^{2}(2\phi) + 2\cos(2\phi)\cos(k_{z}z(1+\cos(2\phi)))} \]
        \[\cos(k_{z}z(1+\cos(2\phi))\]
        \ldots $min = -1$ und  $max=1$
        \[ Betrag(E_{ges,x})_{min}=Betrag(E_{0}(1-\cos(2\phi)))\]
        \[ Betrag(E_{ges,x})_{max}=Betrag(E_{0}(1+\cos(2\phi)))\]
        \[ m=\frac{E_{max}}{E_{min}}=\frac{1+\cos(2\phi)}{1-\cos(2\phi)}\text{,} 1 <m < \infty \]
    \item 6.
        \[ \cos(k_{z}z(1+\cos(2\phi)))=-1 \]
        \[ \cos(\pi(2n-1))=-1 \]
        \[ k_{z}z(1+\cos(2\phi))=\pi(2n-1) \]
        \[ k_{z}z_{n}(1+\cos(2\phi))=2\pin-\pi \]
        \[ k_{z}z_{n+1}(1+\cos(2\phi))=2\pin+\pi \]
        und er löscht es weg\ldots
\end{itemize}

\section*{Beispiel 6}
Abbildung: siehe Beispielskriptum
Beschreibung: schräg einfallende Welle im nicht idealen Raum trifft auf senkrechte Wand.
Welle breitet sich in z-Richtung aus, Wand ist parallel zur x-y-Ebene
\[  \epsilon_{r}=15 \]
\[ \sigma=1 \frac{mS}{m} \]
\[ \mu_{r}=1 \]
\begin{itemize}
    \item 1. Wellengeschwindigkeit
        \[ v_{P}=\frac{1}{\sqrt{\epsilon\mu}}=\frac{1}{\sqrt{\epsilon_{0}\mu_{0}} \sqrt{\epsilon_{r}\mu_{r}} }=\frac{c_{0}}{\sqrt{\epsilon_{r}} } =75\cdot 10^{10}\frac{m}{s}\]

    \item 2.
        \[ \vec{E_{i}(z)=E_{0}e^{-jk_{z}z}}\vec{e_{z}} \]
        \[ jk_{z}=\alpha+j\zeta=j\omega\sqrt{\mu\delta} =(0,049+j1,624)\frac{1}{m} \]
        \[ \alpha=0,049  \frac{NP}{m} \implies \alpha[dB]=0,42 \frac{dB}{m} \]

    \item 3. 
        \[ \vec{E_{i}}(z)=E_{0}e^{-jk_{z}z}\vec{e_{x}} \]
        \[ \vec{E_{i}(z=10cm)=}=1 \frac{V}{m}e^{-j(1,624-j0,049)\frac{1}{m}\cdot 10m}=()-0,53+j0,31)\vec{e_{x}}\frac{V}{m}\]
        \[ \vec{e}(z,t)=Re\{E_{i}(z)e^{j\omega t}\} \]
        \[ \vec{e}(10,t)=Re\{E_{0}e^{-jk_{z}z}e^{j\omega t}\} = 0,615 \frac{V}{m}\cos(\omega t - 16,24)\vec{e_{x}}\]

    \item 4.
        \[ \vec{E_{r}}(z)=Ae^{jk_{z}(z-z0)} \]
        \[ \vec{E_{ges}}(z)=\vec{E_{i}}(z)+\vec{E_{r}}(z) \]
        \[ =[E_{0}e^{-jk_{z}Z}] \]
        \[ \vec{E_{ges}}(z)=[E_{0}e^{-jk_{z}z}-E_{0}^e^{-jk_{z}z_{0}}e^{jk_{z}(z-z_0)}] \vec{e_{x}}\]
        \[ =[E_{0}e^{-jk_{z}z}-E_{0}e^{jk_{z}(z-2z_0)}]\vec{e_{x}} \]
        
    \item 5.
        \[ Betrag(E_{i})=\sqrt{E_{i}E_{i}\konjgiertkomplex}=\ldots=E_{0}e^{-\alpha z}\]
        \[ Betrag(E_{r})=\sqrt{E_{r}E_{r}\konjgiertkomplex}=\ldots=Ae^{\alpha(z-z_{0})} \]
        \[ \vec{E_{ges}}E_{0}(e^{-(\alpha+j\beta)}-e^{-(\alpha+j\beta)(z-2z_{0})}\vec{e_{x}}) \]
        \[ Betrag(\vec{E_{ges}})=\sqrt{E_{ges}E_{ges}\konj}=E_{0}\sqrt{e^{-2\alpha z}(1+e^{4\alpha (z-z_{0})})-2e^{-\alpha z_{0}}\cos(2\beta(z-z_{0}))}  \]

    \item 6.
        \[ alpha = 0 \]
        \[ Betrag(\vec{E_ges})=E_{0}\sqrt{1(1+1)-2\cdot 1\cos(2\beta(z-z_{0}))} \]
        \[ =E_{0}\sqrt{2(1-\cos2\beta(z-z_{0}))}  \]
        \[\cos(2\beta\Delta z)\overset{!}{=}1 \]
        \[ 2\beta\Delta z = 2\pi \]
        \[ \Delta z =\frac{2\pi}{2\beta}=\frac{\pi}{\frac{2\pi}{\lambda}}=\frac{\lambda}{2}\]
\end{itemize}
\end{document}
