\documentclass[a5paper]{article}

\usepackage[a4paper,top=2cm,bottom=2cm,left=3cm,right=3cm,marginparwidth=1.75cm]{geometry}
\usepackage[utf8]{inputenc}
\usepackage[T1]{fontenc}
\usepackage{textcomp}
\usepackage[ngerman]{babel}
\usepackage{amsmath, amssymb, nccmath}
\usepackage{accents}

% figure support
\usepackage{import}
\usepackage{xifthen}
\pdfminorversion=7
\usepackage{pdfpages}
\usepackage{transparent}
\newcommand{\incfig}[1]{%
    \def\svgwidth{\columnwidth}
    \import{./figures/}{#1.pdf_tex}
}

\pdfsuppresswarningpagegroup=1

\title{Mitschrift\\Schaltungstechnik}
\author{DINC Atilla (11917652)}

\begin{document}
\normalsize
\maketitle
%\tableofcontent\newpage

% ~~~~~~~~~~~~~~~~~~~~~~~~~~~~ Start of the document ~~~~~~~~~~~~~~~~~~~~~~~~~~~~

\section*{0. Vorbesprechung}

zukünftig von 11:15 bis 12:45\newline
Übungen immer in der 2. Stunde, Übungstermine folgen in TUWEL

\mathbf{Kapitelzusammenfassung:}
\begin{itemize}
    \item 6: Sehr wichtig, sehr detailreich, mehr informationen als in sonst einer vorlesung, rechenintensiv und prüfungsrelevant
    \item 7: gegenteil von 6, nicht sehr intensiv und durchlesen sollte eher reichen
    \item 8: Anwendung aller vorherigen Kapitel
\end{itemize}

\section*{1. Grundschaltungen}
\section*{1.1 Transistorgrundschaltungen}
\subsection*{1.1.1 Emitterschaltung}
\mathbf{brainstorming:}
\begin{itemize}
    \item negative Verstärkung weil ein positives signal, immer die Ausgangsspannung zur Masse zieht
    \item viel Spannungsverstärkung und viel Stromverstärkung
\end{itemize}

Emitter-Schaltung weil Signal und Emitter auf gleiches Potential (Masse) bezogen werden

TLDR: Kleinsignalersatzschaltbild - Schaltung wird mittels Stützkondensatoren,
diese sind bei hocher frequenz angenähert kurzgeschlossen -> VCC wird zu Masse

Steilheit $S=\frac{\partial I_{c}}{\partial U_{BE}} |_{UCE=const} = \frac{I_{c}}{U_{T}}=I_{c}\cdot \frac{q}{k_{B}\cdot T}=\frac{26mA}{26mV}=1S$ (wichtig!)

# Stromgegenkopplung
Widerstand am emitter

# 1.1.2 Kollektorschaltung
Zur Stromverstärkung (zB als Endstufe bei niedrigen Impedanzen, somit große Lasten treiben).
Der Emitterfolger ist klassisch.

# 1.1.3 Basisschaltung
positive Verstärkung, keine Stromverstärkung!

Der Rest der Grundschaltungen wird dort besprochen, wo es später gebraucht wird.

# 2. Leistungsverstärker
Spannung muss bereits zuvor aufbereitet worden sein.

Klasse A Verstärker: Signal wird bei hohem Arbeitspunkt (mittels hoher Ruhe-
stromversorgung) betrieben, um das gesamte Signal zu bewahren -> hohe Verluste
Klasse B Verstärker: geringe Verlustleistung

# 2.1 Betriebsarten und AP-Einstellung
Bis zum Erreichen der Verlustleistungshyperbel kann noch gekühlt werden.
Beim Erreichen zerstört sich der Transistor unabhängig von der Kühlung selbst.

Stromflusswinkel:
    A-Betrieb alpha = 360°
    B-Betrieb alpha ~= 180° (starke Übernahmeverzerrungen - die hälfte fehlt)
    AB-Betrieb alpha >=180° (geringe Übernahmeverzerrung um IA~=Iq)

Weitere Betriebsarten sind für die Vorlesung nicht sehr relevant.
Beispiele: C-Betrieb (alpha<180°), D-Betrieb (digital Pulse)

Im D-Betrieb gibt es kaum Verluste (hohe Effizienz), weil entweder komplett
durchgeschalten (kein Spannungsabfall) wird, oder komplett gesperrt (kein
Stromfluss) wird.

Der Arbeitspunkt kann mittels Emittergegenkopplung vor Examplarstreuungen
geschützt werden.
Aus der Basis-Masche der Emitterschaltung Ue=UBE+Ic*RE, wenn der 
Transistor statistisch mehr Strom leitet, sinkt UBE und der Strom sinkt.
Gleiches Prinzip zur thermischen Stabilisierung.

Der Emitterwiderstand kann mit einem parallel Kondensator für hohe Frequenzen
überbrückt werden.

\subsection*{30.10.2023}
\section*{VV-Breitbandverstärker}
Hohe Verstärkungen notwendig. Da stellt sich die Frage, welche Spannungsverstärkungen mit Transistorschaltungen möglich sind.

\subsection*{Maximale Verstärkung}

\subsection*{3.1.1 Breitband OV (VV-OPV)}

\subsection*{Kaskodenschaltung}
\subsubsection*{PNP-Kaskode}
\begin{itemize}
    \item T4 \& T6 erhöhen die Eingangsimpedanz
    \item Stromspiegel und Spannungsfolger sind auch mit dabei
\end{itemize}
\subsubsection*{gefaltete Kaskode}
\begin{itemize}
    \item Spart Transistoren
\end{itemize}

\subsubsection*{OPV mit komplementärem Kaskoden-Differenzverstärker}
Sehr hohe Verstärkung möglich

\subsubsection*{Breitband-Gegentakt-OV}
Ist aufgrund der konstantstromquellen auf $2I_{q}=2I_{O}$ beschränkt. Das ist sehr nachteilhaft für das Umladen von (Koppel-)Kapazitäten.

\subsubsection*{Breitband-Differenzverstärker im AB-Betrieb}
Dieses Prinzip, dass bei Bedarf sehr viel Strom geliefert werden kann, wird auch Current-On-Demand genannt.

\section*{Transkonduktanz-Verstärker (VC-OV)}
Typ: Voltage-Current Operationsverstärker (VC-OC)

Der gespiegelte Strom wird nicht mehr in eine Spannung umgesetzt.
Wird auch OTA genannt. Weil die Steilheit proportional zum Ruhestrom ist, ist 
die Steilheit einstellbar.

\subsubsection*{Wideband Transconductance Amplifier (WTA)}
$k_{I}=8$ zeigt an, dass die Emitterfläche 8-mal größer ausgelegt ist (z.B.
durch parallel geschaltete Transistoren) wodurch auch der 8-fache Strom geführt
 wird.

\subsubsection*{Typische Anwendung}
\begin{itemize}
    \item Treiber für Koaxialleiter
    \item Bandpassfilter
        siehe Skriptum S.33
\end{itemize}

\section*{Transimpedanz-Verstärker}
Typ: TIA, Strom-Spannungs Operationsverstärker

Joa, was auch immer, ist nicht so als könnte man dem folgen\ldots

Kaskodierter Stromspiegel für höhere Verstärkung.

Sehr beeindruckend hohe Slew-Rate!

\section*{Strom-Verstärker}
Typ: Strom-Strom Operationsverstärker (CC-OV)

Basis ist hochohmig, Emitter ist niederohmig und kann idealerweise Strom in
beide Richtungen fördern.

Joa, Rest ist es scheinbar wieder nicht wert, angemessen zu erklären.
Hauptsache man kann es in den 3min irgendwie reindrücken.


\section*{06.11.2023}
\section*{Reale OPVs}
\section*{Rückkopplungsnetzwerke}
\section*{Hohe Genauigkeit}
\section*{Elektronische Stromzähler}
\section*{Intelligente Stromzähler}
\section*{Vergleich Stromzähler}
\section*{Kennlinien Instrumentenverstärker}
\section*{Verstärkungsfehler}
Der Verstärkungsfehler muss für die Cent-genaue Abrechnung dimensioniert werden.
Man betrachte die invertierende und nicht-invertierende Verstärkerschaltung.
\section*{Empfindlichkeit rückgekoppelter Schaltungen}
\section*{Gleichtaktfehler}
\section*{Großsignalverhalten}
Der Transistor unten links ist im Stromquellenbereich der Kennlinie. DDieser Bereich ist jedoch nicht perfekt linear.
\section*{Eingangswiderstand}
Hat scheinbar 2 positive reale Eigenschaften.
\section*{Störungsunterdrückung durch Gegenkopplung}
Das Rauschverhältnis kann nicht durch Gegenkopplung verbessert werden.

\section*{Einschub: Gefaltete Kaskode}
Abbildung: siehe Tafelbild
Über unteren n-Mos ist eine Verstärkung von 1, die eigendliche Verstärkung wird
über den zusätzlichen Transistor durchgeführt. Dadurch wird der Eingang vom Ausgang getrennt. (Stichwort Millereffekt?)
Höhere Verstärkung, höherer Ausgangswiderstand, geringerer Spannungshub.

\section*{13.11.2023}
\section*{Stabilität rückgekoppelter Schaltungen}
\begin{itemize}
    \item Frequenzabhängigkeit des rückgekoppelten OV
    \item Interne Frequenzkompensation des OV
    \item \ldots
    \item 4.7 \ldots
\end{itemize}
Sollte diese und nächste Vorlesung abdecken.

OV als Verstärker: Rückkopplung zum invertierenden Eingang (Gegenkopplung $\neq 180^{\circ}$)
OV als Oszillator: Rückkopplung zum nichtinvertierenden Eingang (Mitkopplung $\approx 180^{\circ}$)

\subsection*{Frequenzabhängigkeit des rückgekoppelten OVs}
3 Verstärkerstufen -> 3 Tiefpässe -> 3 Grenzfrequenzen

NPN: Durch Rekombination ergibt sich niedrige Kapzität und hohe Stromverstärkung. sehr schnell
PNP: Hohe N-Wanne notwendig -> teuer (sehr schnell)
PNP (Lateral): Hoher Basisserienwiderstand, hohe Rekombination-> niedrige Stromverstärkung

Vorlesungsfolien wurden noch nicht hochgeladen. Ich hol die Vorlesungen irgendwann nach 

\section*{20.11.2023}
\section*{erste Stunde verpasst}

\section*{Probleme beim Integrator}
Eingangsruhestrom $I_{B}$ und Offsetsspannung $ U_{ed0}$ (können mit Strom- \& Spannungsquelle am invertierenden Eingang abgebildet werden)

\section*{Differentiator - Schwingneigung}

\section*{Logarithmierer}
\begin{itemize}
    \item mit Diode:
    \item mit Transistor:
\end{itemize}

\section*{Logarithmierer - Stabilität}
\section*{Temperaturkompensierter Logarithmierer}
Die Diode ist wieder zum Verpolungsschutz und die Kondensatoren sind zur Stabilisierung.

\section*{Zu den Prüfungen:}
Die Begrenzung auf 49 Plätze wird bei wenigen Leuten auf der Warteliste angepasst. Sollten viel mehr Leute auf der Warteliste sein, wird er je nach Hörsaal-Verfügbarkeit angepasst. Sollte man auf Platz 1, 2 oder 3 auf der Warteliste stehen, soll man trotzdem zur Prüfung kommen, es war noch nie jeder anwesend und selbst wenn, wird sich schon irgendwie ein Platz finden.


\section*{27.11.2023}
\subsection*{Temperaturkompensierter Logarithmierer}

\subsection*{Exponenzierer}
\[ U_{a}=I_{C}\cdot R_{1}=I_{CS}R_{1}e^{\frac{U_{e}}{U_{T}}} \]
\subsection*{Temperaturkompensierter Exponenzierer}

\section*{Gyrator}
Wandler von C zu L.

\section*{5 Analogschalter und Analogmultiplexer}
\subsection*{5.1 Anordnung der Schalter}

\subsection*{5.2 Elektronischer Schalter}
\begin{itemize}
    
    \item FET als Schalter:
        Transmission Gate, ist sehr anschaulich und kommt oft zur Prüfung.
    \item Dioden als Schalter:
        Raumladungszonen müssen umgeladen werden \implies Schaltgeschwindigkeiten begrenzt
    \item Bipolartransistor als Schalter:
        Die erste Schlatung ist aus der Mode gekommen und wird selten verwendet.
        Joa, nichts besonderes, sie lädt uns ein, es selbst zu lesen.
        Serien-KS-Schalter: Sehr schnell
        ECL-NOR-Gatter: Transistoren gehen nicht in Sättigung; sind wesentlich schneller als übliche C-MOS
    \item Differenzverstärker
    \item Transconductance-Verstärker
\end{itemize}

\subsection*{Breitband-Multiplexer}

\subsection*{5.3 Analogschalter mit Verstärker}


\end{document}
