\documentclass[a4paper]{article}

\usepackage[utf8]{inputenc}
\usepackage[T1]{fontenc}
\usepackage{textcomp}
\usepackage[ngerman]{babel}
\usepackage{amsmath, amssymb}


% figure support
\usepackage{import}
\usepackage{xifthen}
\pdfminorversion=7
\usepackage{pdfpages}
\usepackage{transparent}
\newcommand{\incfig}[1]{%
    \def\svgwidth{\columnwidth}
    \import{./figures/}{#1.pdf_tex}
}

\pdfsuppresswarningpagegroup=1

\begin{document}
\section*{0. Vorbesprechung}

zukünftig von 11:15 bis 12:45\newline
Übungen immer in der 2. Stunde, Übungstermine folgen in TUWEL

\bold{Kapitelzusammenfassung:}
-6: Sehr wichtig, sehr detailreich, mehr informationen als in sonst einer vorlesung, rechenintensiv und prüfungsrelevant
-7: gegenteil von 6, nicht sehr intensiv und durchlesen sollte eher reichen
-8: Anwendung aller vorherigen Kapitel


\section*{1. Grundschaltungen}
\section*{1.1 Transistorgrundschaltungen}
\subsection*{1.1.1 Emitterschaltung}
\bold{brainstorming:}
\begin{itemize}
    \item negative Verstärkung weil ein positives signal, immer die Ausgangsspannung zur Masse zieht
    \item viel Spannungsverstärkung und viel Stromverstärkung
\end{itemze}

Emitter-Schaltung weil Signal und Emitter auf gleiches Potential (Masse) bezogen werden

TLDR: Kleinsignalersatzschaltbild - Schaltung wird mittels Stützkondensatoren,
diese sind bei hocher frequenz angenähert kurzgeschlossen -> VCC wird zu Masse

Steilheit $S=\frac{\partial I_{c}}{\partial U_{BE}} |_{UCE=const} = \frac{I_{c}}{U_{T}}=I_{c}\cdot \frac{q}{k_{B}\cdot T}=\frac{26mA}{26mV}=1S$ (wichtig!)

# Stromgegenkopplung
Widerstand am emitter

# 1.1.2 Kollektorschaltung
Zur Stromverstärkung (zB als Endstufe bei niedrigen Impedanzen, somit große Lasten treiben).
Der Emitterfolger ist klassisch.

# 1.1.3 Basisschaltung
positive Verstärkung, keine Stromverstärkung!

Der Rest der Grundschaltungen wird dort besprochen, wo es später gebraucht wird.

# 2. Leistungsverstärker
Spannung muss bereits zuvor aufbereitet worden sein.

Klasse A Verstärker: Signal wird bei hohem Arbeitspunkt (mittels hoher Ruhe-
stromversorgung) betrieben, um das gesamte Signal zu bewahren -> hohe Verluste
Klasse B Verstärker: geringe Verlustleistung

# 2.1 Betriebsarten und AP-Einstellung
Bis zum Erreichen der Verlustleistungshyperbel kann noch gekühlt werden.
Beim Erreichen zerstört sich der Transistor unabhängig von der Kühlung selbst.

Stromflusswinkel:
    A-Betrieb alpha = 360°
    B-Betrieb alpha ~= 180° (starke Übernahmeverzerrungen - die hälfte fehlt)
    AB-Betrieb alpha >=180° (geringe Übernahmeverzerrung um IA~=Iq)

Weitere Betriebsarten sind für die Vorlesung nicht sehr relevant.
Beispiele: C-Betrieb (alpha<180°), D-Betrieb (digital Pulse)

Im D-Betrieb gibt es kaum Verluste (hohe Effizienz), weil entweder komplett
durchgeschalten (kein Spannungsabfall) wird, oder komplett gesperrt (kein
Stromfluss) wird.

Der Arbeitspunkt kann mittels Emittergegenkopplung vor Examplarstreuungen
geschützt werden.
Aus der Basis-Masche der Emitterschaltung Ue=UBE+Ic*RE, wenn der 
Transistor statistisch mehr Strom leitet, sinkt UBE und der Strom sinkt.
Gleiches Prinzip zur thermischen Stabilisierung.

Der Emitterwiderstand kann mit einem parallel Kondensator für hohe Frequenzen
überbrückt werden.
\end{document}
