\documentclass[a4paper]{article}

\usepackage[a4paper,top=2cm,bottom=2cm,left=3cm,right=3cm,marginparwidth=1.75cm]{geometry}
\usepackage[utf8]{inputenc}
\usepackage[T1]{fontenc}
\usepackage{textcomp}
\usepackage[ngerman]{babel}
\usepackage{amsmath, amssymb, nccmath}
\usepackage{accents}

% figure support
\usepackage{import}
\usepackage{xifthen}
\pdfminorversion=7
\usepackage{pdfpages}
\usepackage{transparent}
\newcommand{\incfig}[1]{%
    \def\svgwidth{\columnwidth}
    \import{./figures/}{#1.pdf_tex}
}

\pdfsuppresswarningpagegroup=1

\title{Übungseinheit\\Schaltungstechnik}
\author{DINC Atilla (11917652)}

\begin{document}
\normalsize
\maketitle
%\tableofcontent\newpage

% ~~~~~~~~~~~~~~~~~~~~~~~~~~~~ Start of the document ~~~~~~~~~~~~~~~~~~~~~~~~~~~~
\section*{Beispiel 3:}
siehe Berechnung im Skriptum

\section*{Beispiel 4: Dimensionierung eines Kühlblechs}

\section*{Beispiel 6: Subtrahierverstärker}

\end{document}
