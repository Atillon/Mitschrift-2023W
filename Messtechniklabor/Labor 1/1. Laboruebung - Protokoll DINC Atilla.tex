\documentclass[a4paper]{article}

\usepackage[a4paper,top=2cm,bottom=2cm,left=3cm,right=3cm,marginparwidth=1.75cm]{geometry}
\usepackage[utf8]{inputenc}
\usepackage[T1]{fontenc}
\usepackage{textcomp}
\usepackage[ngerman]{babel}
\usepackage{amsmath, amssymb, nccmath}
\usepackage{accents}

% figure support
\usepackage{import}
\usepackage{xifthen}
\pdfminorversion=7
\usepackage{pdfpages}
\usepackage{transparent}
\newcommand{\incfig}[1]{%
    \def\svgwidth{\columnwidth}
    \import{./figures/}{#1.pdf_tex}
}

\pdfsuppresswarningpagegroup=1

\title{Protokoll zur ersten Laborübung\\Messtechnik Labor 376.091}
\author{DINC Atilla (11917652)}

\begin{document}
\normalsize
\maketitle
\tableofcontents
\begin{center}
\begin{tabular}{|c| c| c| c|}
    Nummer  & Bezeichnung  & Beschreibung                 & Grössen \\ 
    \hline
    -       & MM1          & Digitalmultimeter            & $R$, $U$, $I$ \\
    -       & MM2          & Digitalmultimeter            &  $R$, $U$, $I$ \\
    -       & OZ1          & Digitalspeicheroszilloskop   & $U(t)$ \\
    -       & FG1          & Funktionsgenerator           & $u(t)$
\end{tabular}
\end{center}
\newpage

% ~~~~~~~~~~~~~~~~~~~~~~~~~~~~ Start of the document ~~~~~~~~~~~~~~~~~~~~~~~~~~~~

\section{Vorbereitung}

\section{1. Versuch}
\section{2. Versuch}
\section{3. Versuch}
\section{4. Versuch}

\end{document}
