\documentclass[a4paper]{article}

\usepackage[a4paper,top=2cm,bottom=2cm,left=3cm,right=3cm,marginparwidth=1.75cm]{geometry}
\usepackage[utf8]{inputenc}
\usepackage[T1]{fontenc}
\usepackage{textcomp}
\usepackage[ngerman]{babel}
\usepackage{amsmath, amssymb, nccmath}
\usepackage{accents}

\usepackage{multirow}

% figure support
\usepackage{import}
\usepackage{xifthen}
\pdfminorversion=7
\usepackage{pdfpages}
\usepackage{transparent}
\newcommand{\incfig}[1]{%
    \def\svgwidth{\columnwidth}
    \import{./figures/}{#1.pdf_tex}
}

\pdfsuppresswarningpagegroup=1

\title{Protokoll zur ersten Laborübung\\Messtechnik Labor 376.091}
\author{DINC Atilla (11917652)}

\begin{document}
\newcommand{\unit}[1]{\ensuremath{\, \mathrm{#1}}} % Einheiten in Math-Moder richtig formatieren
\normalsize
\maketitle
\tableofcontents

\begin{center}
\begin{tabular}{|c| c| c| c| c|}
    \hline
    \multicolumn{5}{|c|}{Geräteliste} \\
    \hline

    Bezeichnung & Gerätebeschreibung           & Messgrößen & Inventarnummer & Bemerkungen\\ 
    \hline
    MM0          & Agilent Digitalmultimeter    & - & U1232A & -\\
    MM1          & Digitalmultimeter            & - & #11 & -\\
    MM2          & Digitalmultimeter            & - & #7 & -\\
   \multirow{2}{*}{OZ1}&
   \multirow{2}{*}{
       \begin{tabular}[c]
           Digitalspeicheroszilloskop DSO-x2002A\\
           MMSR
       \end{tabular}
}&
   \multirow{2}{*}{Ch2}&
   \multirow{2}{*}{C0404-5}&
   \multirow{2}{*}{-}\\
        &   &   &   &   \\
    FG1          & Funktionsgenerator           & $u(t)$ & SDG1025 & -\\
    \hline
    \multicolumn{5}{|c|}{Zubehörliste} \\
    \hline

    Bezeichnung & Zubehörbeschreibung           & Messgrößen & Inventarnummer & Bemerkungen\\ 
    \hline
    K1          & Tastkopf (10:1) 100\unit{MHz} 10\unit{M\Omega} 15\unit{pf} & - & - & rot\\
    K2          & Tastkopf (10:1) 150\unit{MHz} 10\unit{M\Omega} 15\unit{pf} & - & - & grau\\
    K3          & Tastkopf (10:1) 150\unit{MHz} 10\unit{M\Omega} 15\unit{pf} & - & - & rosa\\
\hline

\end{tabular}
\end{center}
\newpage
% ~~~~~~~~~~~~~~~~~~~~~~~~~~~~ Start of the document ~~~~~~~~~~~~~~~~~~~~~~~~~~~~

\section{Einleitung}

\section{Messungen mit dem Digitalmultimeter}
\subsection{Spannungsmessung}
\subsection{Strommessung}
\subsection{Widerstandsmessung}
\section{Messungen mit dem Oszilloskop}
\subsection{Tastkopf}
\subsection{AC-Spannungsmessung}
\subsection{RMS im Detail}
\subsection{Amplitudenauflösung}
\subsection{Dynamik}


\end{document}
