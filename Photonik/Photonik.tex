\documentclass[a4paper]{article}

\usepackage[utf8]{inputenc}
\usepackage[T1]{fontenc}
\usepackage{textcomp}
\usepackage[german]{babel}
\usepackage{amsmath, amssymb}


% figure support
\usepackage{import}
\usepackage{xifthen}
\pdfminorversion=7
\usepackage{pdfpages}
\usepackage{transparent}
\newcommand{\incfig}[1]{%
    \def\svgwidth{\columnwidth}
    \import{./figures/}{#1.pdf_tex}
}

\pdfsuppresswarningpagegroup=1

\begin{document}
\subsection*{19.10.2023}
\section*{Wiederholung}
Ableitung des Bildchens: 2 Halbkreise
\begin{itemize}
    \item oberer Halbkreis, einfallende Welle mit $n_i$
    \item unterer Halbkreis, transmittierte Welle mit $n_t$
\end{itemize}
Weiters: Fresnel-Koeffizienten
\[
    E_R^{st}=t.E_i^{sp}
    E_R^irgendwas iregendwa
    Annahme: n_{ix}=n_{iy}=n_{iz}
.\] 
Nun sollen anisotrope Medien betrachtet werden.
Das Medium kann mit Federkonstanten beschrieben werden.
Wenn ein Teilchen mit horizontalen und vertikalen Ferdern suspendiert wird, und diese Federn unterschiedliche Eigenschaften haben, dann ist das Medium auch auf die Propergationsrichtung empfindlich.\newline
\[
    \vec E = \vec E_0 . \exp^{j(wt-kx)}
.\] 
ein bisschen fehlt. Siehe Folie Anisotrope Medien (i) für Abbildungen.

Wellengleichungen:
\begin{itemize}
    \item 1) $\vec \nabla \vec D = \vec \nabla .(\epsilon.\vec E+\vec r)=0$ $\vec E=\vec E_0.\exp^{j(wt-kx)}$
         $-j\vec k(\epsilon_0 \vec E+\veP)=0$  $\nabla $
     \item 3) $\vec \nabla \x \vec H&= \delta \frac{D}{\delta t} \crossed{\vec J}$ 
         $-j\vec k \x \vec H = j\omega \vec D$
\end{itemize}
Man betrachte den Poyntingvektor $\vec S=\vec E \x \vec H$. Der Poyntingvektor liegt nicht mehr parallel zu  $\vec k$.\newline
Es breitet sich eine Welle aus, die nicht mehr parallel zum Wellenvektor ist.

\bold{Definition:}
\begin{itemize}
    \item 1)  $n_x\neq n_y\neq  n_z$ \ldots bi-axial, zweiachsig
    \item 2) $n_x=n_y=n_o$ \ldots ordinary
        $n_z=n_e$ \ldots extraordinary
    \item 3) \ldots isotrop
\end{itemize}
Er will die Wellengleichungen ableiten und klären, was \newline
\[
\vec \nabla \x (\vec \nabla \x \vec E) = -µ_0\frac{\delta}{\delta t}(\vec \nabla \x k)
\vec \nabla \x(\vec \nabla \vec E)=-µ_0 \frac{\delta^2 \ldots}{\delta t^2}
\vec \nabla (\vec \nabla \vec E)- \vec \nabla ^2 \vec E = \ldots
\vec \nabla \implies -jk
\delta t \implies j\omega
-\vec k(\vec k\vec E)+k^2.\vec E=µ_0.\omega^2\epsilon_0\epsilon.\vec E
\vec k = k_0.n.\vec e, \tab k_0=\frac{\omega}{c_0}
n^2\vec E -n^2.\vec e(\vec e.\vec E)-\epsilon_\tilde.\vec E = 0, \epsilon_\tilde = [[\epsilon_x, 0,0][0,\epsilon_y, 0][0, 0, \epsilon_z] ]
M_\tilde.\vec E = 0
.\] 
Mit $det M_\tilde = 0$ folgt deie Gleichung  $n^4.A+n^2.B+C=0$ mit den beiden Lösungen  $n_{1,2}^2=- \frac{B+-\sqrt{B^2-4AC} }{2A}$.
Die Gleichungen zur Bestimmung der Brechungsindizes lauten:
\[
n_1=n_o
\frac{1}{n_2^2}=\frac{e_z^2}{n_o^2}+\frac{e_y^2}{n_e^2}
k=k_o.n_2
\frac{k_z^2}{k_0^2.n_o^2}+\frac{k_y^2}{k_0^2.n_e^2}=1\ldots\text{Kreis/Ellipse}
.\] 
Es kann sein dass Nuller und Os vertauscht wurden (hat auch der Prof gesagt). Deshalb am besten im Buch nachschauen.
Der Professor hat sich irgendwo verrechnet, aber wenn wir seiner Rechnung trotzdem mit Vertrauen folgen, kommen wir irgendwann ans richtige Ziel\ldots
Für Zusammenfassung im Buch nachschauen. Die Differenzierung zwischen optischer Achse, uniaxales Medium, etc. ist etwas umfangreicher.\newline
Wir haben jetzt abgeleitet. Es gibt 2 Brechnungsindizes: einer ist abhängig vom Winkel, der andere ist immer $n_o$. Natürlich kann nicht eine Welle 2 Brechungsindizes sehen. Die beiden unterschiedlichen Polarisationsrichtungen sehen die beiden unterschiedlichen Brechungsindizes, auch wenn sie sich den Wellenvektor teilen.\newline
Am Beispiel lassen wir eine Welle in $\vec e_z$ ausbreiten:
 \[
     \vec e_z: [[n^2-n_o^2, 0, 0][0, n^2-no^2, 0][0,0,-n_e^2]].\vec E = 0
     \vec e_x\text{und}\vec E_y \text{sehen beide} n_o
.\] 
Interessanter ist der Fall, mit der Ausbreitungsrichtung in $e_y$
\[
e_y
e_x=e_z=0
[[n^2-n_o^2, 0, 0][0, -n_o^2, 0][0,0,n^2-n_e^2]]\vec E = 0
E_x.(n^2-n_o^2)=0 \implies n^2 = n_o^2 \implies n=n_o
E_y(-n_o^2)=0 \implies E_y=0
E_z(n^2-n_e^2)=0 \implies n^2 = n_e^2 \implies n = n_e
.\]
Bitte merken bezogen zur Ebene E (Gespannt von Optische Achse und Ausbreitungsrichtung):
\begin{itemize}
    \item senkrecht zu E ist no (ordentlich)
    \item parallel zu E it ne (außerordentlich)
\end{itemize}
Aus der Strahlgeschwindigkeit kann der Winkel zwischen dem E-Feld und dem D-Feld bestimmt werden. Ausgangspunkt ist der Poyntingvektor, der schräg zur Ausbreitungsrichtung liegt.
\[
    v_p=\frac{\omega}{k}=\frac{\omega}{\frac{\omega}{c_0}n}=\frac{c_0}{n}
    v_=\frac{d\omega}{dk}\ldots Gruppengeschwindigkeit (sollte man scheinbar gelernt haben)
    \vec v_S=\vec \nabla \omega \ldots\text{Strahlgeschwindigkeit}
.\] 
Auflösung: Der Strahl der nicht normal zu den Phasenflächen liegt (den Wellenfronten) wird der außergewöhnlich (extraordinary) Strahl genannt.

Verdeutlichung mit zwei Beispiele in anisotropen Anordnungen.\newline
\subsubsection*{Die Wellenplatte}\newline
Wir legen fest, dass die Eintrittsfläche parallel zur optischen Achse (OA) ($e_z$-Richtung) ist.
Die Welle verläuft in $e_y$-Richtung, was machen  $E_x$ und  $E_z$ nach einer gewissen Propagationsdistanz d.
$n_o$ liegt in der  $e_x$-Richtung
 $n_e$ liegt in der  $e_y$-Richtung
\[
y=k.d=k_o.n.d
y_x=k_0.d.n_o
y_z=k_0.d.n_e
\vec J_in=(E_x, E_z)
\vec J_out=(E_{xout}, E_{zout})=((e^jy_x, 0)(0, e^jy_z)).(E_{xin}, E_{{zin}})=e^{jy_x}.((1, 0)(0, e^j\Delta y)).(E_{xin}, E_{zin})
\Delta y = k_0.d(n_o-n_e)=!\frac{\pi}{2}
\Delta n d=\frac{\pi}{2} \frac{1}{\frac{2\pi}{\lambda}}=\frac{\lambda}{4}
.\] 
Lambda halbe macht pi halbe polarisation
Keine Ahnung, schau Tafelbild. Hat er in 30sec erklärt.

\subsubsection*{Beispiel 2 nächste Woche}


\end{document}
