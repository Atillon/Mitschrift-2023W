\documentclass[a4paper]{article}

\usepackage[utf8]{inputenc}
\usepackage[T1]{fontenc}
\usepackage{textcomp}
\usepackage[german]{babel}
\usepackage{amsmath, amssymb}


% figure support
\usepackage{import}
\usepackage{xifthen}
\pdfminorversion=7
\usepackage{pdfpages}
\usepackage{transparent}
\newcommand{\incfig}[1]{%
    \def\svgwidth{\columnwidth}
    \import{./figures/}{#1.pdf_tex}
}

\pdfsuppresswarningpagegroup=1

\begin{document}
\subsection*{19.10.2023}
\section*{Wiederholung}
Ableitung des Bildchens: 2 Halbkreise
\begin{itemize}
    \item oberer Halbkreis, einfallende Welle mit $n_i$
    \item unterer Halbkreis, transmittierte Welle mit $n_t$
\end{itemize}
Weiters: Fresnel-Koeffizienten
\[
    E_R^{st}=t.E_i^{sp}
    E_R^irgendwas iregendwa
    Annahme: n_{ix}=n_{iy}=n_{iz}
.\] 
Nun sollen anisotrope Medien betrachtet werden.
Das Medium kann mit Federkonstanten beschrieben werden.
Wenn ein Teilchen mit horizontalen und vertikalen Ferdern suspendiert wird, und diese Federn unterschiedliche Eigenschaften haben, dann ist das Medium auch auf die Propergationsrichtung empfindlich.\newline
\[
    \vec E = \vec E_0 . \exp^{j(wt-kx)}
.\] 
ein bisschen fehlt. Siehe Folie Anisotrope Medien (i) für Abbildungen.

Wellengleichungen:
\begin{itemize}
    \item 1) $\vec \nabla \vec D = \vec \nabla .(\epsilon.\vec E+\vec r)=0$ $\vec E=\vec E_0.\exp^{j(wt-kx)}$
         $-j\vec k(\epsilon_0 \vec E+\veP)=0$  $\nabla $
     \item 3) $\vec \nabla \x \vec H&= \delta \frac{D}{\delta t} \crossed{\vec J}$ 
         $-j\vec k \x \vec H = j\omega \vec D$
\end{itemize}
Man betrachte den Poyntingvektor $\vec S=\vec E \x \vec H$. Der Poyntingvektor liegt nicht mehr parallel zu  $\vec k$.\newline
Es breitet sich eine Welle aus, die nicht mehr parallel zum Wellenvektor ist.

\bold{Definition:}
\begin{itemize}
    \item 1)  $n_x\neq n_y\neq  n_z$ \ldots bi-axial, zweiachsig
    \item 2) $n_x=n_y=n_o$ \ldots ordinary
        $n_z=n_e$ \ldots extraordinary
    \item 3) \ldots isotrop
\end{itemize}
Er will die Wellengleichungen ableiten und klären, was \newline
\[
\vec \nabla \x (\vec \nabla \x \vec E) = -µ_0\frac{\delta}{\delta t}(\vec \nabla \x k)
\vec \nabla \x(\vec \nabla \vec E)=-µ_0 \frac{\delta^2 \ldots}{\delta t^2}
\vec \nabla (\vec \nabla \vec E)- \vec \nabla ^2 \vec E = \ldots
\vec \nabla \implies -jk
\delta t \implies j\omega
-\vec k(\vec k\vec E)+k^2.\vec E=µ_0.\omega^2\epsilon_0\epsilon.\vec E
\vec k = k_0.n.\vec e, \tab k_0=\frac{\omega}{c_0}
n^2\vec E -n^2.\vec e(\vec e.\vec E)-\epsilon_\tilde.\vec E = 0, \epsilon_\tilde = [[\epsilon_x, 0,0][0,\epsilon_y, 0][0, 0, \epsilon_z] ]
M_\tilde.\vec E = 0
.\] 
Mit $det M_\tilde = 0$ folgt deie Gleichung  $n^4.A+n^2.B+C=0$ mit den beiden Lösungen  $n_{1,2}^2=- \frac{B+-\sqrt{B^2-4AC} }{2A}$.
Die Gleichungen zur Bestimmung der Brechungsindizes lauten:
\[
n_1=n_o
\frac{1}{n_2^2}=\frac{e_z^2}{n_o^2}+\frac{e_y^2}{n_e^2}
k=k_o.n_2
\frac{k_z^2}{k_0^2.n_o^2}+\frac{k_y^2}{k_0^2.n_e^2}=1\ldots\text{Kreis/Ellipse}
.\] 
Es kann sein dass Nuller und Os vertauscht wurden (hat auch der Prof gesagt). Deshalb am besten im Buch nachschauen.
Der Professor hat sich irgendwo verrechnet, aber wenn wir seiner Rechnung trotzdem mit Vertrauen folgen, kommen wir irgendwann ans richtige Ziel\ldots
Für Zusammenfassung im Buch nachschauen. Die Differenzierung zwischen optischer Achse, uniaxales Medium, etc. ist etwas umfangreicher.\newline
Wir haben jetzt abgeleitet. Es gibt 2 Brechnungsindizes: einer ist abhängig vom Winkel, der andere ist immer $n_o$. Natürlich kann nicht eine Welle 2 Brechungsindizes sehen. Die beiden unterschiedlichen Polarisationsrichtungen sehen die beiden unterschiedlichen Brechungsindizes, auch wenn sie sich den Wellenvektor teilen.\newline
Am Beispiel lassen wir eine Welle in $\vec e_z$ ausbreiten:
 \[
     \vec e_z: [[n^2-n_o^2, 0, 0][0, n^2-no^2, 0][0,0,-n_e^2]].\vec E = 0
     \vec e_x\text{und}\vec E_y \text{sehen beide} n_o
.\] 
Interessanter ist der Fall, mit der Ausbreitungsrichtung in $e_y$
\[
e_y
e_x=e_z=0
[[n^2-n_o^2, 0, 0][0, -n_o^2, 0][0,0,n^2-n_e^2]]\vec E = 0
E_x.(n^2-n_o^2)=0 \implies n^2 = n_o^2 \implies n=n_o
E_y(-n_o^2)=0 \implies E_y=0
E_z(n^2-n_e^2)=0 \implies n^2 = n_e^2 \implies n = n_e
.\]
Bitte merken bezogen zur Ebene E (Gespannt von Optische Achse und Ausbreitungsrichtung):
\begin{itemize}
    \item senkrecht zu E ist no (ordentlich)
    \item parallel zu E it ne (außerordentlich)
\end{itemize}
Aus der Strahlgeschwindigkeit kann der Winkel zwischen dem E-Feld und dem D-Feld bestimmt werden. Ausgangspunkt ist der Poyntingvektor, der schräg zur Ausbreitungsrichtung liegt.
\[
    v_p=\frac{\omega}{k}=\frac{\omega}{\frac{\omega}{c_0}n}=\frac{c_0}{n}
    v_=\frac{d\omega}{dk}\ldots Gruppengeschwindigkeit (sollte man scheinbar gelernt haben)
    \vec v_S=\vec \nabla \omega \ldots\text{Strahlgeschwindigkeit}
.\] 
Auflösung: Der Strahl der nicht normal zu den Phasenflächen liegt (den Wellenfronten) wird der außergewöhnlich (extraordinary) Strahl genannt.

Verdeutlichung mit zwei Beispiele in anisotropen Anordnungen.\newline
\subsubsection*{Die Wellenplatte}\newline
Wir legen fest, dass die Eintrittsfläche parallel zur optischen Achse (OA) ($e_z$-Richtung) ist.
Die Welle verläuft in $e_y$-Richtung, was machen  $E_x$ und  $E_z$ nach einer gewissen Propagationsdistanz d.
$n_o$ liegt in der  $e_x$-Richtung
 $n_e$ liegt in der  $e_y$-Richtung
\[
y=k.d=k_o.n.d
y_x=k_0.d.n_o
y_z=k_0.d.n_e
\vec J_in=(E_x, E_z)
\vec J_out=(E_{xout}, E_{zout})=((e^jy_x, 0)(0, e^jy_z)).(E_{xin}, E_{{zin}})=e^{jy_x}.((1, 0)(0, e^j\Delta y)).(E_{xin}, E_{zin})
\Delta y = k_0.d(n_o-n_e)=!\frac{\pi}{2}
\Delta n d=\frac{\pi}{2} \frac{1}{\frac{2\pi}{\lambda}}=\frac{\lambda}{4}
.\] 
Lambda halbe macht pi halbe polarisation
Keine Ahnung, schau Tafelbild. Hat er in 30sec erklärt.

\subsubsection*{Beispiel 2 nächste Woche}

\section*{09.11.2023}
\section*{Ausbesserung der Nebenrechnung}
Wird noch im TISS hochgeladen.

\subsection*{Wiederholung: Doppelbrechnung}
Durch den ausserordentlichen Brechungsindex wird eine weiter Brechung hervorgerufen. Die ordentliche Welle folgt einem Kreis, während die ausserordentliche Welle durch eine Ellipse beschrieben wird, siehe Tafelbild.

\section*{Ding des Tages}
Glam-Taylor-Polarisator
Zwei Medien (oft Calzit) bilden einen Rechteck, Die Kontaktfläche der Medien ist eine Schräge, welche mit einem Klebstoff versehen ist. Die Medien haben je einen ordentlichen und einen ausserordentlichen Brechungsindex.

Da die ausserordentlichen Polarisatoren an einer Ellipse projeziert werden, kann ein Winkel getroffen werden, wo die ausserordentliche Welle durchgelassen, die ordentliche Welle jedoch total reflektiert wird.

Siehe weiters Rachon-Prisma und das Wollaston-Prisma. Siehe Tafelbild

\section*{Interferenz}
Bei orthogonal Komponenten kann es nicht zu Interferenzen kommen, da das Innprodukt 0 ist. Daher werden parallele Komponenten benötigt.
Wie in Halbleiterphysik gelernt, werden nie wirklich elektrische Felder, sondern deren Intensitäten als Betragsquadrat $I=\|E_{1}+E_{2}\|^{2}$ gemessen.
\[ I=(E_{1}+E_{2})(E_{1}+E_{2})komplkonj =  E_{1}E_{1}\komkonj +E_{2}E_{2}\komkonj + E_{1}\komkonj E_{2} + E_{2}\komkonj E_{1} = I_{0} + I_{0} + 2I_{0}\cos(\Delta t) = 2I_{0}(1+\cos(\Delta t))\]
Plot hinzufügen

Wie erhält man nun parallele Komponenten? Stichwort: Interferrometer

Eine Spiegel teilt einen Strahl in zwei Strahlen mit der halben Strahlungsintensität $I_{1}$ und $I_{2}$ auf. Mit Zwei 100% spiegeln werden beide Strahlen wieder rechtwinkelig zueinander gebündelt. Im Knotenpunkt wird wieder ein 50% Spiegel angebracht, welcher die beiden 50% Strahlen in 4 25% Strahlen aufspaltet, und im Endergebnis 2 50% Strahlen erzeugt, die über zwei unterschiedliche Pfade erzeugt wurden. Auf einem Pfad kann nun ein Verzögerungselement eingebaut werden.
\[ I_{1}=\frac{I_{0}}{4}2(1+\cos(\Delta \Phi))=\frac{I_{0}}{2}(1+\cos(\Delta \Phi)) \]

Er spricht sehr viel über die Empfindlichkeit von Interferometern, da sich Änderungen im Nanometerbereich auf die Intensität auswirken können.

\subsubsection*{Michelson-Interferometer}
Die $R=1$ Spiegel (100% Spiegel) zeigen wieder zurück auf den erste $R=0.5$ Spiegel. Das "Queen of Michelson" ist der Michelson-Interferometer, welches zur Observation von Gravitationswellen gebaut wurde, das LIGO.


\subsection*{irgendwas}
Wir schauen uns mal an, wenn wir nicht nur eine Frequenz sondern ein Frequenzpacket haben \ldots glaube ich zumindest.
Wir haben eine Welle gegeben durch
\[ E(z,t)=E_{0}(\omega)e^{j(\omega t-kz)}=E_{0}(\omega)e^{}\]

Wird diese Welle durch ein Interferometer geschickt, so erhält man
\[ E(0,t)=E_{0}(\omega)e^{j\omega t} \]
\[ I(\oemga)=I_{0}(\omega)2(1+\cos(\Delta \Phi)) \]

Man erhält: siehe Abbildung im Tafelbild

\[ \Delta\omega\cdot \Delta t = 2\pi \]
\[ (\omega_{2}-\omega_{1})\Delta t = 2\pi \]
\[ \Delta t = \frac{2\pi}{\omega_{2}-\omega_{1}} \]

\section*{Etalon, Fabry-Perot Interferometer}
Siehe Abbildung im Tafelbild (ist scheinbar sehr gut im Video zu sehen): zwei parallele Platten, die Welle wird schräge eingeführt und wird mehrmals reflektiert.
\[ \Phi_{0}=k_{0}n \frac{d}{\cos(\Theta)} \]
\[ \Delta\Phi=2k_{0}nd\cos(\Theta) \]
Die Herleitung ist nicht sehr einfach zu sehen aber grundsätzlich nicht schwer. Im Video ist es sauber durchgeführt. (Seine Worte)

Die Wellen werden beschrieben als:
\begin{itemize}
    \item $E_{0}t_{01}e^{j\Phi_{0}}t_{12}$
    \item $E_{0}t_{01}e^{j\Phi_{0}}t_{12}r_{01}r_{12}e^{j\Delta\Phi}$
    \item $E_{0}t_{01}e^{j\Phi_{0}}t_{12}r_{01}^{2}r_{12}^{2}e^{j2\Delta\Phi}$
\end{itemize}
\[ E_{ges}=E_{0}t_{01}e^{j\Phi_{0}}t_{12} \sum_{n=0}{\infty} (r_{01}r_{12}e^{j\Delta\Phi})^{n} \]
\[ E_{ges}=E_{0}t_{01}e^{j\Phi_{0}}t_{12} \frac{1}{1-r_{01}r_{12}e^{j\Delta\Phi}} \]

es fehlt ein wenig
\implies 
\[ \frac{I_{t}}{I_{0}}=\frac{T^{2}}{1-2R\cos(\Delta\Phi)+R^{2}} \]

Bitte im Video nachgucken
\[ \Delta\Phi=0 \implies \frac{I_{t}}{I_{0}}=\frac{T^{2}}{(1-R)^{2}}=\frac{(1-R^)^{2}}{(1-R)^{2}}=1\]
\[ R+T=1 \]
\[ \cos(\Delta t) =1-2\sin^{2}(\frac{\Delta t}{2})\]

\[ \frac{I_{t}}{I_{0}}=\frac{T^{2}}{(1-R)^{2+4R\sin^{2}(\frac{\Delta t}{2})}}=\frac{T^{2}}{(1-R)^{2}} \frac{1}{1+\frac{4R}{(1-R)^{2}\sin^{2}(\frac{\Delta t}{2})}} \]
Finesse: $F=\pi \frac{\sqrt{\pi} }{1-R}$ 
\[ \frac{I_{t}}{I_{0}}=\frac{1}{1+(\frac{2F}{\pi})^{2}\sin^2 \frac{\Delta t}{2}} \]

Siehe Abbildung in Tafelbild: Plot der Transmittivität bei unterschiedlichen Finesse's. Es zeigen sich Peaks bei unterschiedlichen $\omega_{m}$ 's
\[ \omega_{m}=\frac{\pi c_{0}m}{\cos(\Theta)nd} \]
\[ \Delta \omega = \omega_{m,1}-\omega_{m}=\frac{\pi c_{0}}{\cos(\Theta nd)}\]
\[ \Delta\omega_{res}=\frac{\Delta\omega}{F} \]

Mit diesem Interferometer kann man durch Durchstimmung von $n$,  $d$ und  $\Theta$ irgendwas machen.
Scheinbar hat er eine Diplomarbeit damit gemacht, der Student hat somit ein optisches Mikrofon gebaut.

\section*{16.11.2023}
\section*{Wiederholung}
Interferrometer mit 3 Brechungsindizes: erzeugt mit jeder internen Reclexion eine in $n_{2}$ ausbreitende Welle.
Durch Superposition erhalten wir ein Interferenzmuster mit Peaks, welche einen sogenannten Rsonanzabstand $\Delta\omega_{r}$ haben.

\section*{Interferenzen, Tranmissionen und Reflexionen in Kavitäten}
|    $n_{1}$   |
R=1      R=1
Stehende Wellen mit $m \frac{\lambda_{0}}{2}=n_{1}d$ 


|    $n_{1}$   |
R=0.8        R=0.8
T=0.2        T=0.2

Wieso sind die blauen Peaks anders geformt als die reflektierte Intensitäten?
Siehe Tafelbild

\subsection*{eine Filmschicht auf einem Substrat}
Anti-Reflex-Beschichtung an einer Brille
$n_{0}$ | $n_{1}$ | $n_{s}$ 
Luft   |        | Glas
       |        |
       |        |
\ldots      |        |
$I_{0}\|t\|^{2}e^{j\pi}$|    <   |
$I_{0}$>|    >   |
       |        |
       $\pi$    $\pi$  \ldots Phasensprünge an den Flächen (wie damals abgeleitet)
Siehe Tafelbild
\[ 2k_{0}n_{1}d=\pi \implies n_{1}d=\frac{\lambda_{0}}{4}\text{\ldots Bragg-Bedingung}\]

Betrachten wir das ganze als Farug-Peru? (keine Ahnung oida), so erhalten wir
\[ T=\frac{(1-\|r_{01}\|^{2})(1-\|r_{1s}\|)^{2}}{1-\|r_{1s}\|\|r_{01}\|^{2}+\|r_{1s}\|^{2}\|r_{01}\|^{2}}=1 \]
\[ n_{1}\sqrt{n_{0}n_{s}}  \]
\[ r=\frac{n_{0}n_{s}-n_{1}^{2}}{n_{0}n_{s}+n_{1}^{2}} \]

Durch multiple Schichten kann eine Total-Reflex-Beschichtung realisiert werden.
Aufbau: $n_{0}n_{H}n_{L}n_{H}n_{L}n_{H}n_{S}$ 
$n_{H}=2.3$, $n_{L}=1.28$, $n_{s}=1.5$
\[ R=[\frac{1-(\frac{n_{L}}{n_{H}})^{2m}}{1+(\frac{n_{L}}{n_{H}})^{2m}}]^{2} \]
Wie im Skriptum zu sehen, ergibt sich eine relative hohe Bandbreite.
\[ \frac{\Delta}{\omega_{g}}=\frac{4}{\pi}\arcsin(\frac{n_{H}-n_{L}}{n_{H}+n_{L}}) \]

\section*{Beugung (engl. Defraction = Defraktion)}
Wir setzen eine Kugelwelle an: $\frac{e^{jkr}}{r}$ 
Punkt $P$ sendet eine Feldstärke $A=T\cdot E_{0}$ aus. Punkt $P$ befindet sich in einem Raum mit Rand  $R$.

Wir summieren die Welle an der Randfläche auf: $\int\int \frac{e^{jkr}}{r}A(x',y')=E(x,y,z)$

Wir wollen jetzt $\frac{e^{jkr}}{r}$ in $z$ ausdrücken:
\[ r=\sqrt{z +(x-x')^{2} + (y-y')^{2}} \]
mittels Taylorreihenentwicklung: $\sqrt{1+u}=1+\frac{u}{2} $
 \[ r=z+\frac{(x-x')^{2}}{2}+\frac{(y-y')^{2}}{2} \]
 \[ E(x,y,z)=\frac{1}{\lambda} irgendwas \text{\ldots Fresnel-Näherung} \]
\[ z\gg \frac{x'^{2}+y'^{2}}{\lambda} \]

Frauenhofer-Nähgerung:
\[ E(x,y,z)=\frac{1}{\lambda z} \\int_{}^{} \int_{}^{} A(x',y')e^{-j \frac{2\pi}{\lambda z}x x'}e^{-j \frac{2\pi}{\lambda z}y y'} dx'dy' \]

mit der "Paraxialen Näherung" $tan(\alpha)=\frac{x'}{z}$ mit $\alpha \approx \frac{x'}{z}$ kann man dann jenes Diagram konstruieren

\end{document}
